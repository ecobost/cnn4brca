\chapter{Solution}
\label{ch:Model}

% We [outline|describe|introduce] our [proposed] [solution|model], [justify] design decisions and [implementation details| detail implementation| document implementation details].
In this chapter, we describe our solution, justify design decisions and detail implementation.


% An scheme of the solution is found there and we do it here...

\section{Task definition}
We segment digital mammograms into two separate regions: breast mass (benign or malignant) and general tissue.
Breast area is previously separated from the background by simple thresholding.
In particular, we train a convolutional network to estimate the probability of each pixel belonging to a mass and use these values to generate a valid segmentation.
%The breast area is separated from the background by simple thresholding.
%Background and breast area are separated using thresholded to zero.

% Maybe order as in
% Experiment 1 (architecture 1)
%		Hyperparamtere search
% Experiment 2 (architecture 1, weighted loss)
%		Hyperparameter search
% Experiment 3 (architecture 2, maybe weighted loss)
%		Hyperparameter search

\section{Data set}
%We document the retrieval, enhancement, labelling and augmentation of [the|our] data set [used for the experiments].
% We document how we retrieve, enhance, label and augment our data set.
We document the retrieval, enhancement and augmentation of our data set.

\subsection{Database}
We use the Breast Cancer Digital Repository (BCDR-DM) database, specifically, the BDCR-D01 data set, which is composed of patients with at least one breast mass.
%We curated it to delete patients with breast implants.
We select 256 digital mammograms from 63 patients.
A patient with breast implants (511) was ignored.
Digital mammograms have higher image quality and lack any marks or scanning artifacts present in digitized film mammograms; this allows the network to learn sharper features easying segmentation.
We overcome having few mammograms at our disposal by augmenting our data set and training on overlapping patches.

Mammograms in the BCDR-D01 data set are 8-bit grayscale images with 0.07mm spatial resolution sized $3328\times 4084$, $2816\times 3072$ or $2560 \times 3328$ pixels equivalent to $23.3 \times 28.6$, $19.7 \times 21.5$ and $17.9 \times 23.3$ centimeters.
%Each lesion's segmentation, type (mass, microcalcification, calcification, axillary adenopathy, architectural distortion or stroma distortion) and biopsy result (benign or malignant) are provided.
The data set provides the segmentation, type (mass, microcalcification, calcification, axillary adenopathy, architectural distortion or stroma distortion) and biopsy result (benign or malignant) of each lesion.
%Segmentation, type and biopsy result for each lesion are provided. 
%Each lesion is segmented and its type (mass, microcalcification, calcification, axillary adenopathy, architectural distortion or stroma distortion) and biopsy result (benign or malignant) are supplied.
We disregard patient data (age and breast density) and image features (intensity, texture, shape and location descriptors).
%do not make use of/employ/ ignore/disregard

We generate our labels thresholding the mammogram to zero to separate the background and using the provided lesion outlines to separate the lesions.
%and separating the background by thresholding to zero. 
%Labels were generated using the provided outlines and thresholding the background to zero.
Masses (benign or malignant) appear as white; breast area as gray and background as black (Fig.~\ref{subfig:Preprocessinga}).

\subsection{Data division}
For each fold, we randomly assign 80\% of patients to the training set and 20\% to the test set~(Tab.~\ref{tab:DataSetSummary}). In total, our data set counts with 63 patients, 256 mammograms and 139 lesions.

\begin{table}[h]
	\centering
	\begin{tabular}{lcccccccc}
		\hline
		& \multicolumn{2}{c}{\textbf{Patients}} & \multicolumn{2}{c}{\textbf{Mammograms}} &\multicolumn{2}{c}{\textbf{Masses}}\\
		& \textbf{Training} & \textbf{Test} & \textbf{Training} & \textbf{Test} & \textbf{Training} & \textbf{Test} \\
		\hline 
		Fold 1	&50	&13	&189	&67	&106	&33\\
		Fold 2	&50	&13	&209	&47	&112	&27\\
		Fold 3	&50	&13	&204	&52	&110	&29\\
		Fold 4	&51	&12	&209	&47	&110	&29\\
		Fold 5	&51	&12 &213	&43	&118	&21\\
		Average &50.6 &12.6 &204.8 &51.2 &111.2 &27.8\\
		\hline
	\end{tabular}
	\caption[Data set summary]{Data set summary}
	\label{tab:DataSetSummary}
\end{table}

\subsection{Image enhancement (Exp. 1.3 and Exp. 3)}
We set to zero any pixel below the mean pixel intensity of the image (calculated only on the breast area) and scale the rest linearly to cover the entire intensity range (0-255); this reduces to black small variations in the background and increases the contrast of the image (Fig.~\ref{subfig:Preprocessingb}).
%Background reduction reduces all small variations in the background to black and linear normalization increases the contrast of the remaining image. 

Background reduction plus contrast normalization highlights breast masses, which are brighter than normal breast tissue; normalizes images from patients with darker or lighter tissue and improves convergence~\cite{Arevalo2016}. However, it may destroy important texture information by blending it with the background or cause false positives by highlighting dense tissue.

\subsection{Resizing}
Our convolutional networks have an effective receptive field, the spatial dimensions around a pixel that affect its prediction, of roughly $128 \times 128$ pixels.
We resize our images to contain $2 \times 2$ cm in this area---roughly a 2.2 downsampling factor~\footnote{We use $96 \times 96$ for Experiment 1, whose network has an smaller receptive field.} (Fig.~\ref{subfig:Preprocessingc}). Considering that masses are rarely bigger than 2cm (length of the long axis)~\cite{Sahiner1996}, the network sees a good portion of the lesion during classification. 

We resize images with PILLOW, the Python Image Library, using Lanczos interpolation for mammograms and nearest neighbor interpolation for labels. Lanczos interpolation is a high quality downsampling filter recommended by PILLOW and nearest neighbor interpolation assures that the reduced label contains only valid values (white, gray and black).

\subsection{Cropping}
We calculate the bounding box of the breast area in the label image and crop the mammogram and label to delete unnecesary black spaces (Fig.~\ref{subfig:Preprocessingd}). 
Because our networks downsample images to later upsample them by the same factor, we ensure that this factor divides the dimensions of the cropped image (cropping a slightly bigger box if needed) to recover the exact dimensions after upsampling.

\begin{figure}[h]
	\centering
	\begin{subfigure}{4.2 cm}
		\centering
                \includegraphics[height = 5cm]{plots/mammogram.png}
    \end{subfigure}
	\begin{subfigure}{4.2 cm}
		\centering
                \includegraphics[height = 5cm]{plots/mammogram_enhanced.png}
    \end{subfigure}
	\begin{subfigure}{4.2 cm}
		\centering
                \includegraphics[height = 5cm]{plots/mammogram_resized.png}
    \end{subfigure}
	\begin{subfigure}{2.4 cm}
		\centering
                \includegraphics[height = 5cm]{plots/mammogram_v1.png}
    \end{subfigure}
	\\
	\begin{subfigure}{4.2 cm}
		\centering
                \includegraphics[height = 5cm]{plots/label.png}
		\caption{Original image}
		\label{subfig:Preprocessinga}
    \end{subfigure}
	\begin{subfigure}{4.2 cm}
		\centering
                \includegraphics[height = 5cm]{plots/label_enhanced.png}
		\caption{Enhancement}
		\label{subfig:Preprocessingb}
    \end{subfigure}
	\begin{subfigure}{4.2 cm}
		\centering
                \includegraphics[height = 5cm]{plots/label_resized.png}
		\caption{Downsampling}
		\label{subfig:Preprocessingc}
    \end{subfigure}
	\begin{subfigure}{2.4 cm}
		\centering
		\includegraphics[height = 5cm]{plots/label_v1.png}
		\caption{Final image}
		\label{subfig:Preprocessingd}
    \end{subfigure}
	\caption[Preprocessing pipeline]{A mammogram (top) and its label (bottom) being preprocessed: (1) original images ($4084 \times 3328$), (2) background reduction plus contrast normalization, (3) downsampling and  (4) cropping to delete black spaces ($560 \times 1424$). Augmentation is not shown.}
	 \label{fig:Preprocessing}
% img_108_146_1_RCC.png
\end{figure}
	
\subsection{Data augmentation}
We mirror each image (mammograms and labels) and rotate the original and mirrored version at 0, 90, 180 and 270 degrees to increase our training set by a factor of 8. 
%Mammograms and labels are mirrored and both the original and mirrored version rotated at 0, 90, 180 and 270 degrees to increase our training and validation set by a factor of 8. 
Images in the test set are not augmented. 

This transformations are common when training convolutional networks with small data sets. In principle, the test set should not be augmented as it is a proxy for real data. %In principle it is not neccesary to store the augmented images because they can be easily generated during training but if the disk space is not prohibitive explicitly storing them simplifies training.

%\subsection{Memory compromises} Hopefully not
% Cutting mammograms in 4 or 16 pieces for training would not be that bad, I would have to cut images surrounded with 48 pixels of surrounding regions so the netwrok also sees that and does not see black spaces, then to the output of the network i have to discard a surrounding region of 3 pixels all around to obtain the segmentation of my wanted image: this is exactkly as if training with the big entire image. This if i have no memory, or maybe if i want to change up a bit.

\subsection{Storage}
All mammograms and their respective labels are stored as grayscale 8-bit images preserving their original names (plus a suffix) and folder organization.
% For training, a file enlisting the names of all the images is also generated.
The entire data set before augmentation weights approximately 120 megabytes.


\section{Model}
We describe our chosen architecture and loss function.

\subsection{Architecture}
%Following recommendations from Section~\ref{sec:PracticalDL}, we define a network with seven convolutional layers and two fully connected layers (Tab.~\ref{tab:convNetArchitecture}). 
We model our architecture on a VGG network~\cite{Simonyan2014}, winner of the 2014 ImageNet competition~(Tab.~\ref{tab:convNetArchitecture}).
\begin{table}[h]
	\centering
	\begin{tabular}{lccccr}
	\hline
	\textbf{Layer} & \textbf{Filter} & \textbf{Stride} &\textbf{Pad} & \textbf{Volume} & \textbf{Parameters} \\
	\hline
	\texttt{INPUT}	& -	& - & - & $112 \times 112 \times 1$ & -\\
	\texttt{CONV -> Leaky RELU} & $6 \times 6$ & 2 & 2 & $56 \times 56 \times 56$ & 2\,072\\
	\texttt{CONV -> Leaky RELU} & $3 \times 3$ & 1 & 1 & $56 \times 56 \times 56$ & 28\,280\\
	\texttt{MAXPOOL} & $2 \times 2$ & 2 & 0 & $28 \times 28 \times 56$ & -\\
	\texttt{CONV -> Leaky RELU} & $3 \times 3$ & 1 & 1 & $28 \times 28 \times 84$ & 42\,420\\
	\texttt{CONV -> Leaky RELU} & $3 \times 3$ & 1 & 1 & $28 \times 28 \times 84$ & 63\,588\\
	\texttt{MAXPOOL} & $2 \times 2$ & 2 & 0 & $14 \times 14 \times 84$ & -\\
	\texttt{CONV -> Leaky RELU} & $3 \times 3$ & 1 & 1 & $14 \times 14 \times 112$ & 84\,784\\
	\texttt{CONV -> Leaky RELU} & $3 \times 3$ & 1 & 1 & $14 \times 14 \times 112$ & 113\,008\\
	\texttt{CONV -> Leaky RELU} & $3 \times 3$ & 1 & 1 & $14 \times 14 \times 112$ & 113\,008\\
	\texttt{MAXPOOL} & $2 \times 2$ & 2 & 0 & $7 \times 7 \times 112$ & -\\
	\texttt{FC -> Leaky RELU} & $7 \times 7$ & 1 & 3 & $7 \times 7 \times 448$ & 2\,459\,072\\
%	\texttt{CONVFC -> Leaky RELU} & $7 \times 7$ & 1 & 3 & $7 \times 7 \times 448$ & 2\,459\,072\\
	\texttt{FC -> SIGMOID} & $1 \times 1$ & 1 & 0 & $7 \times 7 \times 1$ & 449 \\
%	\texttt{CONVFC -> SIGMOID} & $1 \times 1$ & 1 & 0 & $7 \times 7 \times 1$ & 449 \\
%	\texttt{BICUBIC} & - & - & - & $112 \times 112 \times 1$ & -\\
	\hline
	\end{tabular}
	\caption[Selected convolutional network architecture]{Architecture of the network used for experiments. It shows the filter size, stride and padding in each layer as well as the resulting volume and number of learnable parameters per layer.}
	\label{tab:convNetArchitecture}
\end{table}


	The first convolutional layer reduces the spatial dimensions of the input from $112 \times 112$ to $56 \times 56$; this reduces the number of parameters and memory requirements of the network. Subsequent convolutional layers preserve the dimensions of its input volume relegating subsampling to pooling layers. 
Produced segmentations are 16 times smaller than the original images (Sec.~\ref{sec:Segmentation}).

This architecture defines 2.91 million parameters. %2 906 681

% Do i need to explain more why this architecture?. Small enough for our data and gpu memory but big enough for the task and to see enough texture, filters are 3x3 as recommended, two convs before pooling.

% A second architecture modelled on the ResNet~\cite{}, winner of the 2015 ImageNet competition is also used for experiments (Tab~\cite{}).

\subsection{Upsampling}
To simplify the architecture, we downsample our labels to match the produced segmentations rather than adding an upsampling layer at the end of the network.
% Could automatically resize using tf.imageresize_bilinear or bicubic
% change in design decisions and in the layer architecture table above

\subsection{Loss function}
We compute the logistic loss function for each pixel in the produced segmentation and average over all pixels in the breast area, i.e., gradients accumulate over pixels in the breast area (breast tissue and masses) while background is ignored.
%We compute the logistic loss function for each pixel in the produced segmentation and sum over all pixels in the breast area, i.e., gradients accumulate over pixels in the breast area but background is ignored. This amounts to using a weighted loss function where breast area (breast tissue and masses) has weight one and background has weight zero.
% I can weight the losses for breast tissue and breast masses differently using the mask (multiplied by the loss function). To calculate a proper weight, you would have to estimate the prior of a loss being from mass(~0.2) or from area (0.8) and multiply masses by .8 and tissue by 0.2, for instance. Or just decrease the weight of breats tissue, thus masses at 1 and tissue at 0.2 or 0.1 (remember there are images with not a single mass, so mass is quite improbable).


\section{Hyperparameter selection}
We started with 30 networks ranged from 10 unif (-6, 0) for alpha and 10 uniform*-3,3) for lambda. Then refine this tranges to ...

\section{Training}
We offer details about the hardware and software used for experiments, regularization methods, hyperparameters and implementation notes.

\subsection{Hardware}
Training deep neural networks is computationally intensive and requires equipment with powerful GPUs. We enlist here the resources available for this thesis.
\begin{table}[h]
	\centering
	\begin{tabular}{cp{3.8cm}p{1.7cm}p{1.8cm}cc}
	\hline
	\textbf{PC}	& \textbf{GPU}	& \textbf{HD}	& \textbf{CPU}	& \textbf{RAM}	& \textbf{\#} \\
	\hline
	Personal	& Nvidia NVS 5400M \newline 96 cores, 1GB, compute capability 2.1 & 57 GB \newline (30 free)	& i5-3210M \newline 2.5GHz	& 4 GB	& 1 \\
	A4-401	& Nvidia Quadro K620 \newline 384 cores, 2GB, compute capability 5.0 & 240 GB \newline (230 free)	& i5-4570 \newline 3.2GHz	& 8 GB	& 27\\
	HP-Z400	& AMD Gallium 0.4  & 240 GB \newline (230 free)	& Xeon W3530 \newline 2.8GHz	& 2 GB	& 2\\
	\hline
	\end{tabular}
	\caption{Available hardware for experiments}
\end{table}
We did our experiments in x computers from A3-401

\subsection{Software}
We use Tensorflow to train the networks and Python to develop any other tools (image retrieval and augmentation, model evaluation, figure generation, etc.).
Python3, PILLOW, libraries used,etc.
code available at:

\subsection{Regularization}
Each picture was zero-mean centered, (do not subtract 127)
Dropout and l2-norm, batch normalization


\subsection{Hyperparameters}
Learning rate decay, dropout p, ...
0.1 learning rate divided by 10 when convergence

\subsection{Optimization rule}
ADAM
%Although the gradient rule can be seen as stochastic gradient descen because it uses a single image to compute the gradients it  can also be seen as stochastic radient descent given that it sums the gradients produced by applying the network in different spatial positions.

\subsection{Implementation details}
Upsampling
% If using an upsampling layer, maybe use a simple deconv layer, but change its weights in the forward pass to be the weights needed for bilinear interpolation (or set its weights to always be those for bilinear interpolation), that way gradient descent can take care of it in the backward pass
% For interpolations, see http://paulbourke.net/miscellaneous/interpolation/ y las definiciones de bilinear y bicubic above
Loss function backprop weight mask
% Create the mask when you zero-mean the image (may not work).
Weight initialization


\section{Evaluation}
% Make sure the image inputted is grayscale, if not transform it, that way i can deal with color images, too.

\subsection{Post-processing}
Say that the logits out of the network were transformed intoprobabilities via softmax
Probabilities vs UNcorrected threshold vs Threshold and cluster extent correction vs threshold-free cluster enhancement vs, show different results at different thresholds
data is smooth, 

uncorrected threshold
Present the probabilities of tumor on each pixel. anmd threshold it on a given prob nd delete any clusters less than x.
Threshold free cluster enhancement (see module 29 in introduction to fmri)
Choose a threshold and delete any clusters less than x...

we could also present a heatmap with the different probabilities in each pixel of the image, for instance on mammograms we could present a grayscale image of whether a lesion is present.

% CRFs: public available implementation of (Krahenbuhl & Koltun)
% For the figure draw a random one  of 100x100 with a little signal in there and apply all corrections.

% Python implementation of CRF: https://pystruct.github.io/auto_examples/image_segmentation.html

% If using the threshold/ value under the cluster one, use the validation set t select pairs(threshold, total acceptable value)

% Background is thresholded at zero and shown in black

% Background is not counted to calculate the metrics. We use a mask to sum values only over the spaces that were a breast mass previously.

	\subsection{Evaluation}
	When dividing the data set we make sure \textit{all} image patches obtained from the same patient are assigned to either the training set or test set (not distributed) to avoid any possible overfit to the test set. Given that our data is unbalanced, with far more negative than positive examples, we use PRAUC (see Section~\ref{subsec:Classification}) to choose between models for hyperparameter selection and as an overall performance metric. Other metrics are also reported for completeness. 

	We could also evaluate the network on all augmentations of an image and output the average prediction; in theory, this would give us better results. For simplicity, we do not apply it for model selection.

	For detection of lesions on entire mammograms we slide the trained convolutional network across the mammogram computing a per-pixel prediction. The generated heatmap preserves the size of the original mammogram (with some zero-padding) and can be presented side to side to the original mammogram as a CAD system. In case this heatmap is noisy (predictions changes abruptly from pixel to pixel) we could use a median or gaussian filter to smooth it out.% We do not evaluate the network on the entire mammogram (or per patient), we limit ourselves to show the results.
	\subsection{Evaluation metrics}
F1-score AUc and those things are calculated only over breast tissue and lessions.

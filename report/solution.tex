\chapter{Solution}
\label{ch:Model}

% We [outline|describe|introduce] our [proposed] [solution|model], [justify] design decisions and [implementation details| detail implementation| document implementation details].
In this chapter, we describe our experiments, justify design decisions and detail our implementation.

% An scheme of the solution is found there and we do it here...

\section{Task definition}
We segment digital mammograms into two separate regions: breast mass (benign or malignant) and general tissue.
Breast area is previously separated from the background by simple thresholding.
In particular, we train a convolutional network to estimate the probability of each pixel belonging to a mass and use these predictions to generate a valid segmentation.

\section{Data set}
%We document the retrieval, enhancement, labelling and augmentation of [the|our] data set [used for the experiments].
% We document how we retrieve, enhance, label and augment our data set.
We document the retrieval, enhancement and augmentation of our data set.

\subsection{Database}
We use the Breast Cancer Digital Repository (BCDR-DM) database, specifically, the BDCR-D01 data set, which is composed of patients with at least one breast mass.
%We curated it to delete patients with breast implants.
We select 256 digital mammograms from 63 patients.
A patient with breast implants (511) was ignored.
Digital mammograms have higher image quality and lack any marks or scanning artifacts present in digitized film mammograms; this allows the network to learn sharper features easying segmentation.
We overcome having few mammograms at our disposal by augmenting our data set and training on overlapping patches.

Mammograms in the BCDR-D01 data set are 8-bit grayscale images with 0.07mm spatial resolution sized $3328\times 4084$, $2816\times 3072$ or $2560 \times 3328$ pixels equivalent to $23.3 \times 28.6$, $19.7 \times 21.5$ and $17.9 \times 23.3$ centimeters.
%Each lesion's segmentation, type (mass, microcalcification, calcification, axillary adenopathy, architectural distortion or stroma distortion) and biopsy result (benign or malignant) are provided.
The data set provides the segmentation, type (mass, microcalcification, calcification, axillary adenopathy, architectural distortion or stroma distortion) and biopsy result (benign or malignant) of each lesion.
%Segmentation, type and biopsy result for each lesion are provided. 
%Each lesion is segmented and its type (mass, microcalcification, calcification, axillary adenopathy, architectural distortion or stroma distortion) and biopsy result (benign or malignant) are supplied.
We disregard patient data (age and breast density) and image features (intensity, texture, shape and location descriptors).
%do not make use of/employ/ ignore/disregard

We generate our labels thresholding the mammogram to zero to separate the background and using the provided lesion outlines to separate the lesions.
%and separating the background by thresholding to zero. 
%Labels were generated using the provided outlines and thresholding the background to zero.
Masses (benign or malignant) appear as white; breast area as gray and background as black (Fig.~\ref{subfig:Preprocessinga}).

\subsection{Data division}
For each fold, we randomly assign 80\% of patients to the training set and 20\% to the test set~(Tab.~\ref{tab:DataSetSummary}). In total, our data set counts with 63 patients, 256 mammograms and 139 lesions.

\begin{table}[h]
	\centering
	\begin{tabular}{lcccccccc}
		\hline
		& \multicolumn{2}{c}{\textbf{Patients}} & \multicolumn{2}{c}{\textbf{Mammograms}} &\multicolumn{2}{c}{\textbf{Masses}}\\
		& \textbf{Training} & \textbf{Test} & \textbf{Training} & \textbf{Test} & \textbf{Training} & \textbf{Test} \\
		\hline 
		Fold 1	&50	&13	&189	&67	&106	&33\\
		Fold 2	&50	&13	&209	&47	&112	&27\\
		Fold 3	&50	&13	&204	&52	&110	&29\\
		Fold 4	&51	&12	&209	&47	&110	&29\\
		Fold 5	&51	&12 &213	&43	&118	&21\\
		Average &50.6 &12.6 &204.8 &51.2 &111.2 &27.8\\
		\hline
	\end{tabular}
	\caption[Data set summary]{Data set summary}
	\label{tab:DataSetSummary}
\end{table}

\subsection{Image enhancement (Exp. 1.3 and Exp. 3)}
We set to zero any pixel below the mean pixel intensity of the image (calculated only on the breast area) and scale the rest linearly to cover the entire intensity range (0-255); this reduces to black small variations in the background and increases the contrast of the image (Fig.~\ref{subfig:Preprocessingb}).
%Background reduction reduces all small variations in the background to black and linear normalization increases the contrast of the remaining image. 

Background reduction plus contrast normalization highlights breast masses, which are brighter than normal breast tissue; normalizes images from patients with darker or lighter tissue and improves convergence~\cite{Arevalo2016}. However, it may destroy important texture information by blending it with the background or cause false positives by highlighting dense tissue.

\subsection{Resizing}
Our convolutional networks have an effective receptive field, the spatial dimensions around a pixel that affect its prediction, of roughly $128 \times 128$ pixels.
We resize our images to contain $2 \times 2$ cm in this area---roughly a 2.2 downsampling factor~\footnote{We use $96 \times 96$ for Experiment 1, whose network has an smaller receptive field.} (Fig.~\ref{subfig:Preprocessingc}). Considering that masses are rarely bigger than 2cm (length of the long axis)~\cite{Sahiner1996}, the network sees a good portion of the lesion during classification. 

We resize images with PILLOW, the Python Image Library, using Lanczos interpolation for mammograms and nearest neighbor interpolation for labels. Lanczos interpolation is a high quality downsampling filter recommended by PILLOW and nearest neighbor interpolation assures that the reduced label contains only valid values (white, gray and black).

\subsection{Cropping}
We calculate the bounding box of the breast area in the label image and crop the mammogram and label to delete unnecesary black spaces (Fig.~\ref{subfig:Preprocessingd}). 
Because our networks downsample images to later upsample them by the same factor, we ensure that this factor divides the dimensions of the cropped image (cropping a slightly bigger box if needed) to recover the exact dimensions after upsampling.

\begin{figure}[h]
	\centering
	\begin{subfigure}{4.2 cm}
		\centering
                \includegraphics[height = 5cm]{plots/mammogram.png}
    \end{subfigure}
	\begin{subfigure}{4.2 cm}
		\centering
                \includegraphics[height = 5cm]{plots/mammogram_enhanced.png}
    \end{subfigure}
	\begin{subfigure}{4.2 cm}
		\centering
                \includegraphics[height = 5cm]{plots/mammogram_resized.png}
    \end{subfigure}
	\begin{subfigure}{2.4 cm}
		\centering
                \includegraphics[height = 5cm]{plots/mammogram_v1.png}
    \end{subfigure}
	\\
	\begin{subfigure}{4.2 cm}
		\centering
                \includegraphics[height = 5cm]{plots/label.png}
		\caption{Original image}
		\label{subfig:Preprocessinga}
    \end{subfigure}
	\begin{subfigure}{4.2 cm}
		\centering
                \includegraphics[height = 5cm]{plots/label_enhanced.png}
		\caption{Enhancement}
		\label{subfig:Preprocessingb}
    \end{subfigure}
	\begin{subfigure}{4.2 cm}
		\centering
                \includegraphics[height = 5cm]{plots/label_resized.png}
		\caption{Downsampling}
		\label{subfig:Preprocessingc}
    \end{subfigure}
	\begin{subfigure}{2.4 cm}
		\centering
		\includegraphics[height = 5cm]{plots/label_v1.png}
		\caption{Final image}
		\label{subfig:Preprocessingd}
    \end{subfigure}
	\caption[Preprocessing pipeline]{A mammogram (top) and its label (bottom) being preprocessed: (1) original images ($4084 \times 3328$), (2) background reduction plus contrast normalization, (3) downsampling and  (4) cropping to delete black spaces ($560 \times 1424$). Augmentation is not shown.}
	 \label{fig:Preprocessing}
% img_108_146_1_RCC.png
\end{figure}
	
\subsection{Data augmentation}
We mirror each image (mammograms and labels) and rotate the original and mirrored version at 0, 90, 180 and 270 degrees to increase our training set by a factor of 8. 
%Mammograms and labels are mirrored and both the original and mirrored version rotated at 0, 90, 180 and 270 degrees to increase our training and validation set by a factor of 8. 
Images in the test set are not augmented. 

This transformations are common when training convolutional networks with small data sets. In principle, the test set should not be augmented as it is a proxy for real data. %In principle it is not neccesary to store the augmented images because they can be easily generated during training but if the disk space is not prohibitive explicitly storing them simplifies training.

%\subsection{Memory compromises} Hopefully not
% Cutting mammograms in 4 or 16 pieces for training would not be that bad, I would have to cut images surrounded with 48 pixels of surrounding regions so the netwrok also sees that and does not see black spaces, then to the output of the network i have to discard a surrounding region of 3 pixels all around to obtain the segmentation of my wanted image: this is exactkly as if training with the big entire image. This if i have no memory, or maybe if i want to change up a bit.

\subsection{Storage}
All mammograms and their respective labels are stored as grayscale 8-bit images preserving their original names (plus a suffix) and folder organization.
% For training, a file enlisting the names of all the images is also generated.
The entire data set before augmentation weights approximately 120 megabytes.


\section{Experiment 1}
\label{sec:Experiment1}
We use a simple architecture to have a baseline performance for convolutional networks and test the effect of using a weighted loss function and enhancing the input images.

\subsection{Architecture}
\begin{table}[h!]
	\centering
	\begin{tabular}{lcccccr}
	\hline
	\textbf{Layer} & \textbf{Filter} & \textbf{Stride} & \textbf{Pad} & \textbf{Dilation} & \textbf{Volume} & \textbf{Parameters} \\
	\hline
	\texttt{INPUT}	&- & -	& - & - & $52 \times 52 \times 1$ & -\\
	\texttt{CONV -> RELU}	& $5 \times 5$ & 2 & 2 & 1 & $26 \times 26 \times 32$ & 832\\
	\texttt{CONV -> RELU}	& $3 \times 3$ & 1 & 1 & 1 & $26 \times 26 \times 32$ & 9\,248\\
	\texttt{CONV -> RELU}	& $3 \times 3$ & 2 & 1 & 1 & $13 \times 13 \times 64$ & 18\,496\\
	\texttt{CONV -> RELU}	& $3 \times 3$ & 1 & 1 & 1 & $13 \times 13 \times 64$ & 36\,928\\
	\texttt{CONV -> RELU}	& $3 \times 3$ & 1 & 2 & 2 & $13 \times 13 \times 96$ & 55\,392\\
	\texttt{CONV -> RELU}	& $3 \times 3$ & 1 & 2 & 2 & $13 \times 13 \times 96$ & 83\,040\\
	\texttt{CONV}			& $5 \times 5$ & 1 & 6 & 3 & $13 \times 13 \times 1$ & 2\,401\\
	\texttt{BILINEAR (x4)}	& - & - && - & $52 \times 52 \times 1$ & -\\
	\hline
	\end{tabular}
	\caption[Convolutional network architecture for Experiment 1]{Architecture of the network used for the first experiments. It shows the filter size, stride, padding and dilation in each layer as well as the resulting volume and number of learnable parameters per layer.}
	\label{tab:convNetArchitecture1}
\end{table} % 206 337 params

Each image is whitened (zero-mean centered and divided by its standard deviation) individually before being input. 
The first convolutional layer reduces the spatial dimensions of the input from $52 \times 52$ to $26 \times 26$ to reduce the number of parameters and memory requirements while augmenting its receptive field. Subsequent convolutional layers preserve the dimensions of its input volume relegating subsampling to pooling layers. 
Finally, the volume is upsampled by a factor of 4 to recover the dimensions of the original image (Sec.~\ref{sec:Segmentation}). The network outputs a heatmap of logits with the same size as the input image.
%We produce a heatmap of logits rather than probabilities to improve numerical stability.

The effective receptive field of the network is $101 \times 101$ pixels that equates to $2.1 \times 2.1$ cms. This architecture uses 206\,337 parameters. %2 906 681

\subsection{Regularization}
We use l2-norm regularization with $\lambda$ selected using a validation set as explained in Sec.~\ref{subsec:Hyperparameters}.

\subsection{Loss function}
We compute the logistic loss function for each pixel in the produced segmentation and average the loss over pixels in the breast area---background is ignored. This amounts to using a weighted loss function where errors in the breast tissue and masses are weighted by one and those in the background are weighted by zero.

\subsection{Experiments}
We performed three experiments using this architecture:
\subsubsection{Experiment 1.1} 
To obtain a performance baseline, we trained this simple network in minimally processed images, i.e., mammograms without any enhancement.

\subsubsection{Experiment 1.2} 
\label{subsec:Experiment1_2}
To fight the class imbalance produced by breast mass pixels being rare in comparison to normal breast tissue pixels, we modify the loss function by weighting the losses over masses by 15, those over normal breast tissue by 1 and ignoring those over the background. Assigning a higher value to errors on breast masses forces the network to invest more resources in correctly classifying masses, thus learning better features to avoid this costly errors~\cite{Provost2000}. As in Experiment 1.1, we use input without enhancement
%This technique balances the total sum of losses from the common class (normal breast tissue) with that of the rare class (breast masses) and is often used to fight class imbalance.

\subsubsection{Experiment 1.3}
In the final experiment, we combine both a weighted loss function and enhanced input images.


\section{Experiment 2}
\label{sec:Experiment2}
Given the unsatisfactory results from our first experiment, we modify our model to tackle a possible cause: class imbalance. Architecture and regularization are preserved as described in Sections~\ref{subsec:Architecture1} and~\ref{subsec:Regularization1}.

\subsection{Loss function}
We compute the logistic loss function on each pixel in the image, weight the losses over masses by 0.9 and normal breast tissue by 0.1 and average over all pixels in the breast area---background is ignored. Assigning a higher value to errors on breast masses forces the network to invest more resources in correctly classifying masses---hopefully learning better features for this task.
%---to avoid costly errors. 
This technique balances the total sum of losses from the common class (normal breast tissue) with that of the rare class (breast masses) and is often used to fight class imbalance~\cite{Provost2000}.

\subsection{Hyperparameter search}
We trained 30 networks with combinations of $\alpha = 10^{unif(-6, -1)}$ and $\lambda = 10^{unif(-4, 1)}$ for two epochs and results were evaluated by computing IOU in the validation set.


\section{Experiment 3}
\label{sec:Experiment3}
To investigate whether negative results are the product of a poorly chosen architecture we designed a different network that is deeper but has fewer parameters than the one used in previous experiments. We use the weighted loss function defined in Section~\ref{subsec:LossFunction2}

\subsection{Architecture}
We model the architecture on the Residual network~\cite{He2015b}, winner of the 2015 ImageNet competition(Tab~\ref{tab:convNetArchitecture3}).

% Arch #28
\begin{table}[h]
	\centering
	\begin{tabular}{lcccccr}
	\hline
	\textbf{Layer} & \textbf{Filter} & \textbf{Stride} & \textbf{Pad} & \textbf{Dilation} & \textbf{Volume} & \textbf{Parameters} \\
	\hline
	\texttt{INPUT}	&- & -	& - & - & $128 \times 128 \times 1$ & -\\
	\texttt{CONV -> LRELU}	& $6 \times 6$ & 2 & 2 & 1 & $64 \times 64 \times 32$ & 1\,184\\
	\texttt{CONV -> LRELU}	& $3 \times 3$ & 1 & 1 & 1 & $64 \times 64 \times 32$ & 9\,248\\
	\texttt{CONV -> LRELU}	& $3 \times 3$ & 2 & 1 & 1 & $32 \times 32 \times 64$ & 18\,496\\
	\texttt{CONV -> LRELU}	& $3 \times 3$ & 1 & 1 & 1 & $32 \times 32 \times 64$ & 36\,928\\
	\texttt{CONV -> LRELU}	& $3 \times 3$ & 1 & 2 & 2 & $32 \times 32 \times 128$ & 73\,856\\
	\texttt{CONV -> LRELU}	& $3 \times 3$ & 1 & 2 & 2 & $32 \times 32 \times 128$ & 147\,584\\
	\texttt{CONV -> LRELU}	& $3 \times 3$ & 1 & 2 & 2 & $32 \times 32 \times 128$ & 147\,584\\
	\texttt{CONV -> LRELU}	& $3 \times 3$ & 1 & 2 & 2 & $32 \times 32 \times 128$ & 147\,584\\
	\texttt{CONV -> LRELU}	& $3 \times 3$ & 1 & 4 & 4 & $32 \times 32 \times 256$ & 295\,168\\
	\texttt{CONV}	& $8 \times 8$ & 1 & 14 & 4 & $32 \times 32 \times 1$ & 16\,385\\
	\texttt{BILINEAR (x4)}		& - & - && - & $128 \times 128 \times 1$ & -\\
	\hline
	\end{tabular}
	\caption[Convolutional network architecture for Experiment 3]{Architecture of the network used for experiments. It shows the filter size, stride, dilation and padding in each layer as well as the resulting volume and number of learnable parameters per layer. \texttt{LRELU} stands for leaky ReLU.}
	\label{tab:convNetArchitecture3}
\end{table}

\begin{comment}
\begin{table}[h]
	\centering
	\begin{tabular}{lcccccr}
	\hline
	\textbf{Layer} & \textbf{Filter} & \textbf{Stride} & \textbf{Pad} & \textbf{Dilation} & \textbf{Volume} & \textbf{Parameters} \\ % Receptive field 
	\hline
	\texttt{INPUT}	&- & -	& - & - & $128 \times 128 \times 1$ & -\\
	\texttt{CONV -> LRELU}	& $6 \times 6$ & 2 & 2 & 1 & $64 \times 64 \times 32$ & 1\,184\\ % 6x6
	\texttt{CONV -> LRELU}	& $3 \times 3$ & 1 & 1 & 1 & $64 \times 64 \times 32$ & 9\,248\\ % 8x8
	\texttt{CONV -> LRELU}	& $3 \times 3$ & 1 & 2 & 2 & $64 \times 64 \times 64$ & 18\,496\\ % 12x12
	\texttt{CONV -> LRELU}	& $3 \times 3$ & 1 & 2 & 2 & $64 \times 64 \times 64$ & 36\,928\\ % 16x16
	\texttt{CONV -> LRELU}	& $3 \times 3$ & 1 & 4 & 4 & $64 \times 64 \times 128$ & 73\,856\\ % 24x24
	\texttt{CONV -> LRELU}	& $3 \times 3$ & 1 & 4 & 4 & $64 \times 64 \times 128$ & 147\,584\\ % 32x32
	\texttt{CONV -> LRELU}	& $3 \times 3$ & 1 & 4 & 4 & $64 \times 64 \times 128$ & 147\,584\\ % 40x40
	\texttt{CONV -> LRELU}	& $3 \times 3$ & 1 & 4 & 4 & $64 \times 64 \times 128$ & 147\,584\\ % 48x48
	\texttt{CONV -> LRELU}	& $3 \times 3$ & 1 & 8 & 8 & $64 \times 64 \times 256$ & 295\,168\\ % 64x64
	\texttt{CONV}	& $8 \times 8$ & 1 & 28 & 8 & $64 \times 64 \times 1$ & 16\,385\\ % 124x124
	\texttt{BILINEAR (x2)}		& - & - && - & $128 \times 128 \times 1$ & -\\
	\hline
	\end{tabular}
	\caption[Convolutional network architecture for Experiment 3]{Architecture of the network used for experiments. It shows the filter size, stride, dilation and padding in each layer as well as the resulting volume and number of learnable parameters per layer. \texttt{LRELU} stands for leaky ReLU.}
	\label{tab:convNetArchitecture3}
\end{table}
\end{comment}

	 We replace pooling by strided convolutions~\cite{Szegedy2014} or dilated convolutions~\cite{Chen2015}. Input images are whitened independently and passed to the network that produces the predicted segmentation. Because the volume is downsampled only by a factor of four before the upsampling layer, we are able to produce fine-grained segmentations.
	 
	 This 10-layer architecture uses 894\,017 parameters.
	
\subsection{Regularization}
We use l2-norm regularization ($\lambda$ chosen using the validation set) and dropout after RELU layers with probabilities 0.9, 0.9, 0.8, 0.8, 0.7, 0.7, 0.7, 0.7 and 0.6.

\subsection{Hyperparameter search}
We selected $\alpha$ and $\lambda$ using the validation set. We trained 30 networks with sampled $\alpha$ from $10^{unif(-6, -2)}$ and $\lambda$ from $10^{unif(-5, 0)}$; each network was trained for two epochs.


\section{Training}
We offer details about the hardware and software used for experiments, regularization methods, hyperparameters and implementation notes.

\subsection{Hardware}
Training deep neural networks is computationally intensive and requires equipment with powerful GPUs. We enlist here the resources available for this thesis.
\begin{table}[h]
	\centering
	\begin{tabular}{cp{3.8cm}p{1.7cm}p{1.8cm}cc}
	\hline
	\textbf{PC}	& \textbf{GPU}	& \textbf{HD}	& \textbf{CPU}	& \textbf{RAM}	& \textbf{\#} \\
	\hline
	Personal	& Nvidia NVS 5400M \newline 96 cores, 1GB, compute capability 2.1 & 57 GB \newline (30 free)	& i5-3210M \newline 2.5GHz	& 4 GB	& 1 \\
	A4-401	& Nvidia Quadro K620 \newline 384 cores, 2GB, compute capability 5.0 & 240 GB \newline (230 free)	& i5-4570 \newline 3.2GHz	& 8 GB	& 27\\
	HP-Z400	& AMD Gallium 0.4  & 240 GB \newline (230 free)	& Xeon W3530 \newline 2.8GHz	& 2 GB	& 2\\
	\hline
	\end{tabular}
	\caption{Available hardware for experiments}
\end{table}
We did our experiments in x computers from A3-401

\subsection{Software}
We use Tensorflow to train the networks and Python to develop any other tools (image retrieval and augmentation, model evaluation, figure generation, etc.).
Python3, PILLOW, libraries used,etc.
code available at:

\subsection{Regularization}
Each picture was zero-mean centered, (do not subtract 127)
Dropout and l2-norm, batch normalization


\subsection{Hyperparameters}
Learning rate decay, dropout p, ...
0.1 learning rate divided by 10 when convergence

\subsection{Optimization rule}
ADAM
%Although the gradient rule can be seen as stochastic gradient descen because it uses a single image to compute the gradients it  can also be seen as stochastic radient descent given that it sums the gradients produced by applying the network in different spatial positions.

\subsection{Implementation details}
Upsampling
% If using an upsampling layer, maybe use a simple deconv layer, but change its weights in the forward pass to be the weights needed for bilinear interpolation (or set its weights to always be those for bilinear interpolation), that way gradient descent can take care of it in the backward pass
% For interpolations, see http://paulbourke.net/miscellaneous/interpolation/ y las definiciones de bilinear y bicubic above
Loss function backprop weight mask
% Create the mask when you zero-mean the image (may not work).
Weight initialization


\section{Evaluation}
We describe how we evaluate our results.

\subsection{Post-processing}
We aim to evaluate convolutional networks as a single end-to-end segmentation model;
%component in an end-to-end segmentation task;
thus, we choose to use the produced heatmaps without any post-processing. However, adding post-processing to our best performing architecture will certainly improve results and remains as a viable future endeavor.% In any case, having a strong network to start with is needed to produce good results.

% Options:
% uncorrected threshold: Select a threshold, take that as segementation
% Cluster-extent correction: Delete thresholds that are under a number of pixels 
% or whose total probability is less than a desired number. Needs this number
% Threshold-free cluster enhancement (module 29 in introduction to fmri): Same as aove but withouth a cluster (?)
% Number of clusters per image: do threshold-free enhancement where the metric is to 
% leave only the most promising cluster
% Conditional random fields (CRF): Work quite well. Python implementation of CRF: https://pystruct.github.io/auto_examples/image_segmentation.html Here is another take: http://www.robots.ox.ac.uk/~szheng/CRFasRNN.html Any look fine.

\subsection{Segmentation}
We generate a segmentation by setting each pixel whose value was zero in the original mammogram to background (0), each non-background pixel whose logit is greater than a threshold to breast mass (255) and any remaining pixel to normal breast tissue (127) (Fig.~\ref{fig:Post-processing}).

\begin{figure}[h]
	\centering
	\begin{subfigure}{0.2\textwidth}
		\centering
                \includegraphics[width=\textwidth]{plots/probs.png}
         \caption{Prediction}
	\end{subfigure}
	\quad
	\begin{subfigure}{0.2\textwidth}
		\centering
                \includegraphics[width=\textwidth]{plots/segmentation.png}
         \caption{Segmentation}
	\end{subfigure}
	\caption[Post-processing pipeline]{Heatmap of probabilities and segmentation produced by assigning all background to black and thresholding at probability 0.5.}
	 \label{fig:Post-processing}
\end{figure}
% Using the validation set, we compute the IOU for different thresholds and select the one that produces the best result. The final segmentation is generated by setting each pixel whose value was zero in the original mammogram to background (0), each non-background pixel whose logit is greater than the threshold to breast mass (255) and any remaining pixel to normal breast tissue (127).

\subsection{Metrics}
We use five-fold cross-validation and the free-response ROC curve to evaluate our models. We count a breast mass as a true positive if a blob in the segmentation covers at least 10\% of its area. To avoid any bias, we compute the number of false positives per image only in the images without a breast mass~\cite{Chakraborty2013}. Furthermore, we force the number of false positives in an image to be non-decreasing for lower, i.e., laxer thresholds.


\section{Summary}
Mammograms from the BCDR-D01 database were enhanced, resized and divided to obtain our data set. An architecture based on the VGG network (9 layers, 2.9M parameters) was used for the first experiments, one of which used a weighted loss function to fight class imbalance. A different architecture based on the Residual network (10 layers, 0.9M parameters) was used for the last experiment. We performed hyperparameter search to fine tune the learning rate and regularization parameter of each network; other hyperparameters were set manually. Networks were written in TensorFlow and optimized using ADAM. Using a validation set, we selected a threshold to produce the final segmentation from the heatmap of predictions. Many different performance metrics are reported.

%[[Image] segmentation [is the process of|is the task of] partitions [an|the] image into [multiple [disjoint]|disjoint] regions[, essentially [labelling each pixel as belonging to a class from a predefined set of classes|assigning a [class|label] to each pixel. [in the image]]  |Partitioning an image into multiple regions is called image segmentation.][[[In machine learning,] this reduces to|This is equivalent to| similarly,] [classifying|assigning] each pixel [into|as part of| as belonging to|to] a [class from a] given set of classes]] For instance, each pixel in a street image could be classified as [part of a] road, building, tree, pedestrian or bicycle.
Image segmentation partitions an image into multiple regions, essentially assigning a class to every pixel in the image; for instance, classifying each pixel in a street image as road, building, sky, tree, car, pedestrian, bycicle or background. Lesion segmentation is tasked with separating lesions from normal tissue in medical images. When searching for abnormal findings, radiologists perform lesion segmentation---although implicitly. Traditional CAD systems for lesion segmentation are based on computer vision methods that are often convoluted and hard-to-adapt~\footnote{See~\cite{Ashraf2013} for an example.}. 

Despite their widespread use and relative success, various limitations should be address to further advance the field:
\begin{itemize}
	\item Lack of standard preprocessing techniques. Some techniques are commonly used but their performance can vary.
	\item Handcrafted features. The features extracted from the image are chosen beforehand (maybe designed with the help of experts) and special filters and image techniques are used to extract them.
	\item Expertise needs. They require knowledge in various fields such as radiology, oncology, image processing, computer vision, machine learning, etc.
	\item Pipeline structure. Systems are composed of many sequential steps. At each stage, the researcher chooses among many techniques and estimates many parameters representing a cost in time and performance as it is improbable to achieve an optimal combination.
	\item Low ceiling. Techniques are already complex and require much work to achieve only incremental improvements.
	\item Complexity. Issues such as non-desired or unknown dependencies between subsystems, difficulty to localize errors and maintainability could arise. 
\end{itemize}

In this thesis, we use convolutional networks, a recent development in Machine Learning, (see Section~\ref{sec:ConvNets}) to remedy some of these problems. In particular, we simplify the system pipeline by using a single end-to-end trainable model that learns the relevant preprocessing and image features from raw data. We can also focus on improving the learning mechanism, both the model and algorithm, rather than working on designing novel image features or improving specific subsystems.

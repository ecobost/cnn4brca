\chapter{Results}
\label{ch:results}
%Choose what is this experiment: The network could be slided across an image. Options: (1) a network for detection of microcalcification and one for masses (and slide both across and plot their results with differerent colors) (2) a network for diagnosis of microcalc and one for masses(slide them both) and (3) one that detects micro+mass vs non-lession (stanford guys did bad with this one) and (4) one that detects any lession(micro+mass+other) vs no lession and (5) one that also detects more than one network but has multiple output. 
	Task selected. Architecture selected. Hyperparameters selected. results and discussion.
We do this as described in chapter 3.

\section{Experiment 1}

\subsection{Hyperparameter search}
The best resulting alpha and lambda where closer to the bottom of their range. Lambda in range 0-0.15 and alpha in range 0.000001 to 0.0001. Add plots.
For val loss as long as alpha is small it will go down, no matter whether lambda is big or not but a big lambda makes it converge to all negative values.
try alpha in 10unif(-6, -4), lambda in 10 unif(-5, -1)

\subsection{Evaluation}
The best performing architecture had that and that

\subsection{Qualitatiuve results}
images of actual segmentations in the test set

\section{Experiment 2}
\subsection{Hyperparameter search}
\subsection{Evaluation}
\subsection{Qualitative results}


\section{Discussion}
We found that....

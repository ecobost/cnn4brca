Yet to write 
\begin{comment} Goes in thesis, not here because we still don't have results or discussion.
Hook(first paragarpah) : automatic breast cancer detection is a difficult taks for current automatic computationnal systems. In this article we apply deep learning techniques to digitaml mammogrpahic images and obtain better rredults than opresneted to date. 
\end{comment}

\begin{comment}
Cance is this and this, bereast cancer is this and this. breats cancer has high incicdence and high mortality,

Breast cancer has the highest incidence of any cancer in the United States, an estimated 14.1\% of all new cancer cases in 2015 will be breast cancer and the third highest mortality of any cancer accounting for 6.9\% of cancer related deaths \cite{http://www.cancer.org/acs/groups/content/@editorial/documents/document/acspc-044552.pdf}. Among women it is by large the most commonly diagnosed cancer (28.6\% of all cancers) and has the highest death rate (14.5\%) besides lung cancer.\cite{http://www.cancer.org/acs/groups/content/@editorial/documents/document/acspc-044552.pdf}. 
mammograms are this and radiologist do that
all the other things from problem definition (problem definition section does not exist on thesis just on plan. )
Michael: aqui comienza lo nuevo
This work centers on using mammogrmas (2d Xray images of the breast) to detect microcalcifications and masses and diagnos its likelihood of breast cancer, i.e, make a prediction on the probability that a mmaogram signals breast cancer. this is of importance because of ... 
this is withr the view of form part odf abigger CAd project developed in the insitution that could help radiologists in the detection of breast cancer either by proposing zones where the radiologists could take a second look, or by giving a second informed opinion on the same patient.

(why convnets?)
convnets are a natural extension to artificial neural networks that deal with images and that
Summary of the related work part here(this guys dd it first and this guys and i'm gteting my ideas from here and there and my contribution is this...)
We will do this and this and that (methods and plan of work)
Goals: We want to improve over the published results, ... and know if we can a ctuallly use this  
\end{comment}

\begin{comment}
En esta sección se espera que el autor describa en forma más amplia a como
se presentó en el {\it Resumen}, algunos aspectos como el contexto de la
investigación, el problema a resolver, la forma propuesta de resolver, algunos
antecedentes, entre otros.

Los puntos importantes en la {\it Introducción} son:
\begin{itemize}
	\item Introducción al contexto donde se va a realizar la propuesta
	\item Determinar la situación problemática 
	\item Definir el problema y los factores y aspectos más importantes que intervienen  en el problema
	\item Justificar por qué es importante resolver ese problema
 Michael: Hasta aqui es pareciod a lo de definicion de problema. M
	\item Explicar lo qué se ha hecho para resolver ese problema
	\item Describir el modelo de solución del problema
	\item Establecer los posibles logros en la solución del problema
	\item Describir la organización del documento
\end{itemize}

{\bf Ejemplo de Cita Bibliográfica:}

En años recientes se ha manifestado interés en el área de Algoritmos Genéticos
con la Teoría de Dificultad
Walsh polynomials \cite{Clear2}....... \\


{\bf No olviden incluir en donde corresponde
 la motivación, la justificación, el alcance de la
investigación, los recursos y las suposiciones...}
\end{comment}

%%%%%%%%%%%%%%%%%%%%%%%%%%%%%%%%%%%%%%%%%%%%%%%%%%%%%%%%
% Written by: Erick Cobos Tandazo (a01184587@itesm.mx)
% Date: 7-April-2014
%
% Project proposal for my Master's thesis
%%%%%%%%%%%%%%%%%%%%%%%%%%%%%%%%%%%%%%%%%%%%%%%%%%%%%%%%%

\documentclass[11pt]{article}

% Packages
\usepackage[utf8]{inputenc}	% Spanish accents
\usepackage{proposal} 		% Title pages and overall format
\usepackage{verbatim} 		% Block Comm\usepackage{textcomp}ents
\usepackage{textcomp}		% For the degree sign(°)
\usepackage{hyperref}		% To manage \url and citations as links to the reference part.
\usepackage{subcaption}		% For subfigure in breastcancer.tex


% Set properties of the document
\propautor{Erick Michael Cobos Tandazo}	% Author name
\propautormat{1184587}			% Author ID number
\propmes{April 7}			% Date (month)
\propanio{2015}				% Year
\propcd{Monterrey, N.L.}		% Place
\proptitulo{Early Detection and Diagnosis of Breast Cancer Lessions using Deep Convolutional Networks in Digital Mammograms}	% Title
\propasesor{Dr.~Hugo Terashima Mar\'{i}n}	% Advisor
\propsinodalA{Por definir}		% First supervisor
\propsinodalB{Por definir}		% Second supervisor.
\propdirectorPG{Dr.~Ram\'{o}n Brena Pinero}	% Program Director
\propPG{Ingenier\'{i}a y Ciencias}		% Program Departnment
\propcampus{Monterrey}			% University Campus
\propgrado{Master in Science}		% Academic degree
\propgradosiglas{M.Sc.}			% Academic degree abbreviation
\propespecialidad{Intelligent Systems}	% Subject
%%%%%%%%%%%%%%%%%%%%%%%%%%%%%%%%%%%%%%%%%%%%%%%%%%%%%%%%%%%%%%%%%%%%%%%%%%%%%



\begin{document}

\propportada                        % Genera la portada.
\proppagfirmas                      % Genera la pagina de firmas.
\thispagestyle{empty}
\tableofcontents                    % Genera indice general.
\newpage
\sloppy
\newpage

\pagenumbering{arabic}

\begin{abstract}
Yet to write

\begin{comment}
Normalmente, cuando se presenta un documento de este tipo o un artículo, es
importante incluir un {\it Resumen}, que en alrededor de 200 palabras 
informa al
lector los aspectos más relevantes del trabajo. Esto es de importancia, por
ejemplo, para buscar bibliografía o seleccionar aquellos documentos que en
determinado momento son de interés para alguien.  

Las preguntas a contestar en el Resumen son las siguientes:
\begin{itemize}
	\item ?`Para qué Maestría es la Propuesta?
	\item ?`Cuál es la el contexto y situación problemática en la que se encuentra?
	\item ?`Qué problema particular se piensa resolver? ¿por qué se quiere
	  resolver  y para qué?
	\item ?`Qué se ha hecho antes?
	\item ?`Cómo se piensa resolver?
	\item ?`Qué se espera obtener?
\end{itemize}
\end{comment}
\end{abstract}


\section{Introduction}
\chapter{Introduction}
\label{ch:Introduction}

This chapter introduces the problem statement and hypothesis of the thesis.

\section{Introduction}
\label{sec:Introduction}
\chapter{Introduction}
\label{ch:Introduction}

This chapter introduces the problem statement and hypothesis of the thesis.

\section{Introduction}
\label{sec:Introduction}
\chapter{Introduction}
\label{ch:Introduction}

This chapter introduces the problem statement and hypothesis of the thesis.

\section{Introduction}
\label{sec:Introduction}
\input{introduction/introduction}

\section{Problem Statement and Motivation}
\label{sec:ProblemDefinition}
\input{introduction/problem}

\section{Objectives}
\label{sec:Objectives}
\input{introduction/objectives}

\section{Hypothesis}
\label{sec:Hypothesis}
\input{introduction/hypotheses}

\section{Methodology}
\label{sec:Methodology}
\input{introduction/methodology}

\section{Contributions}
% Describe the exact contributions of this thesis
Yet to write

\section{Outline of the thesis}
Yet to write


\section{Problem Statement and Motivation}
\label{sec:ProblemDefinition}
Breast cancer is the most commonly diagnosed cancer in woman and its death rates are among the highest of any cancer. It is estimated that about 1 in 8 U.S. women will be diagnosed with breast cancer at some point in their lifetime. Early detection is key in reducing the number of deaths from breast cancer; detection in its earlier stage (\textit{in situ}) increases the survival rate to virtually 100\%~\cite{Howlader2014}.

With current technology, a high quality mammogram is ``the most efective way to detect breast cancer early''~\cite{Mammograms2014}. Mammograms are used by radiologists to search for early signs of cancer such as tumors or microcalcifications. About 85\% of breast cancers can be detected with a screening mammogram~\cite{PerformanceMammography2013}. This high sensitivity is the product of careful examination of the mammograms by experienced radiologists. A computer-aided diagnosis tool (CAD) could automatically detect and diagnose these abnormalities saving the time and training needed by expert radiologists and avoiding any human error. Computer based approaches could also be used by radiologists as a help during the screening proccess or as a second informed opinion on a diagnosis.

CAD systems are based on image and classification techniques coming from Artificial Intelligence and Machine Learning. Traditional CAD tools for breast cancer diagnosis are composed of three steps: feature extraction, feature selection and classification. In the feature extraction phase, the system uses filters and image transformations to preprocess the mammogram and find geometric patterns which are used to produce a set of features for the image; expert knowledge is sometimes used in this phase. Feature selection or regularization is used to focus only on the important features for the classification task. Once a vector of features is obtained for each image, an standard binary classifier can be used to perform the final detection or diagnosis. These techniques have been used for many years and are standard in the industry~\footnote{See~\cite{Hernandez2014} for an example of a CAD system developed in this institution.}.

Despite its widespread use and efficiency, systems based on traditional computer vision techniques have various limitations that should be addressed to further improve its performance:
\begin{itemize}
	\item There is no standard way of preprocessing mammograms. Some techniques are commonly used but their performances can vary.
	\item It uses handcrafted features. The features extracted from the image are chosen beforehand (maybe designed with the help of experts) and special filters and image techniques are used to extract them.
	\item Segmentation and image feature extraction are error-prone and could greatly affect the classification results.
	\item It normally uses a small patch of the mammogram and makes a prediction on that patch but it does not consider the entire mammogram neither to make a prediction on the patient or to account for correlation between patches.
	\item To produce good results it requires knowledge in various fields such as radiology, oncology, image processing, computer vision, machine learning, etc.
	\item It is composed of many sequential steps. At each stage, there are many techniques from which the researcher can choose and many parameters which have to be estimated. This represents a cost in time and results as it is improbable that the optimal selection of techniques and parameters is achieved.
	\item As it is a complex system with different subsystems involved many other issues can arise such as non desired or unknown dependencies between subsystems, difficulty to localize errors, maintainability, etc.  
	\item The techniques currently used are complex but the improvements achieved are not substantial. Much work is needed to make only incremental improvements and it is hard to know to which part of the system dedicate more resources.
\end{itemize}

This project will center around using Convolutional Networks, a recent development in computer vision, (see Section~\ref{subsec:ConvNets}) to tackle some of these limitations, especifically automate preprocessing, feature extraction and segementation, use entire mammogram images and simplify the system pipeline by using a convolutional network as a replacement for many steps traditionally performed in succesion.

\begin{comment}
El {\it Problema} es el núcleo de la propuesta. En esta parte se define y se
justifica clara y ampliamente la situación que se pretende
resolver. Normalmente el problema particular a resolver cae dentro de un
contexto más amplio, dentro de una situación problemática de la cual se
derivan regularmente más de un problema. Los aspectos a considerar en esta
parte consisten en lo siguiente:
\begin{itemize}
	\item Describir la Situación problemática, es decir, identificar los
	problemas o áreas de oportunidad donde se ubica su investigación y los
	antecedentes de esa situación. 
	\item Definir el problema a detalle con sus factores, aspectos, relaciones
  y desarrollar la importancia de ese problema. Debería estar basado en
  literatura.
\end{itemize}
\end{comment}


\section{Objectives}
\label{sec:Objectives}
Yet to write

improve the results obtained with more traiditional methods
Dejar el sistema aqui y el codigo de las convolutional networks so that it could be use don some other tasks or in 3d tommography 
comenzar en deep learning en la institucion Kickstart the work on convolutiopnal netowkr or deep learning in the intitution. 
generate reslts tha culd result in an conference or journal article
Perfomr a careful evaluationn of the convnets to determine what can be improved and work on it. 
Test the different hypothesis and give a concise answer to 
Sauy if this is a mehtid worht to put the resources on, . If it is yes, point to some directions wher eit could be imoporved. if it  is not, poit to some of the problems that are preventing it from doing it. 




\begin{comment}
Especificar en esta sección qué es lo que quiere lograr con respecto al problema identificado
en forma general y particular. Puede incluir alcances y cualquier otro
elemento que considere pertinente para delimitar su trabajo. 


{\bf Por ejemplo:}

El objetivo general de este trabajo.......

Los objetivos particulares a cumplir en este trabajo de investigación son los
siguientes: 
\begin{itemize}
	\item El primer objetivo...
	\item El segundo objetivo...
\end{itemize}

Esta sección puede contener también el {\it Modelo Particular}, que es el modelo de solución propuesto para el
problema y que obviamente debe ser consistente con los objetivos
establecidos. Se le llama {\it Modelo Particular}
 porque es en el cual se guía
el trabajo de investigación y que desemboca en lo que es la
 {\bf CONTRIBUCIÓN PERSONAL}.
Aquí es donde los aspectos de creatividad e innovación deben verse aplicados a nuestro
trabajo.
\end{comment}


\section{Hypothesis}
\label{sec:Hypothesis}
Yet to write

Can we do better than what has alreayd been reported using convnets. can we do better than what has been eported using other methods
Can we simplify the task of image recognition for this task
what are the best parameters ofr teh neural network (number of hidden networks, maxout vs pool, RELus vs logistic, kernel sizes)Is there a big improvement on refining and tuning the nertwork paraameters for the task in hand
How good are the resutls on the enrtire mammogram image?. Is there a way to join the results on the small patche to make a prediction on the patient?
Is the GPu optimization neccesary. 
Will the data be eough or willht network overfit to the small amount of data.
Can the features obtained from a convolotional network trainedon  a different database(like the imageNet database) be used ot o btain results on our images. are thos results better than using a shallow convnet trained on medical images
Are convolutionla netowkrs traine don pixel images better at this task than non convolutional neural networks or other non linear classifiers (SVMs, k-means) trained on handcarfted features?
Is this a good path to keep working on to try to solve these bproblem or sjould we put resources on other methods?
can we achieve human-like performance 
\begin{comment}
Las {\it Hipótesis}, que de acuerdo a Sampieri {\it indican lo que estamos
  buscando 
o tratando de probar y pueden definirse como explicaciones tentativas del
fenómeno investigado y formuladas a manera de proposiciones}. Las hipótesis
surgen normalmente de los {\it Objetivos} y proponen contestar tentativamente
  las preguntas de investigación.

{\bf Las preguntas de investigación se incluyen aquí ......}
\end{comment}


\section{Methodology}
\label{sec:Methodology}
In order to achieve the proposed objectives and test our hypotheses we will need to carry out various tasks. We list them here in the order in which we plan to execute them:

\begin{itemize}
	\item Literature review: A thorough review of the published work using the databases and resources available in the institution. By the end of this task, a complete theoretical background should be obtained and written. This will also help refine the scope of the project and the experiments to be conducted.
	\item Software review: Once a clear idea of what are the possible experiments to be executed, we will need to find appropiate software to perform them. Software for database managing, preprocessing and implementation of different neural networks should be either located or developed.
	\item Database preprocessing: We will ready the database images for the experiments; these implies joining different databases, obtaining the required features, preprocessing the images, assigning labels, etc.
	\item Assesing image preprocessing: We will train a standard convolutional network with fixed parameters on mammograms with three different preprocessings: no preprocessing, image enhancement using median or gaussian filters and wavelet filtered images. Furthermore, we will train a deeper convolutional network on nonpreprocessed images. We want to answer three research questions: which is the best preprocessing for convolutional networks, is using the best filter significantly better than using nonpreprocessed images and can a convolutional network automatically preprocess the images?
% Q: Is it better to make different preprocessings oin the same convolutional network or to fit each convolutional network for each preprocessing, thus, giving it the best chance to perform but taking more time.
	\item Exploratory experiments: We will train standard convolutional networks in two different inputs: small image patches obtained from mammograms and whole mammogram images. We will also train a linear classifier, probably rectified linear units, on the features obtained from a convolutional network trained on the ImageNet database, i.e., we will use an already trained convolutional network instead of one trained specifically in mammograms. Here we will use the image preprocessing technique that showed better results in the previous step. We want to answer two research questions: Can a convolutional network trained on whole mammograms perform as well as one trained on small patches and can we use an already trained convolutional network to classify mammograms?
	\item Model selection: Using the insights from previous sections and the current literature on convolutional networks, we will select a network architecture along with novel features, preprocessing, training and regularization procedures. We aspire to find the best convolutional network configuration for mammogram classification.
	\item Further experiments: We will train the chosen convolutional network on our mammographic database. We will perform crossvalidation to adjust the most important network parameters and use regularization to avoid possible overfitting. We want to answer two research questions: is the performance of the convolutional network considerably improved by parameter tuning and, more importantly, is this a good performance?.
%Maybe train one with no tune fitting.
	\item Gathering results: Produce results on the test set and elaborate figures and tables. This could be obtained directly from software output or from further program executions.
	\item Reporting results: Write the thesis and any article or technical guide which may result from this work. Both this and the previous step will be performed along the execution of the project, hopefully benefiting from the supervisors' feedback.
\end{itemize}
Finally, we want to note that this is an idealized workflow and some changes may occur due to time limitations or resources unavailability. In the unlikely case that the work is finished before the project deadline, we will either reiterate on model selection, experiments, result gathering and reporting or look into digital tomosynthesis, network ensembles or evolving convolutional networks.

\begin{comment}
La {\it Metodología} (o lo que algunos autores llaman el {\it Método})
 es el proceso o
conjunto de pasos que debe efectuarse para llegar a cumplir con los
objetivos. Esos pasos deben contener  los experimentos a realizar, la forma de
llevarlos a cabo, la evaluación de los resultados, la prueba de las hipótesis,
la respuesta a las preguntas de investigación y el último paso debe ser el
reporte escrito de los resultados.
\end{comment}


\section{Contributions}
% Describe the exact contributions of this thesis
Yet to write

\section{Outline of the thesis}
Yet to write


\section{Problem Statement and Motivation}
\label{sec:ProblemDefinition}
Breast cancer is the most commonly diagnosed cancer in woman and its death rates are among the highest of any cancer. It is estimated that about 1 in 8 U.S. women will be diagnosed with breast cancer at some point in their lifetime. Early detection is key in reducing the number of deaths from breast cancer; detection in its earlier stage (\textit{in situ}) increases the survival rate to virtually 100\%~\cite{Howlader2014}.

With current technology, a high quality mammogram is ``the most efective way to detect breast cancer early''~\cite{Mammograms2014}. Mammograms are used by radiologists to search for early signs of cancer such as tumors or microcalcifications. About 85\% of breast cancers can be detected with a screening mammogram~\cite{PerformanceMammography2013}. This high sensitivity is the product of careful examination of the mammograms by experienced radiologists. A computer-aided diagnosis tool (CAD) could automatically detect and diagnose these abnormalities saving the time and training needed by expert radiologists and avoiding any human error. Computer based approaches could also be used by radiologists as a help during the screening proccess or as a second informed opinion on a diagnosis.

CAD systems are based on image and classification techniques coming from Artificial Intelligence and Machine Learning. Traditional CAD tools for breast cancer diagnosis are composed of three steps: feature extraction, feature selection and classification. In the feature extraction phase, the system uses filters and image transformations to preprocess the mammogram and find geometric patterns which are used to produce a set of features for the image; expert knowledge is sometimes used in this phase. Feature selection or regularization is used to focus only on the important features for the classification task. Once a vector of features is obtained for each image, an standard binary classifier can be used to perform the final detection or diagnosis. These techniques have been used for many years and are standard in the industry~\footnote{See~\cite{Hernandez2014} for an example of a CAD system developed in this institution.}.

Despite its widespread use and efficiency, systems based on traditional computer vision techniques have various limitations that should be addressed to further improve its performance:
\begin{itemize}
	\item There is no standard way of preprocessing mammograms. Some techniques are commonly used but their performances can vary.
	\item It uses handcrafted features. The features extracted from the image are chosen beforehand (maybe designed with the help of experts) and special filters and image techniques are used to extract them.
	\item Segmentation and image feature extraction are error-prone and could greatly affect the classification results.
	\item It normally uses a small patch of the mammogram and makes a prediction on that patch but it does not consider the entire mammogram neither to make a prediction on the patient or to account for correlation between patches.
	\item To produce good results it requires knowledge in various fields such as radiology, oncology, image processing, computer vision, machine learning, etc.
	\item It is composed of many sequential steps. At each stage, there are many techniques from which the researcher can choose and many parameters which have to be estimated. This represents a cost in time and results as it is improbable that the optimal selection of techniques and parameters is achieved.
	\item As it is a complex system with different subsystems involved many other issues can arise such as non desired or unknown dependencies between subsystems, difficulty to localize errors, maintainability, etc.  
	\item The techniques currently used are complex but the improvements achieved are not substantial. Much work is needed to make only incremental improvements and it is hard to know to which part of the system dedicate more resources.
\end{itemize}

This project will center around using Convolutional Networks, a recent development in computer vision, (see Section~\ref{subsec:ConvNets}) to tackle some of these limitations, especifically automate preprocessing, feature extraction and segementation, use entire mammogram images and simplify the system pipeline by using a convolutional network as a replacement for many steps traditionally performed in succesion.

\begin{comment}
El {\it Problema} es el núcleo de la propuesta. En esta parte se define y se
justifica clara y ampliamente la situación que se pretende
resolver. Normalmente el problema particular a resolver cae dentro de un
contexto más amplio, dentro de una situación problemática de la cual se
derivan regularmente más de un problema. Los aspectos a considerar en esta
parte consisten en lo siguiente:
\begin{itemize}
	\item Describir la Situación problemática, es decir, identificar los
	problemas o áreas de oportunidad donde se ubica su investigación y los
	antecedentes de esa situación. 
	\item Definir el problema a detalle con sus factores, aspectos, relaciones
  y desarrollar la importancia de ese problema. Debería estar basado en
  literatura.
\end{itemize}
\end{comment}


\section{Objectives}
\label{sec:Objectives}
Yet to write

improve the results obtained with more traiditional methods
Dejar el sistema aqui y el codigo de las convolutional networks so that it could be use don some other tasks or in 3d tommography 
comenzar en deep learning en la institucion Kickstart the work on convolutiopnal netowkr or deep learning in the intitution. 
generate reslts tha culd result in an conference or journal article
Perfomr a careful evaluationn of the convnets to determine what can be improved and work on it. 
Test the different hypothesis and give a concise answer to 
Sauy if this is a mehtid worht to put the resources on, . If it is yes, point to some directions wher eit could be imoporved. if it  is not, poit to some of the problems that are preventing it from doing it. 




\begin{comment}
Especificar en esta sección qué es lo que quiere lograr con respecto al problema identificado
en forma general y particular. Puede incluir alcances y cualquier otro
elemento que considere pertinente para delimitar su trabajo. 


{\bf Por ejemplo:}

El objetivo general de este trabajo.......

Los objetivos particulares a cumplir en este trabajo de investigación son los
siguientes: 
\begin{itemize}
	\item El primer objetivo...
	\item El segundo objetivo...
\end{itemize}

Esta sección puede contener también el {\it Modelo Particular}, que es el modelo de solución propuesto para el
problema y que obviamente debe ser consistente con los objetivos
establecidos. Se le llama {\it Modelo Particular}
 porque es en el cual se guía
el trabajo de investigación y que desemboca en lo que es la
 {\bf CONTRIBUCIÓN PERSONAL}.
Aquí es donde los aspectos de creatividad e innovación deben verse aplicados a nuestro
trabajo.
\end{comment}


\section{Hypothesis}
\label{sec:Hypothesis}
Yet to write

Can we do better than what has alreayd been reported using convnets. can we do better than what has been eported using other methods
Can we simplify the task of image recognition for this task
what are the best parameters ofr teh neural network (number of hidden networks, maxout vs pool, RELus vs logistic, kernel sizes)Is there a big improvement on refining and tuning the nertwork paraameters for the task in hand
How good are the resutls on the enrtire mammogram image?. Is there a way to join the results on the small patche to make a prediction on the patient?
Is the GPu optimization neccesary. 
Will the data be eough or willht network overfit to the small amount of data.
Can the features obtained from a convolotional network trainedon  a different database(like the imageNet database) be used ot o btain results on our images. are thos results better than using a shallow convnet trained on medical images
Are convolutionla netowkrs traine don pixel images better at this task than non convolutional neural networks or other non linear classifiers (SVMs, k-means) trained on handcarfted features?
Is this a good path to keep working on to try to solve these bproblem or sjould we put resources on other methods?
can we achieve human-like performance 
\begin{comment}
Las {\it Hipótesis}, que de acuerdo a Sampieri {\it indican lo que estamos
  buscando 
o tratando de probar y pueden definirse como explicaciones tentativas del
fenómeno investigado y formuladas a manera de proposiciones}. Las hipótesis
surgen normalmente de los {\it Objetivos} y proponen contestar tentativamente
  las preguntas de investigación.

{\bf Las preguntas de investigación se incluyen aquí ......}
\end{comment}


\section{Methodology}
\label{sec:Methodology}
In order to achieve the proposed objectives and test our hypotheses we will need to carry out various tasks. We list them here in the order in which we plan to execute them:

\begin{itemize}
	\item Literature review: A thorough review of the published work using the databases and resources available in the institution. By the end of this task, a complete theoretical background should be obtained and written. This will also help refine the scope of the project and the experiments to be conducted.
	\item Software review: Once a clear idea of what are the possible experiments to be executed, we will need to find appropiate software to perform them. Software for database managing, preprocessing and implementation of different neural networks should be either located or developed.
	\item Database preprocessing: We will ready the database images for the experiments; these implies joining different databases, obtaining the required features, preprocessing the images, assigning labels, etc.
	\item Assesing image preprocessing: We will train a standard convolutional network with fixed parameters on mammograms with three different preprocessings: no preprocessing, image enhancement using median or gaussian filters and wavelet filtered images. Furthermore, we will train a deeper convolutional network on nonpreprocessed images. We want to answer three research questions: which is the best preprocessing for convolutional networks, is using the best filter significantly better than using nonpreprocessed images and can a convolutional network automatically preprocess the images?
% Q: Is it better to make different preprocessings oin the same convolutional network or to fit each convolutional network for each preprocessing, thus, giving it the best chance to perform but taking more time.
	\item Exploratory experiments: We will train standard convolutional networks in two different inputs: small image patches obtained from mammograms and whole mammogram images. We will also train a linear classifier, probably rectified linear units, on the features obtained from a convolutional network trained on the ImageNet database, i.e., we will use an already trained convolutional network instead of one trained specifically in mammograms. Here we will use the image preprocessing technique that showed better results in the previous step. We want to answer two research questions: Can a convolutional network trained on whole mammograms perform as well as one trained on small patches and can we use an already trained convolutional network to classify mammograms?
	\item Model selection: Using the insights from previous sections and the current literature on convolutional networks, we will select a network architecture along with novel features, preprocessing, training and regularization procedures. We aspire to find the best convolutional network configuration for mammogram classification.
	\item Further experiments: We will train the chosen convolutional network on our mammographic database. We will perform crossvalidation to adjust the most important network parameters and use regularization to avoid possible overfitting. We want to answer two research questions: is the performance of the convolutional network considerably improved by parameter tuning and, more importantly, is this a good performance?.
%Maybe train one with no tune fitting.
	\item Gathering results: Produce results on the test set and elaborate figures and tables. This could be obtained directly from software output or from further program executions.
	\item Reporting results: Write the thesis and any article or technical guide which may result from this work. Both this and the previous step will be performed along the execution of the project, hopefully benefiting from the supervisors' feedback.
\end{itemize}
Finally, we want to note that this is an idealized workflow and some changes may occur due to time limitations or resources unavailability. In the unlikely case that the work is finished before the project deadline, we will either reiterate on model selection, experiments, result gathering and reporting or look into digital tomosynthesis, network ensembles or evolving convolutional networks.

\begin{comment}
La {\it Metodología} (o lo que algunos autores llaman el {\it Método})
 es el proceso o
conjunto de pasos que debe efectuarse para llegar a cumplir con los
objetivos. Esos pasos deben contener  los experimentos a realizar, la forma de
llevarlos a cabo, la evaluación de los resultados, la prueba de las hipótesis,
la respuesta a las preguntas de investigación y el último paso debe ser el
reporte escrito de los resultados.
\end{comment}


\section{Contributions}
% Describe the exact contributions of this thesis
Yet to write

\section{Outline of the thesis}
Yet to write


\section{Problem Definition}
Breast cancer is the most commonly diagnosed cancer in woman and its death rates are among the highest of any cancer. It is estimated that about 1 in 8 U.S. women will be diagnosed with breast cancer at some point in their lifetime.~\cite{CSR2014} Early detection is key in reducing the effects of breast cancer: a detection in its earlier stage (in situ) allows for better treatment and increases the survival rate to virtually 100\%.~\cite{CSR2014}

With current technology, a high quality mammogram is ''the most efective way to detect breast cancer early`` \cite{MammogramFactSheet2014}. Mammograms are x-ray images of each breast used by radiologists to search for early signs of cancer such as tumors or microcalcifications. 



90\% of breats cancers can be detected using  ammomgram and radiologist can detect up to .Nonetheless, these tasks is not easy and requires training an dexperience to be realized effectuvely  a computer aided diagnostic tool that can automatically detect these signs from digitizes images of the mammogrmas couls save the time to expert and increase the aomoun t of cancers detect or pint to experst whthe zones that could be treated woith more rcareExepret radiologists can detect up to ... but normally you do double duty... 
Radiologists normally get this amunt of time right. and a second decision would increase the amount of positives(?) detected and save time for the experts.

 
In this work we will focus on using mammograms
The project developed here has already dealed with this problem using .... and .... cite. but we tried to use convnets so that we can improve the results obtained here and in the literature with some advanced image pattern algorithms. A further review was given on t eintroduction.

What is the problem/limitations? Not yet efficient, requireds handcrafted fuigures and a lot of parameter fitting an estimation to get the better results. It depends on a lot of steps and thus could be prone to errors and requires work in various differents fields (image processing, radiology, macjhine learning) to prouce good results. Furthermore, some of tit depends on expert information (like the shapes of microcalcifications and masses and important image features) to produce results while a better systmem could let the computer figure out what are those neccesary features  

\begin{comment}
El {\it Problema} es el núcleo de la propuesta. En esta parte se define y se
justifica clara y ampliamente la situación que se pretende
resolver. Normalmente el problema particular a resolver cae dentro de un
contexto más amplio, dentro de una situación problemática de la cual se
derivan regularmente más de un problema. Los aspectos a considerar en esta
parte consisten en lo siguiente:
\begin{itemize}
	\item Describir la Situación problemática, es decir, identificar los
	problemas o áreas de oportunidad donde se ubica su investigación y los
	antecedentes de esa situación. 
	\item Definir el problema a detalle con sus factores, aspectos, relaciones
  y desarrollar la importancia de ese problema. Debería estar basado en
  literatura.
\end{itemize}
\end{comment}


\section{Objectives}
Yet to write

improve the results obtained with more traiditional methods
Dejar el sistema aqui y el codigo de las convolutional networks so that it could be use don some other tasks or in 3d tommography 
comenzar en deep learning en la institucion Kickstart the work on convolutiopnal netowkr or deep learning in the intitution. 
generate reslts tha culd result in an conference or journal article
Perfomr a careful evaluationn of the convnets to determine what can be improved and work on it. 
Test the different hypothesis and give a concise answer to 
Sauy if this is a mehtid worht to put the resources on, . If it is yes, point to some directions wher eit could be imoporved. if it  is not, poit to some of the problems that are preventing it from doing it. 




\begin{comment}
Especificar en esta sección qué es lo que quiere lograr con respecto al problema identificado
en forma general y particular. Puede incluir alcances y cualquier otro
elemento que considere pertinente para delimitar su trabajo. 


{\bf Por ejemplo:}

El objetivo general de este trabajo.......

Los objetivos particulares a cumplir en este trabajo de investigación son los
siguientes: 
\begin{itemize}
	\item El primer objetivo...
	\item El segundo objetivo...
\end{itemize}

Esta sección puede contener también el {\it Modelo Particular}, que es el modelo de solución propuesto para el
problema y que obviamente debe ser consistente con los objetivos
establecidos. Se le llama {\it Modelo Particular}
 porque es en el cual se guía
el trabajo de investigación y que desemboca en lo que es la
 {\bf CONTRIBUCIÓN PERSONAL}.
Aquí es donde los aspectos de creatividad e innovación deben verse aplicados a nuestro
trabajo.
\end{comment}


\section{Hypotheses}
Yet to write

Can we do better than what has alreayd been reported using convnets. can we do better than what has been eported using other methods
Can we simplify the task of image recognition for this task
what are the best parameters ofr teh neural network (number of hidden networks, maxout vs pool, RELus vs logistic, kernel sizes)Is there a big improvement on refining and tuning the nertwork paraameters for the task in hand
How good are the resutls on the enrtire mammogram image?. Is there a way to join the results on the small patche to make a prediction on the patient?
Is the GPu optimization neccesary. 
Will the data be eough or willht network overfit to the small amount of data.
Can the features obtained from a convolotional network trainedon  a different database(like the imageNet database) be used ot o btain results on our images. are thos results better than using a shallow convnet trained on medical images
Are convolutionla netowkrs traine don pixel images better at this task than non convolutional neural networks or other non linear classifiers (SVMs, k-means) trained on handcarfted features?
Is this a good path to keep working on to try to solve these bproblem or sjould we put resources on other methods?
can we achieve human-like performance 
\begin{comment}
Las {\it Hipótesis}, que de acuerdo a Sampieri {\it indican lo que estamos
  buscando 
o tratando de probar y pueden definirse como explicaciones tentativas del
fenómeno investigado y formuladas a manera de proposiciones}. Las hipótesis
surgen normalmente de los {\it Objetivos} y proponen contestar tentativamente
  las preguntas de investigación.

{\bf Las preguntas de investigación se incluyen aquí ......}
\end{comment}


\section{Background}
We offer an introduction to some of the essential concepts needed to understand the rest of this document. %concepts used in this document.

	\subsection{Breast Cancer}
	\emph{Cancer} is an umbrella term for a group of diseases caused by abnormal cell growth in different parts of the body. The accumulation of extra cells usually forms a mass of tissue called a \emph{tumor}. Tumors can be benign or malignant: \emph{benign tumors} are noncancerous, lack the ability to invade surrounding tissue and will not regrow if removed from the body;  malignant or \emph{cancerous tumors} are harmful, can invade nearby organs and tissues (\emph{invasive cancer}), can spread to other parts of the body (\emph{metastasis}) and will sometimes regrow when removed~\cite{WYNTKABreastCancer2012}.

\emph{Breast cancer} forms in tissues of the breast. The two most common types of breast cancer are \emph{ductal carcinoma} and \emph{lobular carcinoma}, which start in the breast ducts and lobules, respectively (see Fig.~\ref{fig:BreastAnatomy}). Breast cancer \emph{incidence rate}, the number of new cases in a specified population during a year, is the highest of any cancer among American women. Its \emph{mortality rate}, the number of deaths during a year, is also one of the highest of any cancer~\cite{Howlader2014}.

\begin{figure}[h]
	\centering
	\includegraphics[width = 0.35\textwidth]{plots/breastAnatomy.png}
	\caption[Female Breast Anatomy]{Anatomy of the female breast. Image courtesy of NCI.}
	\label{fig:BreastAnatomy}
\end{figure}

The \emph{cancer stage} depends on the size of the tumor and whether the cancer cells have spread to neighboring tissue or other parts of the body. It is expressed as a Roman numeral ranging from 0 through IV; stage I cancer is considered \emph{early-stage breast cancer} and stage IV cancer is considered \emph{advanced}. Stage 0 describes non-invasive breast cancers, also known as \emph{carcinoma in situ}. Stage I, II and III describe invasive breast cancer, i.e., cancer has invaded normal surrounding breast tissue. Stage IV is used to describe metastatic cancer, i.e., it has spread beyond nearby tissue to other organs of the body.

\subsubsection{Mammograms}
A \emph{mammogram} is an x-ray image of the breast. Radiologists use \emph{screening mammograms} (normally composed of two mammograms of each breast) to check for breast cancer signs on women who lack symptoms of the disease. If an abnormality is found, a \emph{diagnostic mammogram} is ordered, these are detailed x-ray pictures of the suspicious region~\cite{Mammograms2014}. A standard mammogram is shown in Fig.~\ref{fig:normalMammogram}.

\begin{figure}[h]
	\centering
	\includegraphics[width = 0.25\textwidth]{plots/normalMammogram.jpg}
	\caption[Digital Mammogram]{A standard mammogram.}
	\label{fig:normalMammogram}
\end{figure}

Having a screening mammogram in a regular basis is the most effective method for detecting breast cancer early; around 85\% of breast cancers can be detected in a screening mammogram~\cite{PerformanceMammography2013}. Nevertheless, screening mammograms have many limitations: a high false positive rate, overtreatment in Stage 0 cancer, false negative results for women with high breast density, radiation exposure and physical and psychological discomfort~\cite{Mammograms2014}.

Radiologists look primarily for microcalcifications and breast masses. \emph{Microcalcifications} are tiny deposits of calcium in the breast tissue that can be a sign of early breast cancer if found in clusters with irregular layout and shapes. \emph{Breast masses} or breast lumps are a variety of things: fluid-filled cysts, fatty tissues, fibric tissues, noncancerous or cancerous tumors, among others. A mass can be a sign of breast cancer if it has an irregular shape and poorly defined margins. See Fig.~\ref{fig:breastCancerSigns} for an example of possible signs of breast cancer. Radiologists will also consider the breast density of the patient when reading a mammogram given that high breast density is linked to a higher risk of breast cancer~\cite{MammogramsACS2014}.

\begin{figure}[h]
	\centering
	\begin{subfigure}{0.25\textwidth}
                \includegraphics[width=\textwidth]{plots/breastMicrocalcification.jpg}
        \end{subfigure}
	~
	\begin{subfigure}{0.25\textwidth}
                \includegraphics[width=\textwidth]{plots/breastMass.jpg}
        \end{subfigure}
	\caption[Breast Cancer Signs]{Signs of possible breast cancer in a mammogram. Left: A cluster of microcalcifications in an irregular layout. Right: A poorly defined breast mass.}
	\label{fig:breastCancerSigns}
\end{figure}

Conventional mammography uses film to record x-ray images of the breast. \emph{Digital mammography}, on the other hand, uses digital receptors to convert the x-rays into electrical signals and stores the image electronically. Digital mammograms offer a clearer picture of the breast and can be digitally manipulated and shared between health care providers. However, researchers still debate its effectiveness to identify breast cancer over film mammograms~\cite{Kerlikowske2011, Pisano2008, Skaane2007}. Digital mammography is steadily becoming the standard for breast cancer screening. Fig.~\ref{fig:normalMammogram} is, in fact, a digital mammogram.

\emph{Digital tomosynthesis}, also called three-dimensional mammography, is a new technology that produces 3-dimensional x-ray images of the breast and is expected to improve the efficacy of regular 2-d mammograms. Studies comparing the two techniques have not yet been published~\cite{Mammograms2014}.

We center on using mammograms, either digital or digitized from film, to detect microcalcifications and masses and predict the likelihood of breast cancer on the patient.

We wrote most of this section using information from the National Cancer Institute. We recommend to visit its website (\url{www.cancer.gov}) for further details.


	\subsection{Classification}

	\subsection{Artificial Neural Networks}

	\subsection{Convolutional Networks}
	\label{subsec:convnets}

	\subsection{Convolutional Networks applied to Breast Cancer}

	\subsection{Mammography DB}

	\subsection{Related Work}
	\label{subsec:RelatedWork}
	In this section we offer a summary of some of the first work on using convolutional networks for breast cancer diagnosis as well as some of the articles that have influenced this thesis.

Lo et al.\cite{Lo1995} were the first group to use convolutional networks for breast cancer detection. They used a CNN with two hidden layers to detect microcalcifications. A high sensitivity image processing technique was used to obtain a set of 2104 patches (16 by 16 pixels) of all potential disease areas from 68 digital mammograms; of these, 265 were true microcalcifications and 1821 were ``false subtle microcalcifications". Prior to training the CNN, a wavelet high-pass filtering technique was used to remove the background of these images. Each image was flipped over (left-right) and 4 rotations for each the original and flipped images were used for training (0°, 90°, 180° and 270°). The CNN was composed of one input unit ($16\times16$), 12 units in the first hidden layer ($12\times12$), 12 units in the second hidden layer($8\times 8$) and two output nodes(one for YES and one for NOT). The input size ($16\time16$), number of hidden layers ($2$) and kernel size ($5\times5$) was obtained via cross validation, altough not many other options were explored: they tried input sizes of 8, 16 or 32, one or two hidden layers and kernel sizes of 2, 3, 5 or 13. The CNN reached 0.87 average AUC when identifying individual microcalcifications and 0.97 AUC for clustered microcalcifications. Only a minimum of three calcifications was considered a detection. Sensitivity and specificity test results were not reported. This article proved that simple convolutional networks can be efficiently used for medical image pattern recognition.
%Lesson learned: two hidden layer newtwork produces better results, background reduction is neccesary and using matrices invariance to augment the data helps. Convnets work helps and convnets work.

%A refined approach was presented some years later by the same authors along some non convolutional neural networks~\cite{Lo1998}. The setting is very similar but ..... Results are..

%work done at Tec


\section{Methodology}
In order to achieve the proposed objectives and test our hypotheses we will need to carry out various tasks. We list them here in the order in which we plan to execute them:

\begin{itemize}
	\item Literature review: A thorough review of the published work using the databases and resources available in the institution. By the end of this task, a complete theoretical background should be obtained and written. This will also help refine the scope of the project and the experiments to be conducted.
	\item Software review: Once a clear idea of what are the possible experiments to be executed, we will need to find appropiate software to perform them. Software for database managing, preprocessing and implementation of different neural networks should be either located or developed.
	\item Database preprocessing: We will ready the database images for the experiments; these implies joining different databases, obtaining the required features, preprocessing the images, assigning labels, etc.
	\item Assesing image preprocessing: We will train a standard convolutional network with fixed parameters on mammograms with three different preprocessings: no preprocessing, image enhancement using median or gaussian filters and wavelet filtered images. Furthermore, we will train a deeper convolutional network on nonpreprocessed images. We want to answer three research questions: which is the best preprocessing for convolutional networks, is using the best filter significantly better than using nonpreprocessed images and can a convolutional network automatically preprocess the images?
% Q: Is it better to make different preprocessings oin the same convolutional network or to fit each convolutional network for each preprocessing, thus, giving it the best chance to perform but taking more time.
	\item Exploratory experiments: We will train standard convolutional networks in two different inputs: small image patches obtained from mammograms and whole mammogram images. We will also train a linear classifier, probably rectified linear units, on the features obtained from a convolutional network trained on the ImageNet database, i.e., we will use an already trained convolutional network instead of one trained specifically in mammograms. Here we will use the image preprocessing technique that showed better results in the previous step. We want to answer two research questions: Can a convolutional network trained on whole mammograms perform as well as one trained on small patches and can we use an already trained convolutional network to classify mammograms?
	\item Model selection: Using the insights from previous sections and the current literature on convolutional networks, we will select a network architecture along with novel features, preprocessing, training and regularization procedures. We aspire to find the best convolutional network configuration for mammogram classification.
	\item Further experiments: We will train the chosen convolutional network on our mammographic database. We will perform crossvalidation to adjust the most important network parameters and use regularization to avoid possible overfitting. We want to answer two research questions: is the performance of the convolutional network considerably improved by parameter tuning and, more importantly, is this a good performance?.
%Maybe train one with no tune fitting.
	\item Gathering results: Produce results on the test set and elaborate figures and tables. This could be obtained directly from software output or from further program executions.
	\item Reporting results: Write the thesis and any article or technical guide which may result from this work. Both this and the previous step will be performed along the execution of the project, hopefully benefiting from the supervisors' feedback.
\end{itemize}
Finally, we want to note that this is an idealized workflow and some changes may occur due to time limitations or resources unavailability. In the unlikely case that the work is finished before the project deadline, we will either reiterate on model selection, experiments, result gathering and reporting or look into digital tomosynthesis, network ensembles or evolving convolutional networks.

\begin{comment}
La {\it Metodología} (o lo que algunos autores llaman el {\it Método})
 es el proceso o
conjunto de pasos que debe efectuarse para llegar a cumplir con los
objetivos. Esos pasos deben contener  los experimentos a realizar, la forma de
llevarlos a cabo, la evaluación de los resultados, la prueba de las hipótesis,
la respuesta a las preguntas de investigación y el último paso debe ser el
reporte escrito de los resultados.
\end{comment}


\section{Work Plan}
We present here the expected work plan for this master's thesis. A description of the activities can be found in Section~\ref{sec:Methodology}
\begin{figure}[h]
	\centering
	\includegraphics[width = \textwidth]{plots/workplan.png}
	\caption[Thesis Work Plan]{Thesis work plan.}
	\label{fig:workplan}
\end{figure}

\begin{comment}
Una vez establecida la {\it Metodología} es importante establecer las
actividades con sus tiempos respectivos en lo que se llama el {\it Plan de
Trabajo}. Ello nos da una idea clara de la extensión en tiempo del trabajo
propuesto. Además de ser necesario, lo cual es normalmente cierto en
propuestas de proyecto industrial, es importante establecer el PLAN FINANCIERO
el cual desglosa los recursos necesarios para el desarrollo del proyecto y sus
costos.

{\bf Un ejemplo de Plan de Trabajo}

La figura \ref{ttphd} presenta el cronograma de las actividades que llevarán a
cabo los objetivos de esta tesis. 

\begin{figure}[th]
	\centerline{\includegraphics[width=4in,height=3in]{plots/ttphd.pdf}}
	\caption{Cronograma de Actividades para desarrollar la Tesis}
	\label{ttphd}
\end{figure}
\end{comment}


% Bibliography
\bibliographystyle{plain}
\bibliography{bibliography}

\end{document} 

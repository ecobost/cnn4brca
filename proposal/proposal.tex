%%%%%%%%%%%%%%%%%%%%%%%%%%%%%%%%%%%%%%%%%%%%%%%%%%%%%%%%
% Written by: Erick Cobos Tandazo (a01184587@itesm.mx)
% Date: 7-April-2014
%
% Project proposal for my Master's thesis
%%%%%%%%%%%%%%%%%%%%%%%%%%%%%%%%%%%%%%%%%%%%%%%%%%%%%%%%%

\documentclass[11pt]{article}

% Packages
\usepackage[utf8]{inputenc}	% Spanish accents
\usepackage{proposalEMCT} 	% Title pages and overall format
\usepackage{verbatim} 		% Block Comm\usepackage{textcomp}ents
\usepackage{textcomp}		% For the degree sign(°)
\usepackage{hyperref}		% To manage \url and citations as links to the reference part.
\usepackage{subcaption}		% For subfigure in breastcancer.tex
\usepackage{amssymb}		% For R (real numbers) symbol, among others.
\usepackage{amsmath}		% Add text to math and further math equations.
\usepackage{lscape}		% For vertical tables

% Set properties of the document
\propautor{Erick Michael Cobos Tandazo}	% Author name
\propautormat{1184587}			% Author ID number
\propmes{Jun 25}			% Date (month)
\propanio{2015}				% Year
\propcd{Monterrey, N.L.}		% Place
\proptitulo{Early Detection and Diagnosis of Breast Cancer Lesions in Digital Mammograms using Deep Convolutional Networks}	% Title
\propasesor{Dr.~Hugo Terashima Mar\'{i}n}	% Advisor
\propsinodalA{Por definir}		% First supervisor
\propsinodalB{Por definir}		% Second supervisor.
\propdirectorPG{Dr.~Ram\'{o}n Brena Pinero}	% Program Director
\propPG{Engineering and Sciences}		% School/Department
\propcampus{Monterrey}			% University Campus
\propgrado{Master of Science}		% Academic degree
\propgradosiglas{M.Sc.}			% Academic degree abbreviation
\propespecialidad{Intelligent Systems}	% Subject
%%%%%%%%%%%%%%%%%%%%%%%%%%%%%%%%%%%%%%%%%%%%%%%%%%%%%%%%%%%%%%%%%%%%%%%%%%%%%



\begin{document}

\propportada                        % Genera la portada.
\proppagfirmas                      % Genera la pagina de firmas.
\thispagestyle{empty}
\tableofcontents                    % Genera indice general.
\newpage
\sloppy
\newpage
\pagenumbering{arabic}

\begin{abstract}
\begin{comment}
Breast cancer is one of the most common and deadliest cancer for woman around the world. The best tools used today for early breast cancer diagnosis are screening mammograms; mammograms are x-ray pictures of the breast used by radiologists to identify microcalcifications and breast masses, signs of early breast cancer development. Traditional computer systems use complex image techniques and handcrafted features to detect these lesions in mammographic images. In this work, we plan to use convolutional networks, a recent development in machine learning, which can automatically learn the relevant features for the classification task given enough training data. Convolutional networks have been used in a few studies for breast cancer detection but we hope to introduce newer features and carefully tune the architecture to produce improved results. Additionally, this will be the first approximation to use deep learning techniques as part of an ongoing project in the institution which aims to develop a computer-aided diagnosis system for breast cancer. This thesis proposal is presented for approval to obtain the degree of Master of Science in Intelligent Systems.
\end{comment}
Breast cancer is one of the most common and deadliest cancer in women but early diagnosis can greatly increase its survival rate.
Computer-aided detection (CADe) and diagnosis (CADx) assist doctors in the search of microcalcifications and masses, signs of early breast cancer. 
We plan to apply convolutional networks to mammograms, x-ray pictures of the breast, to automatically detect these lessions. 
Convolutional networks are a recent development in machine learning that learn the relevant features for classification from data; in contrast, traditional techniques use complex handcrafted features.
A few studies have used convolutional networks for breast cancer detection but we plan to introduce newer features and carefully tune the architecture to produce improved results.
Additionally, this will be the first approximation to use deep learning techniques as part of an ongoing project in the institution that aims to develop a CAD system for breast cancer.
This thesis proposal is presented for approval to obtain the degree of Master of Science in Intelligent Systems.
%Keywords: convolutional neural  networks, breast cancer, CAD, mammograms
\end{abstract}

\section{Introduction}
\chapter{Introduction}
\label{ch:Introduction}

This chapter introduces the problem statement and hypothesis of the thesis.

\section{Introduction}
\label{sec:Introduction}
\chapter{Introduction}
\label{ch:Introduction}

This chapter introduces the problem statement and hypothesis of the thesis.

\section{Introduction}
\label{sec:Introduction}
\chapter{Introduction}
\label{ch:Introduction}

This chapter introduces the problem statement and hypothesis of the thesis.

\section{Introduction}
\label{sec:Introduction}
\input{introduction/introduction}

\section{Problem Statement and Motivation}
\label{sec:ProblemDefinition}
\input{introduction/problem}

\section{Objectives}
\label{sec:Objectives}
\input{introduction/objectives}

\section{Hypothesis}
\label{sec:Hypothesis}
\input{introduction/hypotheses}

\section{Methodology}
\label{sec:Methodology}
\input{introduction/methodology}

\section{Contributions}
% Describe the exact contributions of this thesis
Yet to write

\section{Outline of the thesis}
Yet to write


\section{Problem Statement and Motivation}
\label{sec:ProblemDefinition}
Breast cancer is the most commonly diagnosed cancer in woman and its death rates are among the highest of any cancer. It is estimated that about 1 in 8 U.S. women will be diagnosed with breast cancer at some point in their lifetime. Early detection is key in reducing the number of deaths from breast cancer; detection in its earlier stage (\textit{in situ}) increases the survival rate to virtually 100\%~\cite{Howlader2014}.

With current technology, a high quality mammogram is ``the most efective way to detect breast cancer early''~\cite{Mammograms2014}. Mammograms are used by radiologists to search for early signs of cancer such as tumors or microcalcifications. About 85\% of breast cancers can be detected with a screening mammogram~\cite{PerformanceMammography2013}. This high sensitivity is the product of careful examination of the mammograms by experienced radiologists. A computer-aided diagnosis tool (CAD) could automatically detect and diagnose these abnormalities saving the time and training needed by expert radiologists and avoiding any human error. Computer based approaches could also be used by radiologists as a help during the screening proccess or as a second informed opinion on a diagnosis.

CAD systems are based on image and classification techniques coming from Artificial Intelligence and Machine Learning. Traditional CAD tools for breast cancer diagnosis are composed of three steps: feature extraction, feature selection and classification. In the feature extraction phase, the system uses filters and image transformations to preprocess the mammogram and find geometric patterns which are used to produce a set of features for the image; expert knowledge is sometimes used in this phase. Feature selection or regularization is used to focus only on the important features for the classification task. Once a vector of features is obtained for each image, an standard binary classifier can be used to perform the final detection or diagnosis. These techniques have been used for many years and are standard in the industry~\footnote{See~\cite{Hernandez2014} for an example of a CAD system developed in this institution.}.

Despite its widespread use and efficiency, systems based on traditional computer vision techniques have various limitations that should be addressed to further improve its performance:
\begin{itemize}
	\item There is no standard way of preprocessing mammograms. Some techniques are commonly used but their performances can vary.
	\item It uses handcrafted features. The features extracted from the image are chosen beforehand (maybe designed with the help of experts) and special filters and image techniques are used to extract them.
	\item Segmentation and image feature extraction are error-prone and could greatly affect the classification results.
	\item It normally uses a small patch of the mammogram and makes a prediction on that patch but it does not consider the entire mammogram neither to make a prediction on the patient or to account for correlation between patches.
	\item To produce good results it requires knowledge in various fields such as radiology, oncology, image processing, computer vision, machine learning, etc.
	\item It is composed of many sequential steps. At each stage, there are many techniques from which the researcher can choose and many parameters which have to be estimated. This represents a cost in time and results as it is improbable that the optimal selection of techniques and parameters is achieved.
	\item As it is a complex system with different subsystems involved many other issues can arise such as non desired or unknown dependencies between subsystems, difficulty to localize errors, maintainability, etc.  
	\item The techniques currently used are complex but the improvements achieved are not substantial. Much work is needed to make only incremental improvements and it is hard to know to which part of the system dedicate more resources.
\end{itemize}

This project will center around using Convolutional Networks, a recent development in computer vision, (see Section~\ref{subsec:ConvNets}) to tackle some of these limitations, especifically automate preprocessing, feature extraction and segementation, use entire mammogram images and simplify the system pipeline by using a convolutional network as a replacement for many steps traditionally performed in succesion.

\begin{comment}
El {\it Problema} es el núcleo de la propuesta. En esta parte se define y se
justifica clara y ampliamente la situación que se pretende
resolver. Normalmente el problema particular a resolver cae dentro de un
contexto más amplio, dentro de una situación problemática de la cual se
derivan regularmente más de un problema. Los aspectos a considerar en esta
parte consisten en lo siguiente:
\begin{itemize}
	\item Describir la Situación problemática, es decir, identificar los
	problemas o áreas de oportunidad donde se ubica su investigación y los
	antecedentes de esa situación. 
	\item Definir el problema a detalle con sus factores, aspectos, relaciones
  y desarrollar la importancia de ese problema. Debería estar basado en
  literatura.
\end{itemize}
\end{comment}


\section{Objectives}
\label{sec:Objectives}
Yet to write

improve the results obtained with more traiditional methods
Dejar el sistema aqui y el codigo de las convolutional networks so that it could be use don some other tasks or in 3d tommography 
comenzar en deep learning en la institucion Kickstart the work on convolutiopnal netowkr or deep learning in the intitution. 
generate reslts tha culd result in an conference or journal article
Perfomr a careful evaluationn of the convnets to determine what can be improved and work on it. 
Test the different hypothesis and give a concise answer to 
Sauy if this is a mehtid worht to put the resources on, . If it is yes, point to some directions wher eit could be imoporved. if it  is not, poit to some of the problems that are preventing it from doing it. 




\begin{comment}
Especificar en esta sección qué es lo que quiere lograr con respecto al problema identificado
en forma general y particular. Puede incluir alcances y cualquier otro
elemento que considere pertinente para delimitar su trabajo. 


{\bf Por ejemplo:}

El objetivo general de este trabajo.......

Los objetivos particulares a cumplir en este trabajo de investigación son los
siguientes: 
\begin{itemize}
	\item El primer objetivo...
	\item El segundo objetivo...
\end{itemize}

Esta sección puede contener también el {\it Modelo Particular}, que es el modelo de solución propuesto para el
problema y que obviamente debe ser consistente con los objetivos
establecidos. Se le llama {\it Modelo Particular}
 porque es en el cual se guía
el trabajo de investigación y que desemboca en lo que es la
 {\bf CONTRIBUCIÓN PERSONAL}.
Aquí es donde los aspectos de creatividad e innovación deben verse aplicados a nuestro
trabajo.
\end{comment}


\section{Hypothesis}
\label{sec:Hypothesis}
Yet to write

Can we do better than what has alreayd been reported using convnets. can we do better than what has been eported using other methods
Can we simplify the task of image recognition for this task
what are the best parameters ofr teh neural network (number of hidden networks, maxout vs pool, RELus vs logistic, kernel sizes)Is there a big improvement on refining and tuning the nertwork paraameters for the task in hand
How good are the resutls on the enrtire mammogram image?. Is there a way to join the results on the small patche to make a prediction on the patient?
Is the GPu optimization neccesary. 
Will the data be eough or willht network overfit to the small amount of data.
Can the features obtained from a convolotional network trainedon  a different database(like the imageNet database) be used ot o btain results on our images. are thos results better than using a shallow convnet trained on medical images
Are convolutionla netowkrs traine don pixel images better at this task than non convolutional neural networks or other non linear classifiers (SVMs, k-means) trained on handcarfted features?
Is this a good path to keep working on to try to solve these bproblem or sjould we put resources on other methods?
can we achieve human-like performance 
\begin{comment}
Las {\it Hipótesis}, que de acuerdo a Sampieri {\it indican lo que estamos
  buscando 
o tratando de probar y pueden definirse como explicaciones tentativas del
fenómeno investigado y formuladas a manera de proposiciones}. Las hipótesis
surgen normalmente de los {\it Objetivos} y proponen contestar tentativamente
  las preguntas de investigación.

{\bf Las preguntas de investigación se incluyen aquí ......}
\end{comment}


\section{Methodology}
\label{sec:Methodology}
In order to achieve the proposed objectives and test our hypotheses we will need to carry out various tasks. We list them here in the order in which we plan to execute them:

\begin{itemize}
	\item Literature review: A thorough review of the published work using the databases and resources available in the institution. By the end of this task, a complete theoretical background should be obtained and written. This will also help refine the scope of the project and the experiments to be conducted.
	\item Software review: Once a clear idea of what are the possible experiments to be executed, we will need to find appropiate software to perform them. Software for database managing, preprocessing and implementation of different neural networks should be either located or developed.
	\item Database preprocessing: We will ready the database images for the experiments; these implies joining different databases, obtaining the required features, preprocessing the images, assigning labels, etc.
	\item Assesing image preprocessing: We will train a standard convolutional network with fixed parameters on mammograms with three different preprocessings: no preprocessing, image enhancement using median or gaussian filters and wavelet filtered images. Furthermore, we will train a deeper convolutional network on nonpreprocessed images. We want to answer three research questions: which is the best preprocessing for convolutional networks, is using the best filter significantly better than using nonpreprocessed images and can a convolutional network automatically preprocess the images?
% Q: Is it better to make different preprocessings oin the same convolutional network or to fit each convolutional network for each preprocessing, thus, giving it the best chance to perform but taking more time.
	\item Exploratory experiments: We will train standard convolutional networks in two different inputs: small image patches obtained from mammograms and whole mammogram images. We will also train a linear classifier, probably rectified linear units, on the features obtained from a convolutional network trained on the ImageNet database, i.e., we will use an already trained convolutional network instead of one trained specifically in mammograms. Here we will use the image preprocessing technique that showed better results in the previous step. We want to answer two research questions: Can a convolutional network trained on whole mammograms perform as well as one trained on small patches and can we use an already trained convolutional network to classify mammograms?
	\item Model selection: Using the insights from previous sections and the current literature on convolutional networks, we will select a network architecture along with novel features, preprocessing, training and regularization procedures. We aspire to find the best convolutional network configuration for mammogram classification.
	\item Further experiments: We will train the chosen convolutional network on our mammographic database. We will perform crossvalidation to adjust the most important network parameters and use regularization to avoid possible overfitting. We want to answer two research questions: is the performance of the convolutional network considerably improved by parameter tuning and, more importantly, is this a good performance?.
%Maybe train one with no tune fitting.
	\item Gathering results: Produce results on the test set and elaborate figures and tables. This could be obtained directly from software output or from further program executions.
	\item Reporting results: Write the thesis and any article or technical guide which may result from this work. Both this and the previous step will be performed along the execution of the project, hopefully benefiting from the supervisors' feedback.
\end{itemize}
Finally, we want to note that this is an idealized workflow and some changes may occur due to time limitations or resources unavailability. In the unlikely case that the work is finished before the project deadline, we will either reiterate on model selection, experiments, result gathering and reporting or look into digital tomosynthesis, network ensembles or evolving convolutional networks.

\begin{comment}
La {\it Metodología} (o lo que algunos autores llaman el {\it Método})
 es el proceso o
conjunto de pasos que debe efectuarse para llegar a cumplir con los
objetivos. Esos pasos deben contener  los experimentos a realizar, la forma de
llevarlos a cabo, la evaluación de los resultados, la prueba de las hipótesis,
la respuesta a las preguntas de investigación y el último paso debe ser el
reporte escrito de los resultados.
\end{comment}


\section{Contributions}
% Describe the exact contributions of this thesis
Yet to write

\section{Outline of the thesis}
Yet to write


\section{Problem Statement and Motivation}
\label{sec:ProblemDefinition}
Breast cancer is the most commonly diagnosed cancer in woman and its death rates are among the highest of any cancer. It is estimated that about 1 in 8 U.S. women will be diagnosed with breast cancer at some point in their lifetime. Early detection is key in reducing the number of deaths from breast cancer; detection in its earlier stage (\textit{in situ}) increases the survival rate to virtually 100\%~\cite{Howlader2014}.

With current technology, a high quality mammogram is ``the most efective way to detect breast cancer early''~\cite{Mammograms2014}. Mammograms are used by radiologists to search for early signs of cancer such as tumors or microcalcifications. About 85\% of breast cancers can be detected with a screening mammogram~\cite{PerformanceMammography2013}. This high sensitivity is the product of careful examination of the mammograms by experienced radiologists. A computer-aided diagnosis tool (CAD) could automatically detect and diagnose these abnormalities saving the time and training needed by expert radiologists and avoiding any human error. Computer based approaches could also be used by radiologists as a help during the screening proccess or as a second informed opinion on a diagnosis.

CAD systems are based on image and classification techniques coming from Artificial Intelligence and Machine Learning. Traditional CAD tools for breast cancer diagnosis are composed of three steps: feature extraction, feature selection and classification. In the feature extraction phase, the system uses filters and image transformations to preprocess the mammogram and find geometric patterns which are used to produce a set of features for the image; expert knowledge is sometimes used in this phase. Feature selection or regularization is used to focus only on the important features for the classification task. Once a vector of features is obtained for each image, an standard binary classifier can be used to perform the final detection or diagnosis. These techniques have been used for many years and are standard in the industry~\footnote{See~\cite{Hernandez2014} for an example of a CAD system developed in this institution.}.

Despite its widespread use and efficiency, systems based on traditional computer vision techniques have various limitations that should be addressed to further improve its performance:
\begin{itemize}
	\item There is no standard way of preprocessing mammograms. Some techniques are commonly used but their performances can vary.
	\item It uses handcrafted features. The features extracted from the image are chosen beforehand (maybe designed with the help of experts) and special filters and image techniques are used to extract them.
	\item Segmentation and image feature extraction are error-prone and could greatly affect the classification results.
	\item It normally uses a small patch of the mammogram and makes a prediction on that patch but it does not consider the entire mammogram neither to make a prediction on the patient or to account for correlation between patches.
	\item To produce good results it requires knowledge in various fields such as radiology, oncology, image processing, computer vision, machine learning, etc.
	\item It is composed of many sequential steps. At each stage, there are many techniques from which the researcher can choose and many parameters which have to be estimated. This represents a cost in time and results as it is improbable that the optimal selection of techniques and parameters is achieved.
	\item As it is a complex system with different subsystems involved many other issues can arise such as non desired or unknown dependencies between subsystems, difficulty to localize errors, maintainability, etc.  
	\item The techniques currently used are complex but the improvements achieved are not substantial. Much work is needed to make only incremental improvements and it is hard to know to which part of the system dedicate more resources.
\end{itemize}

This project will center around using Convolutional Networks, a recent development in computer vision, (see Section~\ref{subsec:ConvNets}) to tackle some of these limitations, especifically automate preprocessing, feature extraction and segementation, use entire mammogram images and simplify the system pipeline by using a convolutional network as a replacement for many steps traditionally performed in succesion.

\begin{comment}
El {\it Problema} es el núcleo de la propuesta. En esta parte se define y se
justifica clara y ampliamente la situación que se pretende
resolver. Normalmente el problema particular a resolver cae dentro de un
contexto más amplio, dentro de una situación problemática de la cual se
derivan regularmente más de un problema. Los aspectos a considerar en esta
parte consisten en lo siguiente:
\begin{itemize}
	\item Describir la Situación problemática, es decir, identificar los
	problemas o áreas de oportunidad donde se ubica su investigación y los
	antecedentes de esa situación. 
	\item Definir el problema a detalle con sus factores, aspectos, relaciones
  y desarrollar la importancia de ese problema. Debería estar basado en
  literatura.
\end{itemize}
\end{comment}


\section{Objectives}
\label{sec:Objectives}
Yet to write

improve the results obtained with more traiditional methods
Dejar el sistema aqui y el codigo de las convolutional networks so that it could be use don some other tasks or in 3d tommography 
comenzar en deep learning en la institucion Kickstart the work on convolutiopnal netowkr or deep learning in the intitution. 
generate reslts tha culd result in an conference or journal article
Perfomr a careful evaluationn of the convnets to determine what can be improved and work on it. 
Test the different hypothesis and give a concise answer to 
Sauy if this is a mehtid worht to put the resources on, . If it is yes, point to some directions wher eit could be imoporved. if it  is not, poit to some of the problems that are preventing it from doing it. 




\begin{comment}
Especificar en esta sección qué es lo que quiere lograr con respecto al problema identificado
en forma general y particular. Puede incluir alcances y cualquier otro
elemento que considere pertinente para delimitar su trabajo. 


{\bf Por ejemplo:}

El objetivo general de este trabajo.......

Los objetivos particulares a cumplir en este trabajo de investigación son los
siguientes: 
\begin{itemize}
	\item El primer objetivo...
	\item El segundo objetivo...
\end{itemize}

Esta sección puede contener también el {\it Modelo Particular}, que es el modelo de solución propuesto para el
problema y que obviamente debe ser consistente con los objetivos
establecidos. Se le llama {\it Modelo Particular}
 porque es en el cual se guía
el trabajo de investigación y que desemboca en lo que es la
 {\bf CONTRIBUCIÓN PERSONAL}.
Aquí es donde los aspectos de creatividad e innovación deben verse aplicados a nuestro
trabajo.
\end{comment}


\section{Hypothesis}
\label{sec:Hypothesis}
Yet to write

Can we do better than what has alreayd been reported using convnets. can we do better than what has been eported using other methods
Can we simplify the task of image recognition for this task
what are the best parameters ofr teh neural network (number of hidden networks, maxout vs pool, RELus vs logistic, kernel sizes)Is there a big improvement on refining and tuning the nertwork paraameters for the task in hand
How good are the resutls on the enrtire mammogram image?. Is there a way to join the results on the small patche to make a prediction on the patient?
Is the GPu optimization neccesary. 
Will the data be eough or willht network overfit to the small amount of data.
Can the features obtained from a convolotional network trainedon  a different database(like the imageNet database) be used ot o btain results on our images. are thos results better than using a shallow convnet trained on medical images
Are convolutionla netowkrs traine don pixel images better at this task than non convolutional neural networks or other non linear classifiers (SVMs, k-means) trained on handcarfted features?
Is this a good path to keep working on to try to solve these bproblem or sjould we put resources on other methods?
can we achieve human-like performance 
\begin{comment}
Las {\it Hipótesis}, que de acuerdo a Sampieri {\it indican lo que estamos
  buscando 
o tratando de probar y pueden definirse como explicaciones tentativas del
fenómeno investigado y formuladas a manera de proposiciones}. Las hipótesis
surgen normalmente de los {\it Objetivos} y proponen contestar tentativamente
  las preguntas de investigación.

{\bf Las preguntas de investigación se incluyen aquí ......}
\end{comment}


\section{Methodology}
\label{sec:Methodology}
In order to achieve the proposed objectives and test our hypotheses we will need to carry out various tasks. We list them here in the order in which we plan to execute them:

\begin{itemize}
	\item Literature review: A thorough review of the published work using the databases and resources available in the institution. By the end of this task, a complete theoretical background should be obtained and written. This will also help refine the scope of the project and the experiments to be conducted.
	\item Software review: Once a clear idea of what are the possible experiments to be executed, we will need to find appropiate software to perform them. Software for database managing, preprocessing and implementation of different neural networks should be either located or developed.
	\item Database preprocessing: We will ready the database images for the experiments; these implies joining different databases, obtaining the required features, preprocessing the images, assigning labels, etc.
	\item Assesing image preprocessing: We will train a standard convolutional network with fixed parameters on mammograms with three different preprocessings: no preprocessing, image enhancement using median or gaussian filters and wavelet filtered images. Furthermore, we will train a deeper convolutional network on nonpreprocessed images. We want to answer three research questions: which is the best preprocessing for convolutional networks, is using the best filter significantly better than using nonpreprocessed images and can a convolutional network automatically preprocess the images?
% Q: Is it better to make different preprocessings oin the same convolutional network or to fit each convolutional network for each preprocessing, thus, giving it the best chance to perform but taking more time.
	\item Exploratory experiments: We will train standard convolutional networks in two different inputs: small image patches obtained from mammograms and whole mammogram images. We will also train a linear classifier, probably rectified linear units, on the features obtained from a convolutional network trained on the ImageNet database, i.e., we will use an already trained convolutional network instead of one trained specifically in mammograms. Here we will use the image preprocessing technique that showed better results in the previous step. We want to answer two research questions: Can a convolutional network trained on whole mammograms perform as well as one trained on small patches and can we use an already trained convolutional network to classify mammograms?
	\item Model selection: Using the insights from previous sections and the current literature on convolutional networks, we will select a network architecture along with novel features, preprocessing, training and regularization procedures. We aspire to find the best convolutional network configuration for mammogram classification.
	\item Further experiments: We will train the chosen convolutional network on our mammographic database. We will perform crossvalidation to adjust the most important network parameters and use regularization to avoid possible overfitting. We want to answer two research questions: is the performance of the convolutional network considerably improved by parameter tuning and, more importantly, is this a good performance?.
%Maybe train one with no tune fitting.
	\item Gathering results: Produce results on the test set and elaborate figures and tables. This could be obtained directly from software output or from further program executions.
	\item Reporting results: Write the thesis and any article or technical guide which may result from this work. Both this and the previous step will be performed along the execution of the project, hopefully benefiting from the supervisors' feedback.
\end{itemize}
Finally, we want to note that this is an idealized workflow and some changes may occur due to time limitations or resources unavailability. In the unlikely case that the work is finished before the project deadline, we will either reiterate on model selection, experiments, result gathering and reporting or look into digital tomosynthesis, network ensembles or evolving convolutional networks.

\begin{comment}
La {\it Metodología} (o lo que algunos autores llaman el {\it Método})
 es el proceso o
conjunto de pasos que debe efectuarse para llegar a cumplir con los
objetivos. Esos pasos deben contener  los experimentos a realizar, la forma de
llevarlos a cabo, la evaluación de los resultados, la prueba de las hipótesis,
la respuesta a las preguntas de investigación y el último paso debe ser el
reporte escrito de los resultados.
\end{comment}


\section{Contributions}
% Describe the exact contributions of this thesis
Yet to write

\section{Outline of the thesis}
Yet to write


\section{Problem Statement and Motivation}
\label{sec:ProblemDefinition}
Breast cancer is the most commonly diagnosed cancer in woman and its death rates are among the highest of any cancer. It is estimated that about 1 in 8 U.S. women will be diagnosed with breast cancer at some point in their lifetime. Early detection is key in reducing the number of deaths from breast cancer; detection in its earlier stage (\textit{in situ}) increases the survival rate to virtually 100\%~\cite{Howlader2014}.

With current technology, a high quality mammogram is ``the most efective way to detect breast cancer early''~\cite{Mammograms2014}. Mammograms are used by radiologists to search for early signs of cancer such as tumors or microcalcifications. About 85\% of breast cancers can be detected with a screening mammogram~\cite{PerformanceMammography2013}. This high sensitivity is the product of careful examination of the mammograms by experienced radiologists. A computer-aided diagnosis tool (CAD) could automatically detect and diagnose these abnormalities saving the time and training needed by expert radiologists and avoiding any human error. Computer based approaches could also be used by radiologists as a help during the screening proccess or as a second informed opinion on a diagnosis.

CAD systems are based on image and classification techniques coming from Artificial Intelligence and Machine Learning. Traditional CAD tools for breast cancer diagnosis are composed of three steps: feature extraction, feature selection and classification. In the feature extraction phase, the system uses filters and image transformations to preprocess the mammogram and find geometric patterns which are used to produce a set of features for the image; expert knowledge is sometimes used in this phase. Feature selection or regularization is used to focus only on the important features for the classification task. Once a vector of features is obtained for each image, an standard binary classifier can be used to perform the final detection or diagnosis. These techniques have been used for many years and are standard in the industry~\footnote{See~\cite{Hernandez2014} for an example of a CAD system developed in this institution.}.

Despite its widespread use and efficiency, systems based on traditional computer vision techniques have various limitations that should be addressed to further improve its performance:
\begin{itemize}
	\item There is no standard way of preprocessing mammograms. Some techniques are commonly used but their performances can vary.
	\item It uses handcrafted features. The features extracted from the image are chosen beforehand (maybe designed with the help of experts) and special filters and image techniques are used to extract them.
	\item Segmentation and image feature extraction are error-prone and could greatly affect the classification results.
	\item It normally uses a small patch of the mammogram and makes a prediction on that patch but it does not consider the entire mammogram neither to make a prediction on the patient or to account for correlation between patches.
	\item To produce good results it requires knowledge in various fields such as radiology, oncology, image processing, computer vision, machine learning, etc.
	\item It is composed of many sequential steps. At each stage, there are many techniques from which the researcher can choose and many parameters which have to be estimated. This represents a cost in time and results as it is improbable that the optimal selection of techniques and parameters is achieved.
	\item As it is a complex system with different subsystems involved many other issues can arise such as non desired or unknown dependencies between subsystems, difficulty to localize errors, maintainability, etc.  
	\item The techniques currently used are complex but the improvements achieved are not substantial. Much work is needed to make only incremental improvements and it is hard to know to which part of the system dedicate more resources.
\end{itemize}

This project will center around using Convolutional Networks, a recent development in computer vision, (see Section~\ref{subsec:ConvNets}) to tackle some of these limitations, especifically automate preprocessing, feature extraction and segementation, use entire mammogram images and simplify the system pipeline by using a convolutional network as a replacement for many steps traditionally performed in succesion.

\begin{comment}
El {\it Problema} es el núcleo de la propuesta. En esta parte se define y se
justifica clara y ampliamente la situación que se pretende
resolver. Normalmente el problema particular a resolver cae dentro de un
contexto más amplio, dentro de una situación problemática de la cual se
derivan regularmente más de un problema. Los aspectos a considerar en esta
parte consisten en lo siguiente:
\begin{itemize}
	\item Describir la Situación problemática, es decir, identificar los
	problemas o áreas de oportunidad donde se ubica su investigación y los
	antecedentes de esa situación. 
	\item Definir el problema a detalle con sus factores, aspectos, relaciones
  y desarrollar la importancia de ese problema. Debería estar basado en
  literatura.
\end{itemize}
\end{comment}


\section{Objectives}
\label{sec:Objectives}
Yet to write

improve the results obtained with more traiditional methods
Dejar el sistema aqui y el codigo de las convolutional networks so that it could be use don some other tasks or in 3d tommography 
comenzar en deep learning en la institucion Kickstart the work on convolutiopnal netowkr or deep learning in the intitution. 
generate reslts tha culd result in an conference or journal article
Perfomr a careful evaluationn of the convnets to determine what can be improved and work on it. 
Test the different hypothesis and give a concise answer to 
Sauy if this is a mehtid worht to put the resources on, . If it is yes, point to some directions wher eit could be imoporved. if it  is not, poit to some of the problems that are preventing it from doing it. 




\begin{comment}
Especificar en esta sección qué es lo que quiere lograr con respecto al problema identificado
en forma general y particular. Puede incluir alcances y cualquier otro
elemento que considere pertinente para delimitar su trabajo. 


{\bf Por ejemplo:}

El objetivo general de este trabajo.......

Los objetivos particulares a cumplir en este trabajo de investigación son los
siguientes: 
\begin{itemize}
	\item El primer objetivo...
	\item El segundo objetivo...
\end{itemize}

Esta sección puede contener también el {\it Modelo Particular}, que es el modelo de solución propuesto para el
problema y que obviamente debe ser consistente con los objetivos
establecidos. Se le llama {\it Modelo Particular}
 porque es en el cual se guía
el trabajo de investigación y que desemboca en lo que es la
 {\bf CONTRIBUCIÓN PERSONAL}.
Aquí es donde los aspectos de creatividad e innovación deben verse aplicados a nuestro
trabajo.
\end{comment}


\section{Hypothesis}
\label{sec:Hypothesis}
Yet to write

Can we do better than what has alreayd been reported using convnets. can we do better than what has been eported using other methods
Can we simplify the task of image recognition for this task
what are the best parameters ofr teh neural network (number of hidden networks, maxout vs pool, RELus vs logistic, kernel sizes)Is there a big improvement on refining and tuning the nertwork paraameters for the task in hand
How good are the resutls on the enrtire mammogram image?. Is there a way to join the results on the small patche to make a prediction on the patient?
Is the GPu optimization neccesary. 
Will the data be eough or willht network overfit to the small amount of data.
Can the features obtained from a convolotional network trainedon  a different database(like the imageNet database) be used ot o btain results on our images. are thos results better than using a shallow convnet trained on medical images
Are convolutionla netowkrs traine don pixel images better at this task than non convolutional neural networks or other non linear classifiers (SVMs, k-means) trained on handcarfted features?
Is this a good path to keep working on to try to solve these bproblem or sjould we put resources on other methods?
can we achieve human-like performance 
\begin{comment}
Las {\it Hipótesis}, que de acuerdo a Sampieri {\it indican lo que estamos
  buscando 
o tratando de probar y pueden definirse como explicaciones tentativas del
fenómeno investigado y formuladas a manera de proposiciones}. Las hipótesis
surgen normalmente de los {\it Objetivos} y proponen contestar tentativamente
  las preguntas de investigación.

{\bf Las preguntas de investigación se incluyen aquí ......}
\end{comment}


\section{Background}
\label{sec:Background}
We offer an introduction to some of the essential concepts needed to understand the rest of this document. We start by discussing breast cancer and mammograms in Section~\ref{subsec:BreastCancer}, we explore some basic concepts about classification and evaluation metrics in Section~\ref{subsec:Classification}, in Sections~\ref{subsec:ANNs} and~\ref{subsec:ConvNets} we give a short introduction into Artificial Neural Networks and Convolutional Neural Networks, we present an overview of how convolutional networks have been used for breast cancer diagnosis in Section~\ref{subsec:BreastCancerConvNets} and finally we offer some practical advice for deep learning in Section~\ref{subsec:PracticalDL}.

	\subsection{Breast Cancer}
	\label{subsec:BreastCancer}
	\emph{Cancer} is an umbrella term for a group of diseases caused by abnormal cell growth in different parts of the body. The accumulation of extra cells usually forms a mass of tissue called a \emph{tumor}. Tumors can be benign or malignant: \emph{benign tumors} are noncancerous, lack the ability to invade surrounding tissue and will not regrow if removed from the body;  malignant or \emph{cancerous tumors} are harmful, can invade nearby organs and tissues (\emph{invasive cancer}), can spread to other parts of the body (\emph{metastasis}) and will sometimes regrow when removed~\cite{WYNTKABreastCancer2012}.

\emph{Breast cancer} forms in tissues of the breast. The two most common types of breast cancer are \emph{ductal carcinoma} and \emph{lobular carcinoma}, which start in the breast ducts and lobules, respectively (see Fig.~\ref{fig:BreastAnatomy}). Breast cancer \emph{incidence rate}, the number of new cases in a specified population during a year, is the highest of any cancer among American women. Its \emph{mortality rate}, the number of deaths during a year, is also one of the highest of any cancer~\cite{Howlader2014}.

\begin{figure}[h]
	\centering
	\includegraphics[width = 0.35\textwidth]{plots/breastAnatomy.png}
	\caption[Female Breast Anatomy]{Anatomy of the female breast. Image courtesy of NCI.}
	\label{fig:BreastAnatomy}
\end{figure}

The \emph{cancer stage} depends on the size of the tumor and whether the cancer cells have spread to neighboring tissue or other parts of the body. It is expressed as a Roman numeral ranging from 0 through IV; stage I cancer is considered \emph{early-stage breast cancer} and stage IV cancer is considered \emph{advanced}. Stage 0 describes non-invasive breast cancers, also known as \emph{carcinoma in situ}. Stage I, II and III describe invasive breast cancer, i.e., cancer has invaded normal surrounding breast tissue. Stage IV is used to describe metastatic cancer, i.e., it has spread beyond nearby tissue to other organs of the body.

\subsubsection{Mammograms}
A \emph{mammogram} is an x-ray image of the breast. Radiologists use \emph{screening mammograms} (normally composed of two mammograms of each breast) to check for breast cancer signs on women who lack symptoms of the disease. If an abnormality is found, a \emph{diagnostic mammogram} is ordered, these are detailed x-ray pictures of the suspicious region~\cite{Mammograms2014}. A standard mammogram is shown in Fig.~\ref{fig:normalMammogram}.

\begin{figure}[h]
	\centering
	\includegraphics[width = 0.25\textwidth]{plots/normalMammogram.jpg}
	\caption[Digital Mammogram]{A standard mammogram.}
	\label{fig:normalMammogram}
\end{figure}

Having a screening mammogram in a regular basis is the most effective method for detecting breast cancer early; around 85\% of breast cancers can be detected in a screening mammogram~\cite{PerformanceMammography2013}. Nevertheless, screening mammograms have many limitations: a high false positive rate, overtreatment in Stage 0 cancer, false negative results for women with high breast density, radiation exposure and physical and psychological discomfort~\cite{Mammograms2014}.

Radiologists look primarily for microcalcifications and breast masses. \emph{Microcalcifications} are tiny deposits of calcium in the breast tissue that can be a sign of early breast cancer if found in clusters with irregular layout and shapes. \emph{Breast masses} or breast lumps are a variety of things: fluid-filled cysts, fatty tissues, fibric tissues, noncancerous or cancerous tumors, among others. A mass can be a sign of breast cancer if it has an irregular shape and poorly defined margins. See Fig.~\ref{fig:breastCancerSigns} for an example of possible signs of breast cancer. Radiologists will also consider the breast density of the patient when reading a mammogram given that high breast density is linked to a higher risk of breast cancer~\cite{MammogramsACS2014}.

\begin{figure}[h]
	\centering
	\begin{subfigure}{0.25\textwidth}
                \includegraphics[width=\textwidth]{plots/breastMicrocalcification.jpg}
        \end{subfigure}
	~
	\begin{subfigure}{0.25\textwidth}
                \includegraphics[width=\textwidth]{plots/breastMass.jpg}
        \end{subfigure}
	\caption[Breast Cancer Signs]{Signs of possible breast cancer in a mammogram. Left: A cluster of microcalcifications in an irregular layout. Right: A poorly defined breast mass.}
	\label{fig:breastCancerSigns}
\end{figure}

Conventional mammography uses film to record x-ray images of the breast. \emph{Digital mammography}, on the other hand, uses digital receptors to convert the x-rays into electrical signals and stores the image electronically. Digital mammograms offer a clearer picture of the breast and can be digitally manipulated and shared between health care providers. However, researchers still debate its effectiveness to identify breast cancer over film mammograms~\cite{Kerlikowske2011, Pisano2008, Skaane2007}. Digital mammography is steadily becoming the standard for breast cancer screening. Fig.~\ref{fig:normalMammogram} is, in fact, a digital mammogram.

\emph{Digital tomosynthesis}, also called three-dimensional mammography, is a new technology that produces 3-dimensional x-ray images of the breast and is expected to improve the efficacy of regular 2-d mammograms. Studies comparing the two techniques have not yet been published~\cite{Mammograms2014}.

We center on using mammograms, either digital or digitized from film, to detect microcalcifications and masses and predict the likelihood of breast cancer on the patient.

We wrote most of this section using information from the National Cancer Institute. We recommend to visit its website (\url{www.cancer.gov}) for further details.


	\subsection{Classification}
	\label{subsec:Classification}
	%\emph{Machine learning} [is the study of|studies] algorithms that [build|create][mathematical] models of a population or [a] function of interest and estimate their parameters from data  in order to make [good] predictions or inferences.
\emph{Machine learning} is the study of algorithms that build models of a population or function of interest estimating their parameters from data in order to make predictions or inferences. A machine learning expert knows how to choose the right model for the problem in hand (\emph{model selection}), how to efficiently estimate its parameters from the available data (\emph{learning} or \emph{training phase}) and how to evaluate the trained model (\emph{testing phase}).

Machine learning problems divide into three categories depending on the data used to train the model: \emph{supervised learning}, where we learn a function $f(x)$ using examples labelled with their correct output, for instance, learning to estimate the price of a house given its size and number of bedrooms from a data set of houses and their true valuations; \emph{unsupervised learning}, where we look for relationships and structure in unlabelled data, for instance, given a data set of potential customers finding those who are likely to buy and \emph{reinforcement learning}, where feedback is received intermittently, for instance, learning to play Tetris from a data set of world states, actions and rewards received only when points are earned. Supervised learning further divides in regression and classification. If the expected output is numerical, e.g., the price of a house, it is called \emph{regression}, if the expected ouput is categorical, e.g., spam or no spam, it is called \emph{classification}. We focus on classification.

A \emph{classifier} takes as input a vector of \emph{features} $x \in \mathbb{R}^n$ representing a problem instance and produces an \emph{output} $h(x)$ predicting the class $y$ to which that instance belongs, i.e., it models the underlying function $f(x)$ as $h(x)$ ($h$ stands for hypothesis). \emph{Binary classification}, when $y$ can only take two values e.g., cancer/no cancer, is the most common kind of classification and \emph{multiclass classification}, when $y$ can take $K > 2$ different values, can be performed by using $K$ binary classifiers. Some classifiers, such as convolutional networks (Sec.~\ref{sec:ConvNets}), output a \emph{score vector} $h(x) \in \mathbb{R}^K$ where $h(x)_k$ measures the likelihood of $x$ belonging to class $k$. Every classifier partitions the \emph{feature space}, the $n$-dimensional space where features exist, into separate \emph{decision regions}, regions of the space that are assigned the same outcome; a \emph{decision boundary} is the hypersurface that partitions the feature space. Classifiers are sometimes classified as \emph{linear} or \emph{nonlinear} according to the nature of the decision boundary they impose on the feature space. Logistic regression, for instance, is a linear classifier while an artificial neural network with one or more hidden layers is nonlinear.
% A linear classifier can separate perfectly linear data, while for nonlinear data a more powerful classifier is needed. Linearly separable data are those which can be classified by a linear classifier while nonlinear data can not.

The \emph{loss function} $L(\theta)$ of a classifier measures the amount of error the classifier incurs in for a particular choice of parameters $\theta$. This function could be formulated in many ways. A \emph{least-squares loss function} for a binary classifier (such as logistic regression) is presented in Equation~\ref{eq:LossFunction}:
\begin{equation}
	L(\theta) = \frac{1}{2m}\sum_{i=1}^m(y^{(i)} - h_\theta(x^{(i)}))^2
	\label{eq:LossFunction}
\end{equation}
where $m$ is the number of training examples, $y \in \{0,1\}$ is the real class of example $x$ and $h_\theta(x) \in \mathbb{R}$ is the output of the classifier for input $x$ with parameters $\theta$, this represents the probability that $x$ belongs to the positive class 1. We introduce another (rather more complex) loss function in the next section.

A classifier is trained by choosing the parameters $\theta$ that minimize its loss function, hence, minimizing the expected error of the classifier on the training set. \emph{Gradient descent} estimates these parameters by initializing them at random and iteratively updating them using the gradient of the loss function. Specifically, at each iteration it performs the update:
\begin{equation}
	\theta = \theta - \alpha \nabla{L(\theta)}
\end{equation}
where $\alpha$, called the \emph{learning rate}, defines the step size. Gradient descent is guaranteed to converge to a global minimum if the loss function is convex, which depends on the model $h(x)$.

To select the best model $h(x)$ for a particular problem, or equivalently, to select the best classifier for the problem, we train each candidate on a subset of the data and evaluate it on a disjoint subset to estimate their performance. In the {validation set approach} the data set is split into a training set (usually 60-90\%) and a validation set, each model is trained using the training set, evaluated on the validation set and the best-performing model is selected. \emph{k-fold cross validation}, on the other hand, divides the data set in $k$ disjoint subsets (usually 5 or 10) and uses $k-1$ subsets to train the model and the remaining subset for evaluation, this process is repeated $k$ times for each model leaving out a different subset each time and the $k$ performance measures are averaged to obtain a final measure for the model.
%Cross validation produces better error estimates but is computationally costly.
\emph{Model hyperparameters}, settings that adjust the underlying model or learning algorithm, can be selected similarly.

The model representation $h(x)$ needs to be chosen carefully. If we have an overly \emph{flexible} model, i.e, $h(x)$ is a complex function with many parameters to be learned relative to the size of the training set, the classifier will \emph{overfit} the data; this means that parameters are fitted too tightly to the training set and pick up small fluctuations and noise causing the classifier to produce almost-perfect results on the training set but perform poorly on unseen examples. The opposite is also true, when $h(x)$ is very simple the classifier lacks the power to model the underlying function of interest and we say that it \emph{underfits} the data.% This problem is sometimes referred as the \emph{bias-variance tradeoff}. A high variance classifier is prone to overfitting, while a high bias classifier is prone to underfitting.

A popular way to avoid overfitting (and underfitting) is to use a flexible model trained with regularization. \emph{Regularization} modifies the loss function to penalize the complexity of the model, forcing the learning stage to choose parameters that minimize both the training error of the classifier and the complexity of the model. Equation~\ref{eq:l2NormRegularization} shows the least-squares loss function with \emph{$l_2$-norm regularization}:
\begin{equation}
	L(\theta) =  \frac{1}{2m}\sum_{i=1}^m(y^{(i)} - h_\theta(x^{(i)}))^2 + \frac{\lambda}{2m} ||\theta||_2
	\label{eq:l2NormRegularization}
\end{equation}
where $||\cdot||_2$ is the euclidean norm of a vector. In addition to reducing training error, minimizing the regularized loss function will shrinken the parameters $\theta$ hopefully setting some of them to zero and simplifying $h(x)$. The \emph{regularization strength} $\lambda$ regulates the tradeoff between training error and regularization error. \emph{$l_1$-norm regularization} or \emph{lasso} is defined similarly except that it shrinks the $l_1$-norm of $\theta$ rather than the $l_2$-norm.

%\subsection{Evaluation metrics}
We evaluate the performance of a classifier on a separate set of examples, a test set, that should have not been used for training or validation. The standard performance measure in machine learning is classification accuracy; \emph{accuracy} measures the proportion of test set examples correctly classified. Its compliment, \emph{error rate}, measures the proportion of test set examples incorrectly classified. Accuracy, nonetheless, is inappropiate for \emph{unbalanced data sets}, data sets that have many more examples of one class than the other~\footnote{Medical data sets are often unbalanced as most examples belong to the negative class (no disease) than the positive class (disease)}. A classifier that always predicts the predominant class regardless of the input is highly accurate (it is right most of the time) even though it is a bad model for the problem.

In unbalanced data sets, we use metrics based on the confusion matrix of the classifier. A \emph{confusion matrix} summarizes the results of a classifier in the test set (Tab.~\ref{tab:ConfusionMatrix}).
\begin{table}[h]
	\centering
	\begin{tabular}{cc|c|c|}
		\multicolumn{2}{c}{}&\multicolumn{2}{c}{\textbf{Actual class}}\\
		&&Positive & Negative \\
		\cline{2-4}
		\textbf{Predicted}&Positive&True Positives (TP)& False Positives (FP)\\
		\cline{2-4}
		\textbf{class}&Negative&False Negatives (FN) & True Negatives (TN)\\
		\cline{2-4}
	\end{tabular}
	\caption{Confusion matrix for a binary classifier}
	\label{tab:ConfusionMatrix}
\end{table}
\emph{True positives} is the number of positive examples correctly predicted as positive. \emph{False positives} is the number of negative examples incorrectly predicted as positive. True negatives and false negatives are defined similarly. Based on the confusion matrix we can compute some commonly used metrics:
\begin{equation}
	Sensitivity \text{ or } Recall = \frac{TP}{TP+FN}
\end{equation}
\begin{equation}
	Specificity = \frac{TN}{FP+TN}
\end{equation}
\begin{equation}
	Precision = \frac{TP}{TP+FP}
\end{equation}
\emph{Sensitivity} measures the proportion of positive examples predicted as positive and \emph{specificity} measures the proportion of negative examples predicted as negative. \emph{Precision} measures the proportion of examples predicted as positive that are actually positive. A good classifier will have both high sensitivity and high specificity or similarly, high precision and high recall. Sensitivity and specificity are preferred in medical diagnosis while precision and recall are preferred in machine learning. 

It is often useful to have a single metric to evaluate classifiers, for example, to choose between two models; we show two commonly used metrics in Equation~\ref{eq:F1Score} and~\ref{eq:G-mean}.
\begin{equation}
	F_1\text{ }score = 2\times\frac{Precision \times Recall}{Precision + Recall}
	\label{eq:F1Score}
\end{equation}
\begin{equation}
	G\text{-}mean = \sqrt{Sensitivity \times Specificity}
	\label{eq:G-mean}
\end{equation}

The \emph{threshold} of a classifier is the probability at and over which an example is classified as positive. It regulates the trade-off between sensitivity and specificity (or similarly precision and recall): a classifier with a low threshold is prone to classify examples as positive but will potentially produce many false positives thus having high sensitivity but low specificity and viceversa for high thresholds. The \emph{precision-recall curve} of a classifier is a plot of its precision (on the y axis) against its recall (on the x axis) as the threshold varies (Fig.~\ref{fig:AUCandPRAUC}). The \emph{receiver operating characteristic curve} plots sensitivity (also called true positive rate) against 1-specificity (also called false positive rate) as the threshold varies (Fig.~\ref{fig:AUCandPRAUC}). The \emph{area under the precision-recall curve} PRAUC and the \emph{area under the receiver operating characteristic curve} AUC summarize the performance of the classifier over all possible thresholds and can also be used for model selection; they range from 0 to 1 with higher being better. As with previous metrics, AUC is preferred for medical diagnosis while PRAUC is used mostly in machine learning. 
\begin{figure}[h]
	\centering
	\includegraphics[width = 0.83\textwidth]{plots/AUCandPRAUC.png}
	\caption[Sample ROC and PR curves]{A sample receiver operating characteristic curve (left) and precision-recall curve (right). Each algorithm is evaluated on different thresholds and the points produced are used to obtain the curves. Image courtesy of~\cite{Davis2006}}
	\label{fig:AUCandPRAUC}
\end{figure}

For unbalanced data sets, ``using the classifiers produced by standard machine learning algorithms without adjusting the output threshold may well be a critical mistake''~\cite{Provost2000}. It is preferable to use metrics that consider all possible thresholds (AUC or PRAUC) or simpler metrics ($F_1$ score or G-mean) with a threshold obtained via a validation set. The metric used for model selection influences its characteristics and behaviour, hence, it should be chosen carefully: we favor the use of PRAUC over AUC as well as $F_1$ score over G-mean because they concentrate in the positive class (disease) that is more interesting and harder to predict. Furthermore, PRAUC has been shown to have better properties than AUC in unbalanced data sets~\cite{Davis2006}. % In general, results for all these metrics.
We introduce metrics tailored to image segmentation models in Sec.~\ref{sec:Segmentation}.
%F1 better represents a more balanced tradeoff  (an small change in precision is corresponded with a small change in recall) than AUC where an small change in specificity can be corresponded with a big change in sensitivity.

\begin{comment} Discussion of why G-mean over F1
(Not sure about this) As a rule of thumb, using G-mean will generate models that predict more positives given that the sensitivity will greatly improve and specificity will only slightly decrease. Using $F_1$ score, the model will predict less positives as that will improve precision but only slightly decrease sensitivity. 
5 here say PRAUc is better for unbalanced classes

Why G-means? Because it is more important to obtain a low error in specificity than in precision, i .e, would you prefer a 90% in specificity or a 90% in precision?. 90% in specificity means that 10% of actual negatives (10 persons) were told they have no cancer although they actually had cancer, meanwhile 10% of expected positives(a small number, maybe 1) was said he has cancer although he doesn{t. First is worse.

Using G-mean i will predict more positives, no matter what. My sensitivity is going to vastly improve and the specificity will only decrease a little, but the precision is gonna take a hit. Because if I predict less positives sensitivity is gonna go down, specificity is gonna go up (as I add more true negatives) but just a little and precision  would go up (but it wouldn't matter for g-mean).

Using F1 I'll probably predict less positives, sensitivity is going to go slightly down, and precision is going to go up, specificity doesn{t matter but it will decrease a little. Or I'll probably predcit as many poositives as needed. It focuses more on the positive class.

Other diagonal, the algorithm will learn negatives pretty well, so the one that predicts less positives(f1) probably isn{t learning much (it is predicting all negative).
\end{comment}

%We point out that this section introduces basic concepts in machine learning but leaves aside many practical details. Content and notation is based on materials from Stanford's Machine Learning course~\cite{Ng2014}.


	\subsection{Artificial Neural Networks}
	\label{subsec:ANNs}
	\emph{Artificial neural networks} or simply \emph{neural networks} are one of the most popular nonlinear classifiers used today. They were initially inspired by the way biological neurons process information coming from its dendrites and relaying it through its axon to neighboring neurons~\cite{McCulloch1943, Widrow1960, Rosenblatt1962} but evolved to become practical for nonlinear modelling albeit less biologically accurate~\cite{Rumelhart1986}. We discuss here multilayer feedforward neural networks, the name should become obvious after a few paragraphs.

\emph{Multilayer feedforward neural networks} are composed of $L$ layers of \emph{neurons}, units of computation, each of which is fully connected to the next and previous layer (except for the first and last layer). The first layer, called the \emph{input layer}, has $s^{(1)} = n$ units and receives the feature vector $x \in \mathbb{R}^n$ meanwhile the last layer or \emph{output layer} has $s^{(L)} = K$ units corresponding to the $K$ possible classes (or $1$ unit for binary classification). Every other layer is called a \emph{hidden layer} (see Fig.~\ref{fig:NeuralNetwork} for an example). The neural network receives an input $x \in \mathbb{R}^n$, processes it layer by layer and outputs a vector $h_\Theta(x) \in \mathbb{R}^K$, where $h_\Theta(x)_i$ is the predicted probability that $x$ belongs to class $i$. Each unit performs a computation on the input from the units in the previous layer and transmits the result to the units in the next layer through its connections. Furthermore, each connection has a \emph{weight} $w$ which is to be learned in the training phase, i.e, the weights are the parameters $\Theta$ of the model. A neural network is said to be \emph{shallow} or \emph{deep} according to its number of layers or \emph{depth}.\footnote{There is no consensus on when a neural network becomes a deep neural network\cite{Schmidhuber2015}. We consider networks with over 5 layers to be deep.}

\begin{figure}[h]
	\centering
	\includegraphics[width = \textwidth]{plots/neuralNetwork.png}
	\caption[Example of an Artificial Neural Network]{Small neural network example. Input layer with 4 units (blue), two hidden layers of 5 units (green) and output layer of 3 units(red). Bias units appear in gray. It approximates a function $h_\Theta(x): \mathbb{R}^4 \to \mathbb{R}^3$, i.e., it classifies an input vector $x \in \mathbb{R}^4$ into 3 possible classes.}
	\label{fig:NeuralNetwork}
\end{figure}

A unit $i$ in layer $l$ computes a function of the form:
\begin{equation}
	a^{(l)}_i = g \left(\sum_{j=0}^{s^{(l-1)}} \Theta^{(l-1)}_{ij}a_j^{(l-1)}\right)
	\label{eq:NeuronActivation}
\end{equation}
where $a^{(l)}_i$ is called the \emph{activation} or output of unit $i$ in layer $l$; $g(\cdot)$ is an \emph{activation function} (defined below); $s^{(l)}$ is the number of units in layer $l$, $a^{(u)}_0 = 1$, for all $u = 1, \ldots, L-1$ (bias units); $a^{(1)}_v = x_v$ for all $v = 1, \ldots, n$ i.e, the activation of the input layer is the input $x$, and $\Theta^{(l)} \in \mathbb{R}^{s^{l+1} \times s^{l}} $ is the matrix of weights connecting layer $l$ to $l+1$. At each layer (except the output layer) we include a unit which always emits activation 1 ($a^{(1)}_0 = 1$, $a^{(2)}_0 = 1$, etc), these are called \emph{bias units}~
\footnote{They are included for a technical detail: so that the activation function $g(w^Tx+w_0)$ can shift in the x-axis changing its threshold to $-w_0$, which is learned by the neural network.}.
The bias units are assumed to be included into each vector $a^{(l)}$, hence the sumation in Equation~\ref{eq:NeuronActivation} starts at 0 and not 1. It may seem like a convoluted definition but it simply defines the activation of a given unit as the weighted linear combination of activations of the units in the previous layer passed through a nonlinear function $g(\cdot)$. Lastly, notice that $a^{(L)} \in \mathbb{R}^{s^L}$, the vector of activations in the last layer of the network, is equal to the predicted probabilities $h_\Theta(x) \in \mathbb{R}^K$.

The activation function $g(\cdot)$ is usually a \emph{logistic sigmoid function}:
\begin{equation}
	g(z) = \frac{1}{1+ e^{-z}}
\end{equation}
The sigmoid function has range [0,1] and is differentiable with respect to $z$. Because of this characteristics it is used to represent probabilities in the logistic regression classifier. \emph{Logistic regression} for binary classification models the probability that $x \in \mathbb{R}^n$ belongs to the positive class as $g(w^Tx)$ and estimates the parameters $w \in \mathbb{R}^n$ during training. Any input whose output $g(w^Tx)$ is greater than $0.5$ is classified as positive, otherwise it is classified as negative. The sigmoid function equals $0.5$ when $w^Tx = 0$, thus, the decision boundary of a logistic regression classifier is $w^Tx = 0$, which is a linear function. 

However, the sigmoid function, per se, is not linear on its input $z$. Therefore, each unit in a neural network with sigmoid activation functions outputs a nonlinear activation $g(z)$ which in turn is received by units in the next layer, linearly recombined with the activation of other units and passed again through a sigmoid function; these operations are repeated until the input reaches the output layer. As a result, the function calculated by units in the output layer $h_\Theta(x)$ will be highly nonlinear on the original input $x$. This is the reason why neural networks can model functions which are highly nonlinear and why increasing the number of layers in a neural network increases the predictive power of the model. By the same token, it may be insightful to think of each unit in a neural network as a feature detector (via logistic regression): units in the first hidden layer are trained to activate when simple features are found on the input, units on the second hidden layer activate when a combination of these simple features is present on the input and so on. Thus, the network will learn to detect the most relevant features for the classification task and as the number of units increases, it learns ever more complex features (granted that there is enough training data).


The cost function of a neural network classifier is defined as:
\begin{equation}
	J(\Theta) = -\frac{1}{m} \left[\sum_{i=1}^m \sum_{k=1}^K y_k^{(i)}log(h_\Theta(x^{(i)}))_k + (1-y_k^{(i)})log(1-h_\Theta(x^{(i)}))_k)\right]
\end{equation}
where $m$ is the number of examples in the training set and $(x^{(i)},y^{(i)})$ is the $i^{th}$ example. $J(\Theta)$ is differentiable with respect to $\Theta$ but non-convex, nonetheless, gradient descent usually converges to a good estimate of the network weights~\cite{Ng2014}. \emph{Error backpropagation}~\cite{Linnainmaa1970, Werbos1974}, an algorithm where error terms are computed on the output layer and backpropagated layer by layer as the weights are adjusted, is commonly used for gradient descent training. Given the big number of parameters which need to be estimated, neural networks are susceptible to overfitting. The simplest approach to overcome this problenm is to use regularization. Regularization for neural networks is done by performing gradient descent on the regularized cost function presented in Equation~\ref{eq:ANNRegularizedCostFunction}
\begin{equation}
	\begin{split}
	J(\Theta) = &-\frac{1}{m} \left[\sum_{i=1}^m \sum_{k=1}^K y_k^{(i)}log(h_\Theta(x^{(i)}))_k + (1-y_k^{(i)})log(1-h_\Theta(x^{(i)}))_k)\right]\\ 
&+ \frac{\lambda}{2m}\sum_{l=1}^{L-1}\sum_{i=1}^{s^{(l)}}\sum_{j=1}^{s^{(l+1)}} \left(\Theta^{(l)}_{ij}\right)^2
	\end{split}
	\label{eq:ANNRegularizedCostFunction}
\end{equation}


	\subsection{Convolutional Networks}
	\label{subsec:ConvNets}
	\emph{Convolutional networks}, also \emph{ConvNets} or \emph{CNNs}, were first inspired by the way the human visual cortex proccesses information~\cite{Fukushima1980} but, as regular neural networks, they have evolved to favor practical performance over biological accuracy. LeCun et al. used a convolutional network to achieve good classification performance on the MNIST data set of handwritten digits~\cite{LeCun1989, LeCun1998}, the first successful application of modern convolutional networks. Recently, they have been used to achieve state-of-the-art performance on the ImageNet Large-Scale Visual Recognition Challenge~\cite{Krizhevsky2012}, an image classification and object localization challenge with 1000 categories~\cite{Russakovsky2014}. Since then, thanks to various advances (maxpooling , ReLU activations, weight initialization, GPU training, efficient backpropagation, etc.) they have become one of the most popular methods for image classification tasks and (along with recursive neural networks for generative models) an emblem for deep learning.

In this section we show the standard features and training of current convolutional networks, Section~\ref{subsec:PracticalDL} gives some practical advice for choosing hyperparameters and training deep architectures. For an in depth review of convolutional networks, see \cite{Karpathy2015}. For a complete overview of the history and state of deep learning, see \cite{Schmidhuber2015}.

Convolutional networks map raw image pixels to a score vector $h_\Theta(x) \in \mathbb{R}^K$ representing the distribution of (unnormalized log) probabilities over the $K$ classes. We could easily use a regular neural network (presented in Section~\ref{subsec:ANNs}) to do this classification but the amount of learnable parameters (the weights) becomes very big. For instance, a small color image of size $100\times100$ with 3 color channels (RGB) will require $30\,000$ units in the input layer and each unit in the second layer will therefore have $30\,000$ weights to learn. This is impractical not only because it will require a lot of data and time to train but because the loss function has very many local minima and thus it is harder to find a good local minimum.

Convolutional networks are specially designed to handle images reducing the number of connections between layers and the number of parameters to learn. Instead of fully connected layers such as regular neural networks convolutional layers are \emph{sparsely connected}, i.e., a unit is only connected to a small subset of the units in the previous layer. Furthermore, they are \emph{locally connected}, i.e., units are connected considering their position on the original image. The architecture of a convolutional network also imposes \emph{weight sharing} between units in the same layer, i.e., different units are forced to share the same weights (this determines filters and feature maps, defined below). \emph{Pooling} is a subsampling mechanism that reduces the spatial scale and makes the computations invariant to local translation. All these features reduce computation and improve the classification performance of convolutional networks; they are a product of the way convolutional networks are defined, which we explain below.

Each layer is composed of a set of \emph{feature maps}, 2-dimensional grids of unit activations~($\mathbb{R}^{h\times w}$), arranged into a 3-dimensional matrix ($\mathbb{R}^{h\times w \times d}$) where the third dimension is used to put together all feature maps. One could think of each layer as having all unit activations in a single column as in regular neural networks but seeing them as a 3-dimensional volume makes the definitions easier. The input layer could be considered as a 3-dimensional matrix ($\mathbb{R}^{h\times w \times c}$) holding the image of size $w\times h$ with $c$ color channels (usually one for grayscale images or three for RGB). The output layer could also be thought of as a volume of size $R^{1\times 1 \times K}$ where each feature map is just one activation ($R^{1\times 1}$) representing the final score. The convolutional network then receives an input image $x$, transforms it into the first layer of feature maps (which does not need to have the same dimensions as the previous layer) and keeps transforming it until we have an output layer of size $h(x) = R^{1\times 1 \times K}$. See Fig.~\ref{fig:ConvNetVolumes} for an illustration. We describe the possible transformations next.
\begin{figure}[h]
	\centering
	\includegraphics[width = 0.7\textwidth]{plots/convNetVolumes.jpeg}
	\caption[Convolutional network visualization]{A simple representation of the transformations of the input that a convolutional network computes. Input layer is shown in pink, hidden layers are shown in blue and output layer is shown in green. The third layer has 5 feature maps of size $2\times3$. Notice that the width is listed first by convention. Image courtesy of~\cite{Karpathy2015}.}
	\label{fig:ConvNetVolumes}
\end{figure}

There are four types of layers: convolutional layer, ReLU layer, pooling layer and fully connected layer all of which compute a differentiable function on its input and combine to form a convolutional network architecture.

\paragraph{Convolutional layer} Convolutional layers are the heart of convolutional networks. They are composed of a set of learnable filters which will be applied to the volume in the previous layer. A \emph{filter} is a matrix of weights 
which has a small spatial size (width and height) but goes across all feature maps of the volume (the third dimension). For instance, a $3\times 3$ filter to be applied in a volume with 10 feature maps will have 90 parameters ($\mathbb{R}^{3\times3\times10}$). See Figure~\ref{fig:ConvLayer} for an example. Each feature map in this layer is obtained by sliding a filter across the spatial dimensions (width and height) of the previous volume computing the dot product (a weighted sum) between the filter and the input producing a 2-dimensional array of values~\footnote{Each filter has also a bias term which is added to the product.}. Notice that all values in a single feature map are computed using the same filter. If we think of the feature map as a grid of units we can see that every unit is connected with only a small local subset of the units in the previous layer and that all units in the map share the same weights. 

At each convolutional layer, many feature maps are computed (each with its own filter) and stacked together to form the volume in the layer. We can think of each filter as looking for an specific feature of the input and each feature map collecting the probabilities of the feature being present in different positions of the original image.

We need to define various hyperparameters for this layer: the filter size, the stride (the number of places to shift the filter at each step), the amount of zero padding around the image and the number of feature maps. These define the shape of the resulting volume; the first three are usually defined in a way that it preserves the spatial size of the previous volume, the third dimension is solely dependent on the number of feature maps desired.
\begin{figure}[h]
	\centering
	\includegraphics[width = 0.4\textwidth]{plots/convLayer.jpeg}
	\caption[Example of a filter in a convolutional layer]{Example of a filter applied to a volume ($\mathbb{R}^{32\times 32\times 3}$) to obtain the values shown in the blue volume. The filter comprises all 3 feature maps of the input volume. We compute 5 feature maps as shown by the 5 units in the blue volume. For a complete convolution this filter will have to slide across the input volume. Notice all units in the same feature map share the same filter but units in different feature maps do not, even though they can be connected to the same local region of the input. Image courtesy of~\cite{Karpathy2015}.}
	\label{fig:ConvLayer}
\end{figure}

\paragraph{ReLU layer} This layer receives an input volume and performs an elementwise ReLU activation function to it, i.e, each value $z$ in the volume is passed through the nonlinearity $\max(0,z)$. It does not change the dimensions of the volume and has no learnable parameters, although the activation function itself could be considered as a hyperparameter. Usually a convolutional layer is always followed by a ReLU layer (or any other activation function), for this reason they are sometimes considered part of the convolutional layer, we leave them separate for clarity.

\paragraph{Pooling layer} The pooling layer subsamples the volume on the spatial dimensions reducing the size of the feature maps but keeping the number fixed. Standard max pooling slides a fixed size windows (normally $2\times2$) along each feature map with stride 2 (it is, without overlapping) and selects the maximum element on that space. This will reduce each dimension of the feature map by half, thus reducing the total number of activations by 75\%, e.g., a $4\times4$ feature map gets subsampled to size $2\times 2$ where each value is the maximum activation on each of the four quadrants of the original feature map. Notice that the subsampling is applied to each feature map separately contrary to the convolution. A popular variant of max pooling uses $3\times 3$ windows with stride 2, allowing for overlapping in the pooling.

\paragraph{Fully connected layer} One or more fully connected layers are used at the end of the network to compute the final score vector. Feature maps in this layer have size $1 \times 1$ resulting in a row volume or alternatively a row vector of values. Each feature map in this layer is fully connected to all units in the previous volume and outputs a dot product between the input and the connection weights which are the parameters to be learned during training. The output layer of a convolutional network is always a fully connected network with as many feature maps as classes. The interpretation of the scores of the output layer is similar to that of regular neural networks as the (unnormalized log) probability of $x$ belonging to class $k$. Lastly, notice that a fully connected layer can be simulated by a convolutional layer with the same number of feature maps and filter size $w\times h$ where $w$ and $h$ are the dimensions of the feature maps in the previous layer, i.e, filters that comprise the entire previous volume.

\bigskip
Convolutional layers (plus ReLUs) compute features on the input while pooling layers shrinken the volume before passing to the fully connected layers which act as a neural network classifier on the obtained features. The standard convolutional network architecture can be represented textually as:
\begin{verbatim}
       INPUT -> [[CONV -> RELU]*N -> POOL?]*M -> [FC -> RELU]*K -> FC
\end{verbatim}
where \texttt{*N} indicates that the layers are repeated \texttt{N} times, \texttt{?} indicates that the layer is optional and \texttt{N,M,K >= 0}. We can use this template to construct ever more flexible models from a linear classifier \texttt{INPUT -> FC} (\texttt{N,M,K = 0}) to a regular neural network \texttt{INPUT -> [FC -> RELU]+ -> FC} (\texttt{N,M = 0}, \texttt{K > 0}) to a convolutional network \texttt{INPUT -> [[CONV -> RELU]+ -> POOL?]+ -> [FC -> RELU]* -> FC} (\texttt{N,M > 0}, \texttt{K >= 0}). For instance, a typical deep convolutional network could be:
\begin{verbatim}
        INPUT -> [[CONV -> RELU]*2 -> POOL]*3 -> [FC -> RELU]*2 -> FC
\end{verbatim}
This network receives an input volume (the image) computes two sets of convolution plus ReLUs before pooling and repeats this pattern three times followed by fully connected layers plus ReLUs which are repeated twice and the output layer which reports the final classification scores. Although there is no standard way of counting the number of layers of a convolutional network usually the ReLU or pooling layers are not counted as they have no learnable parameters, therefore our example architecture has 10 layers (21 in total) which is a good depth for big data sets. Practical recommendations on building convolutional network architectures is offered in the Section~\ref{subsec:PracticalDL}.

Figure~\ref{fig:ConvNetExample} shows an example of a convolutional network with its different kind of layers. The image is taken from a simulation accesible at \url{cs231n.stanford.edu}.

\begin{figure}[h]
	\centering
	\includegraphics[width = 0.85\textwidth]{plots/convNetExample.jpeg}
	\caption[Example of a convolutional network in action]{Example of a convolutional network with architecture \texttt{INPUT -> [[CONV -> RELU]*2 -> POOL]*3 -> FC}. The input image has size $32\times 32$. Each hidden layer uses 10 feature maps (shown as columns). Notice that although the size of the feature maps looks constant in fact each pooling layer reduces each dimension by half (the feature maps of the final pooling layer have size $4\times 4$). The final scores are shown only for the 5 most probable classes. Image courtesy of~\cite{Karpathy2015}.}
	\label{fig:ConvNetExample}
\end{figure}

Recently there has been a push towards simpler convolutional network architectures. The All Convolutional Net~\cite{Springenberg2014} is a network formed solely by convolutional layers: pooling layers are replaced by convolutional layers with larger strides and fully connected layers are replaced as explained above. Notice that this greatly increases the number of parameters to be learn, therefore it may not be suitable for small data sets.

Converting the fully connected layers to convolutional layers has another advantage: we can use a convolutional network trained on small images to classify bigger images. By the way convolutional layers are defined when a bigger image is used as input the entire convolutional network will slide across the image and be applied to different portions of the image generating a score vector for each of them. Therefore, instead of having a single score for each class we will have an entire matrix of scores (for each position where the convolutional network was applied). We can then average over all scores per class to obtain a single score vector for the bigger image. Furthermore, we can control the stride of the convolution to choose how the convolutional network is slided across the big image.
For instance, if we train a convolutional network with images of size $32\times 32$ which via pooling get reduced to feature maps of size $4\times 4$ in turn passed to the (converted) fully connected layers to obtain a score vector, then when using a $96\times 96$ image as input to the same convolutional network it will get reduced to feature maps of size $12 \times 12$ and the fully connected layers will output a matrix of scores of size $9\times 9$ (for each class), i.e, it slides the $4\times 4$ fully connected layers across the $12\times 12$ feature maps. Averaging each score matrix we obtain the final scores for the big image. We could have also  set a stride of 4 in the first (converted) fully connected layer to get score matrices of size $3\times 3$ for each 9 non-overlapping $32\times 32$ partitions of the original image. It works exactly as if we were applying the convolutional network to the original image at a stride of 32 but does all computations in just one pass. This way we can reuse a pretrained network to classify images of bigger size. 

\emph{Transfer learning} is a related method where a convolutional network is trained on images from a specific domain and later used as a feature extractor for images on a different domain or as a initialized network which is fine tuned with examples of the new domain.

The loss function for a multiclass convolutional neural network is similar to that for a regular neural network (Equation~\ref{eq:ANNRegularizedLossFunction}) except that the vector score $h_\Theta(x)$ is now defined by the architecture of the convolutional network.
\begin{equation}
	J(\Theta) = -\frac{1}{m} \sum_{i=1}^m \log \left ( \frac{ e^{h_\Theta(x^{(i)})_{y^{(i)}}} }{ \sum_{j=1}^K e^{ h_\Theta (x^{(i)})_j} } \right ) + \frac{\lambda}{2m}\sum_{l=1}^{L-1}\sum_{i=1}^{s^{(l)}}\sum_{j=1}^{s^{(l+1)}} \left(\Theta^{(l)}_{ij}\right)^2
	\label{eq:ConvNetLossFunction}
\end{equation}
Furthermore, this loss function is still differentiable with respect to $\Theta$ and thus the entire network can be trained via gradient descent. Gradients of the loss function can be calculated using backpropagation.


	\subsection{Convolutional Networks applied to Breast Cancer}
	\label{subsec:BreastCancerConvNets}
	Yet to write. Redacted version of section Related Work.

Radiographs for medical imaging has been used for many years, lo was the first one in ..

1. initial mass Sahiran 1996
2. intial microcalc: Lo 1995- Lo 1998
3. multipath circular neural net Lo 1998-Kinnard 2002 (maybe not)
4. Optimization of architecture- Gurcan 2000: Gurcan 2002
5. CAD for microcalcification in FFDM. Ge2007(CAD for masses didn't use Convnets, used LDA and rule-based classifiers).
6. CAD things. (maybe not)


The lessions are only sign but there could be cancerous or not. 
We refer as detection to the task of classifying a lession as present or not in the image no matter its malignancy, for instance, classifying an image patch as either clustered microcalcifications or normal tissue, while we talk of diagnosis when classifying a lession as either benign or malign~\footnote{If the feature is not actually present on the image it is considered benign.}.
There is also benign microcalcifications, they a

Most are entirely convolutional, not pooling and small < 10K paramters and two to three layers.
\subsubsection{Related Work}
In this section we offer a summary of the most relevant work in using convolutional networks for breast cancer diagnosis. %Results are presented in a chronological order


%******************************** Wei1995 *******************************************
% Pretty much the same as Sahiner but smaller and simpler.
To the best of the author knowledge, the first attempt to use convolutional networks for breast cancer is reported in "Detection of masses on mammograms using a convolution neural network"~\cite{Wei1995}. This 4-page article was later expanded in~\cite{Sahiner1996}.

%******************************** Sahiner1996 **************************************
\begin{comment} Sahiner1996
- detection of tumors
- uses mass to mean benign or malign tumors (growths of cell with no purpose) and normal tissue is other masses (cysts, liquids, no mass at all, etc)
- 168 mammograms: 168 positive classes, 504 negative classes
- 0.87 AUC, 0.9 sensitivity, 0.69 specificity
- texture "contains useful information that can be used to effectively distinguish masses from normal tissue." Not sure 'bout this
- one output sigmoid
- GD+momentum, adaptive learning rates,  early stopping
- manually extracted ROI's
- "The average size (length of the long axis) of the masses, as estimated by the radiologists, was 12.2 mm., and the standard deviation of the mass size was 4.5
mm." i could use a 2-2.5 cm filter.
- digitized mammograms (not digital)
- For each mamogram, 4 ROIs extracted (a tumor, fatty tissue, dense tissue nad mixed dense/fatty tissue)
- each pixel 0.1 mm
- 256 pixels by 256 pixels initial image (2.56 cm by 2.56 cm)
- background reduction (averaging a 2 by 2 box with an average of 4 cardianl boxes). 
- (Because they didn't have the power) nonoverlapping average pooling (where the avg function is applied instead of the max function) with 16 x 16 filters (rsulting in 16 by 16 image patches) and 8 by 8 filters (resulting in 32 by 32 image patches)
- 8 rotations(0,90,180,270 and flipped). Used all to calculate a single output for training, so all 8 will contribute a single number. Output obtained as average among all of them.
 
- Experiment with single input image:
	single hidden layer network with variable feature maps and kernel size.
	Best results with kernel size 10 for 16 and 20 for 32.
	small grid search(needed to test more values).
	0.83 AUC
- GLDS texture features: contrast, angular second moment, enlropy and mean. Calculated at different subregions of the original 256 by 256 pixel image (it produces a 16 by 16 image).
- Experiment with GLDS plus pool-averaged image: 
	one hidden layer (3 feature maps, 10by 10 filter)
	16 by 16 raw input and one of the four possible GLDS
	good results already 0.86s 0.85
SGLD features: correlation, entropy, and difference entropy 
- Experiment with SGLD plus pool-averaged input:
	one hidden layer (3 feature maps, 10 by 10 filter)
	16 by 16 plus one
	NOt so good 0.84 AUC
- Good one: one of each plus index 
	one hidden layer
	varying kernel size and number of feature maps
	3 input images: pool-averaged, mean GLDS, SGLD correlation
- why not try imputing all possible GLDS+ all SLDS features+ raw  (8 feature maps) or deeper network? Probably because of no comp power
- give more info helps the AUC, maybe the improvement comes from the info lost by subsampling and the shallowness of the network, a deeper network with million parameters (and bigger input) will be able to learn the GLDS or SLDS features.
- texture images improve classification
- conv architecture not as important as texture images. (more image data/info)
- also points the need for bigger networks and the suboptimality of the hyperparamteer search (not all reasonable combination tried)
- no difference on 16 \times 16 vs 32 \times 32 (not sure about these because they were not one tested on the exact same architecture).
\end{comment}
They used a small convolutional network (2 layers, $\sim$1K parameters) for the detection of masses~\footnote{They call mass to what we refer to as tumor (either cancerous or non-cancerous) but not other kind of masses (cysts, fibroadenomas, fatty tissue, etc.). Thus, it actually detects tumors.}. Details of the best performing architecture can be found on Table~\ref{tab:BrCaConvNetArchitectures}. The data set consisted of 672 manually selected possible tumors from 168 digitized mammograms: out of which 168 were real tumors and 504 were not. Background reduction was performed on each image (using a rather convoluted method). The images (size $256 \times 256$ pixels equivalent to a 2.56 $cm^2$ area) were downsampled via non-overlapping average pooling (filter size $16 \times 16$) to size $16\times 16$; downsampling to $32 \times 32$ via an $8 \times 8$ average pooling was also performed and gave similar results. Furthermore, the data was augmented by using 4 rotations (0°, 90°, 180° and 270°) on each original image and on each horizontally flipped image (8 in total per each training image)~\footnote{The original article does not mention any data augmentation but it was probably performed given that they obtained the same results.}. The network was trained via batch gradient descent plus momentum and per parameter adaptive learning rate. Two sets of experiments were performed: in the first, the $16 \times 16$ image patches (and their 8 rotations) were used for training producing 0.83 AUC on the best architecture, later these image patches were complemented with 2 $16 \times 16$ ``texture-images'' calculated using image techniques on the initial mass image (three input feature maps in total) producing 0.87 AUC, 0.9 sensitivity and 0.69 specificity with the best network architecture. The authors showed that the network architecture was not as important for performance as providing the network with texture information. The texture features give back some of the information lost during the downsampling, which explains the improvement observed. The authors also acknowledge that the network architecture is far from optimal given its simplicity (one convolutional layer with three feature maps) and the incomplete hyperparameter tuning. A deeper network with more learnable parameters and a bigger input size could produce similar or better results without a need to include handcrafted texture features which will in theory be learned by the network (if needed).
%Learned lesson: masses from 0.7-1.7 cm^2, conv arch may not be as important, bigger networks are better




%*************************************** Lo1995 *************************************
\begin{comment} Lo1995
- detect microcalcifications
- only years after lecun showed it to be good on the mnist dataset.
- preselected images
- Background removal with wavelet high pass filtering ("a three-level wavelet transform was used and only the lowest frequency was eliminated for high-pass filtering before image reconstruction."). For lung nodules: Background removal like constrast enhancement.
- YES/NO output. For lung nodules: degrees of sensitivity in output(1-10) instead of disease/no disease . 
- Rotation and translation invariance. 0,90,180,270 and flipped over. (all of this on the small 32 by 32 images). No use of translation, it talks about it, though.
- Uses ROC/AUC.
- Each pixel represented 0.105 mm. (for instance 16 pixel input was 1.7mm)
- Same set used for validation and test
- using the data augmented versions one after the other in training gives better performance here (not sure why)
- 30-fold crossvalidation results reported (no test set): 0.89 AUC for individual miscrocalcifications and 0.97 for clustered microcalcif. 
- not quite clear if label were beningn/malign, microcalc/non-microcalc. It hink it is detection not diagonsis
- not clear how they measure the detection of microcalc. I think, of those microcalc detected from the normal algorithm if more than 3 were in the same 1 cm^2 area, it was considered as if the convnet detcted a cluster. 
- Easier to detect clusters these way because there could be 20 micorcalcif in a 1 cm^2 area and it only needs to detect 3.
- Bunch of questions on how on hell is this done. It could be done in a way that would help a lot the results, maybe that is why they have 0.97 AUC
\end{comment}
The first use of convolutional networks for the detection of microcalcifications is reported in~\cite{Lo1995}. They performed various experiments on a small convolutional network (3 layers, $\sim$5.4K parameters), details of the architecture are offered on Table~\ref{tab:BrCaConvNetArchitectures}. The input size ($16\time16$), number of hidden layers ($2$) and kernel size ($5\times5$) was obtained using a validation set, altough only few options were explored: they tried input sizes of 8, 16 or 32, one or two hidden layers and kernel sizes of 2, 3, 5 or 13.
A high sensitivity image technique was used to obtain a set of 2104 image patches ($16 \times 16$ pixels equivalent to an area of 1.7 $mm^2$) of potential microcalcifications from 68 digitized mammograms; of these, 265 were true microcalcifications and 1821 were ``false subtle microcalcifications". Prior to training, a wavelet high-pass filtering technique was used to remove the background. Each image was flipped horizontally and 4 rotations for each the original and flipped images were used for training (0°, 90°, 180° and 270°).
The network reached 0.89 AUC when identifying individual microcalcifications and 0.97 AUC for clustered microcalcifications; results obtained with a 30 fold cross validation. More than two individual microcalcifications detected on a 1 $cm^2$ area is considered a cluster detection, the predicted probability for the cluster is the average of the probabilities of all suspect patches inside the 1 $cm^2$ area~\footnote{This method is not clearly explained either in this article or~\cite{Lo1998} so this interpretation may not be correct. The way this evaluation is performed greatly affects the validity of the reported results.} Other performance metrics were not explicitly reported.
% Do I only consider the patches who were suspect in the first place and are inside the 1 cm2?. waht if a single suspect patch has more than 2 microcalcifications, isthat considered a cluster detection?, how do I make the cluster grouping. how is the labelling in the original images done, are the 1 cm^2 preset, when is a cluster not detected. Are only the dected clusters used to calculate the rsulting NDDI?
% Is this left intentionally vague?
This article showed that deeper networks, background removal and data augmentation improved results. Together with the previous article, it also proved that simple convolutional networks can be used for breast cancer lession detection. 
% Lesson learned: two hidden layer newtwork produces better results, background reduction is neccesary and using matrices invariance to augment the data helps.
% 1 cm2 for clustered microcalcifications

% ********************************* Lo1995 *****************************************
% Shih-Chung B. Lo ; Huai Li ; Jyh-Shyan Lin ; Akira Hasegawa ; Chris Y. Wu, et al. "Artificial convolution neural network with wavelet kernels for disease pattern recognition", Proc. SPIE 2434, Medical Imaging 1995: Image Processing, 579 (May 12, 1995); 
% Detection of microcalcifications
% "Wavelet based kernels". Same results.

%********************************** Chan1995 ****************************************
% Chan, H.-P., Lo, S.-C.B., Sahiner, B., Kwok Leung Lam, Helvie, M.A. Computer-aided detection of mammographic microcalcifications: Pattern recognition with an artificial neural network (1995) Medical Physics, 22 (10), pp. 1555-1567.
% Initial set divided in obvious, average and subtle microcalcifications
% Trained with hard cases and proved different architectures resulting in AUC 0.9

%********************************* Lo1996 *******************************************
%Lo, S.-C.B., Li, H., Lin, J.-S., Hasegawa, A., Tsujii, O., Freedman, M.T., Mun, S.K. Detection of clustered microcalcifications using fuzzy modeling and convolution neural network (1996) Proceedings of SPIE - The International Society for Optical Engineering, 2710, pp. 8-15.
% A fuzzy classification modeling was employed to extract each suspected microcalcification
% Fuzzy function also used to determine the of spots near a cluster as part of the cluster (it received as input the distance to the cluster an the output of the convolutional network)
% Sensitivity 90% at 0.5 FP per image 

%************************************* Lo1998 ***************************************
\begin{comment} Lo 1998
- similar to Lo 1995:
	detect microcalcifications
	pre-selected image patches
	8 rotations per image(0,90,180,270 and flipped)
	sigmoid activation function
	16 by 16 pixel size
	5 by 5 filters
	2 outputs
	clustering method
- only 10 groups per layer
- "Typically, the sizes of microcalcifications vary from 0.16 mm to 1.0 mm."
- each pixel 0.1mm, more than that may make dissapear the microcalc.
- DYSTAL network, regular neural network, and convolutional network. convnet ouptperforms them.
- rotation and translation invariance.
- gaussian-like activation function in input (?)
- 38 "digital" mammograms: 220 true and 1132 subtle microcalcififcations
- divided into two roughly equal sets for test (no cross validation)
- 0.9 AUC for microcalc and 0.97 AUC for clustered microcalcif
\end{comment}
A convolutional network with a similar architecture (3 layers, $\sim$4.5K parameters) was presented by the same group in~\cite{Lo1998}. It detects microcalcifications from $16 \times 16$ image patches which were pre-selected and preprocessed using the same image techniques as above. For these experiments, nonetheless, they used 38 digitized mammograms and extracted 220 true microcalcifications and 1132 false ones which were randomly divided into a training and test set of roughly equal sizes. The network obtained a 0.9 AUC for individual microcalcifications and 0.97 AUC for clustered microcalcifications (also evaluated as in~\cite{Lo1995}). It showed that a convolutional network outperforms a regular neural network and a DYSTAL network in the microcalcification detection task when using raw pixels as input features.
% Lesson learned: better data is good, microcalcifications are 0.2-1 mm




%************************************ Lo1998 ****************************************
% Lo, S.-C.B., Li, H., Hasegawa, A., Wang, Y.J., Freedman, M.T., Mun, S.K. Detection of mammographic masses using sector features with a multiple circular path neural network (1998) Proceedings of SPIE - The International Society for Optical Engineering, 3338, pp. 1205-1214.
% Born of Multipatch circular neural network, explained better in Lo2002

%*********************************** Lo2002 *****************************************
\begin{comment} Lo2002
. 
\end{comment}
\cite{Lo2002} offered an alternative slighlty modified convolutional network specifically designed for breast masses called Multiple circular path convolution network.
% Maybe not
% detection

%******************************** Kinnard2002 **************************************
% Kinnard, L., Lo, S.-C.B., Wang, P., Freedman, M., Chouikha, M. Separation of malignant and benign masses using maximum-likelihood modeling and neural networks (2002) Proceedings of SPIE - The International Society for Optical Engineering, 4684 II, pp. 733-741.
% MCPCNN pus max'likelihood. Maybe not. malignancy





%********************************* Gurcan2000 *************************************
% Gurcan, M.N., Sahiner, B., Chan, H.-P., Hadjiiski, L., Petrick, N. Optimal selection of neural network architecture for CAD using simulated annealing (2000) Annual International Conference of the IEEE Engineering in Medicine and Biology - Proceedings, 4, pp. 3052-3055. 
% Optimal network architecture with simulated annealing (3 pages). Simple

%********************************* Gurcan2001 *************************************
% Gurcan, M.N., Sahiner, B., Chan, H.-P., Hadjiiski, L., Petrick, N. Selection of an optimal neural network architecture for computer-aided detection of microcalcifications - Comparison of automated optimization techniques (2001) Medical Physics, 28 (9), pp. 1937-1948.
% Comparaison of automatic hyperparamter search methods.
% Hyperparameters are 4: number of feature maps in the two layers, kernel sizes in two layers.
% Steepest Descent SD, Simulated Annealing SA and Genetic Algorithms GA
% It compares efficiency of algorithms but not architectures. 
% Simulated annealing beat all.

%********************************* Gurcan2002a **************************************
% Gurcan, M.N., Chan, H.-P., Sahiner, B., Hadjiiskii, L.M., Petrick, N., Helvie, M.A. Optimal neural network architecture selection: Effects on computer-aided detection of mammographic microcalcifications (2002) Proceedings of SPIE - The International Society for Optical Engineering, 4684 III, pp. 1325-1330. 
% Pretty much the same as Gurcan2002b.

%******************************* Gurcan2002b ****************************************
\begin{comment}
% Gurcan, M.N., Chan, H.-P., Sahiner, B., Hadjiiski, L., Petrick, N., Helvie, M.A. Optimal neural network architecture selection: Improvement in computerized detection of microcalcifications (2002) Academic Radiology, 9 (4), pp. 420-429.
% Hyperparamter search: All are in digitized images(?)
% how much does it help the performance.
% "The optima! architecture (N1-N2-K1-K2) was determined to be 14-4-5-5 when the architecture was trained with group ! and tested with group 2 and 14-10-5-7 when the training and thetest sets were switched."
% AUc (?)
% accuracy can be improved by sleecting a good architecture.
% 
\end{comment}
optimal archite tures is ...




%*********************************** Ge2005 *****************************************
%Ge, J., Wei, J., Hadjiiski, L.M., Sahiner, B., Chan, H.-P., Helvie, M.A., Zhou, C. Computer-aided detection of microcalcification clusters on full-field digital mammograms: Multiscale pyramid enhancement and false positive reduction using an artificial neural network (2005) Progress in Biomedical Optics and Imaging - Proceedings of SPIE, 5747 (II), art. no. 83, pp. 806-812.
% Detect microcalcifications in digital (!) mammograms
% "we investigated the performance of a nonlinear multiscale Laplacian pyramid enhancement method in comparison with a box-rim filter at the image enhancement stage"
% 0.97 AUC

%********************************* Ge2006 *****************************************
% Ge, J., Sahiner, B., Hadjiiski, L.M., Chan, H.-P., Wei, J., Helvie, M.A., Zhou, C. Computer aided detection of clusters of microcalcifications on full field digital mammograms (2006) Medical Physics, 33 (8), pp. 2975-2988.
% Focuses on final CAD for cluster mammogram for FFDM.
% Has a lot of stages:
% don't have it!.
We will describe each stage lgihtly and focus on the CNN

%********************************* Ge2007 ******************************************
% Ge, J., Hadjiiski, L.M., Sahiner, B., Wei, J., Helvie, M.A., Zhou, C., Chan, H.-P. Computer-aided detection system for clustered microcalcifications: Comparison of performance on full-field digital mammograms and digitized screen-film mammograms (2007) Physics in Medicine and Biology, 52 (4), art. no. 008, pp. 981-1000.
% Does it use CNN?
% better for FFDM than SDM. 
% 0.96 AUC





%************************* Petrick2013 *****************************************
% Petrick, N., Sahiner, B., Armato III, S.G., Bert, A., Correale, L., Delsanto, S., Freedman, M.T., Fryd, D., Gur, D., Hadjiiski, L., Huo, Z., Jiang, Y., Morra, L., Paquerault, S., Raykar, V., Samuelson, F., Summers, R.M., Tourassi, G., Yoshida, H., Zheng, B., Zhou, C., Chan, H.-P. Evaluation of computer-aided detection and diagnosis systems (2013) Medical Physics, 40 (8), art. no. 087001, .
% on how should CAd systems be evaluated. Maybe important.



\begin{comment} 

Two teams (search for more papers from them)
Georgetown University Medical Center: Shih-Chung Lo, Matthew Freedman, Huai Li
Michigan Medical Center: Heang-Ping Chan, Sahiner, Hadjiiski, Helvie, Gurcan, Wei j and Ge, J
University of Pittsburgh
University of chicago, too:

Deep learning
Cruz-Roa, mitosis detection in breast cancer histology images, tomosynthesis 

%%%%%%%%%%%%% Breast cancer CAD.
2013 breast cancer diagnosis a review or other good review.
Work at Tec.
\end{comment}


	\subsection{Practical Deep Learning}
	\label{subsec:PracticalDL}
	In this section we collect guidelines for building as well as efficiently training deep convolutional networks. While they are specific to this thesis, they may prove useful in similar projects. Lastly, deep learning is a fast-changing field so these recommendations may soon be outdated.
% Some of these details are taken care by the software(either as a defualt or optional feature) and some are qyuite new and need to be taken care manually. 

\paragraph{Image preprocessing} Convolutional networks can handle raw image data, but some level of preprocessing speeds training and improves performance.
\begin{itemize}
	\item Crop images to contain only the relevant regions, denoise, enhance and resize them to maintain the input size fixed and manageable.

	\item Zero-center each image feature (the raw pixels) by substracting its mean across all training images. Optionally, normalize each zero-centered feature to range from $[-1 \dots 1]$ by dividing them by its standard deviation~\cite{Karpathy2015}.

	\item Test data should not be used to calculate any statistic used for preprocessing. Furthermore, the same statistics (calculated from training data) should be used to preprocess test data~\cite{Karpathy2015}.
\end{itemize}



\paragraph{Convolutional network architecture} We provide recommendations for designing convolutional networks and sensible values for related hyperparameters.

\begin{itemize}
	\item Select a network architecture flexible enough to model the data and manage overfitting rather than a simpler architecture that may be incapable to model the data~\cite{Ng2014, Krizhevsky2012}. 

	\item Although, theoretically, neural networks with a single hidden layer are universal approximators provided they have enough units ($\mathcal{O}(2^n)$ where $n$ is the size of the input), practically, deeper architectures produce better results using less units overall. This holds for convolutional networks~\cite{Bengio2014}.

	\item Use at least 8 layers (not counting pooling or ReLU layers) for big data sets, use less layers or transfer learning for small data sets. ``You should use as big of a neural network as your computational budget allows, and use other regularization techniques to control overfitting.''~\cite{Karpathy2015}

	\item Use 2-3 \texttt{CONV -> RELU} pairs before pooling (N above)~\cite{Karpathy2015}. Pooling is a destructive operation, placing two convolutional layers together allows them to detect more complex features.

	\item Use 1-5 \texttt{[CONV -> RELU]+ -> POOL} blocks (M above). This hyperparameter regulates the representational power of the architecture. The exact number depends on the complexity of the features in the data and the computational resources available. It also defines how much the volume is subsampled.

	\item Use less than 3 \texttt{FC -> RELU} pairs before the output layer (K above)~\cite{Karpathy2015}. The volume that arrives to fully connected layers is already complex, adding many layers only increases the number of parameters and risks overfitting.

	\item The number of feature maps per convolutional layer controls the number of features detected at that layer---similar to the number of units per layer in a regular neural network. A common pattern is to start with a small amount of feature maps and increase them layer by layer~\cite{Simonyan2014}. %The reasoning is that at higher layers there are more complex features to learn and moreover as the feature maps become smaller (via pooling) it is computationally feasible to have more of them.

	\item The number of feature maps per fully connected layer decreases layer by layer\footnote{The number of units in a convolutional layer is the number of units in a feature map times the number of feature maps.}. For instance, for a convolutional network with ten possible classes and two fully connected layers, if the last convolutional layer produces a volume of size $8 \times 8 \times 512$ (8192 units), the first fully connected layer could have size $1 \times 1 \times 2048$ and the second--- the output---layer $1\times 1\times 10$.

	\item Use $3\times 3$ filters with stride 1 and zero-padding 1 or $5 \times 5$ filters with stride 1 and zero-padding 2. This preserves the spatial dimensions of the volume and works better in practice~\cite{Springenberg2014}. If the input size is too big, use a bigger filter in the first convolutional layer~\cite{Karpathy2015}.
	
	\item Use $2\times2$ pooling with stride 2. This pooling and the overlapping version presented in Section~\ref{sec:ConvNets} produce similar results~\cite{Krizhevsky2012}.
%krizhevsky says overlapping is slightly better. so does dieleman but it slows thing down.
	\item Use square input images (width = height) with spatial dimensions divisible by 2. These should be divisible by 2 at least as many times as the number of pooling layers in the network.

	\item Use the number of parameters to measure the complexity of an architecture rather than the number of layers or units.
\end{itemize}



\paragraph{Hyperparameters} About setting and searching hyperparameters other than those of the network architecture.

\begin{itemize}
	\item Use a single sufficiently large validation set (15-30\% of data) rather than cross validation~\cite{Bengio2014}. Use cross validation in very small data sets~\cite{Ng2014}.

	\item Use random search rather than grid search. Random search draws each parameter from a value distribution rather than from a set of predefined values.~\cite{Bergstra2012}

	\item Search for the best combination of hyperparameters rather than each individually.

	\item Train each combination of hyperparameters for 1-2 epochs to narrow the search space; then, train for more epochs on these ranges~\cite{Karpathy2015}. Explore further when the best value for a hyperparameter is found in the limit of the range.~\cite{Bengio2012}.

	\item Partial convergence is sufficient to assess hyperparameters~\cite{Karpathy2015}.	

	%\item Use a different validation set if you need to run the hyperparameter search for new parameters. :: Because the old validation set is already good for the hyperparameters chosen, we want to choose good hyperparameters in general not only in that validation set. 

	\item Hyperparameters related to the convolutional architecture, e.g., number of layers, number of feature maps and filter sizes are set manually (as explained above) rather than using a validation set.

	\item Several hyperparameters are set: initial learning rate $\alpha$, learning rate decay schedule, regularization strength $\lambda$, momentum $\mu$, probability of keeping a unit active in dropout $p$, mini-batch size and type of image preprocessing.

	\item We could fit all hyperparameters using a validation set but, in practice, this is computationally unfeasible and results in overfitting to the validation data~\cite{Cawley2010}.

	\item Set $\alpha$, $\lambda$ and optionally the type of preprocessing using a validation set. Other hyperparameters can be set to a sensible default. 
%The learning rate schedule and training epochs are set using heuristics. 

	\item The learning rate $\alpha$ is ``the single most important hyperparameter and one should always make sure that it has been tuned''~\cite{Bengio2012}. It ranges from $10^{-6}$ to $10^{0}$. Use a log scale to draw new values ($\alpha = 10^{unif(-6, 0)}$ where $unif(a,b)$ is the continous uniform distribution)~\cite{Karpathy2015}.
%0.01: Bengio, 2012 says that the optimal learning rate is close to the highest learning rate that does not cause divergence.

	\item The regularization strength $\lambda$ is usually data (and loss function) dependant. It ranges from $10^{-3}$ to $10^4$. Search in log scale ($\lambda = 10^{unif(-3, 4)}$).
%1
	\item Halve the learning rate every time the validation error stops improving or choose a fixed number of epochs by observing when the validation error stops decreasing in a similar network~\cite{Krizhevsky2012}.

	\item Use $\mu=0.9$. If using a validation set try values in \{0.5, 0.9, 0.95, 0.99\}~\cite{Karpathy2015}.

	\item Use 0.9-1 probability $p$ of retaining a unit in the input layer, 0.65-0.85 in the first 2-4 convolutional layers and 0.5 in the last convolutional layers and all fully connected layers~\cite{Srivastava2014}. Less dropout is used on the first layers because they have less parameters~\cite{Karpathy2015}.
% Maybe don't use dropout in the input layer, because putting a zero there has a meaning(black), maybe the advantage of dropout in cinvolutional layers is just that it adds noise to the input.

	\item Use mini-batch size of 64 or 32. A larger batch size requires more memory and training time. Test performance is unaffected~\cite{Bengio2012}.

	\item Choose among standard preprocessing techniques by (qualitatively) inspecting results on images from the validation set. If none seems superior, fit it along $\alpha$ and $\lambda$% other hyperparameters.
\end{itemize}



\paragraph{Training}
Some general tips for training convolutional networks with millions of parameters and big data sets. Using this advice for small networks may be excessive but it will not hurt the performance.

\begin{itemize}
	\item Randomize the order of the trainig examples before training. As we are using an stochastic estimator of the gradient this ensures the examples in each batch are sampled independently. Shuffling the examples after each epoch could also speed convergence~\cite{Bengio2012}.

	\item To estimate the number of examples needed to train a convolutional network divide the total number of learnable parameters by 25-100 (assuming some data augmentation). Some groups have been able to learn up to 40M parameters from as little as 60K training examples~\cite{Dieleman2015, Springenberg2014}.

	\item Weight initialization is very important for a proper convergence of the network. The current recommendation for ReLU units is to initialize each weight as a value drawn from a gaussian distribution $\mathcal{N}(\mu = 0, \sigma = \sqrt{2/n_{in}})$ where $n_{in}$ is the fan-in of the unit, i.e., the number of inputs to the unit. Specifically, each filter weight could be initialized as \texttt{w = randn()*sqrt(2/nIn)} where \texttt{randn()} returns a value drawn from a standard normal distribution and \texttt{nIn} is the number of connections to this filter (9 for a $3\times 3$ filter, for example). Weights for units in the fully connected layer follow the same formula. Biases can be initialized likewise or to zero~\cite{He2015}.

	\item Use mini-batches to compute the gradient. Using the entire training set to compute the gradient of the loss function takes a big amount of computation and points to the steepest descent direction locally but may not be the right direction if the update step is large. Using mini-batches allows us to make more updates, more frequently which results in faster convergence and better test results~\cite{Bengio2012}.

	\item Use Nesterov's Accelerated Gradient (NAG) to update the weights. It is a modified version of gradient descent which has shown to work slightly better for certain architectures~\cite{Bengio2012b}. Stochastic Gradient Descent with Momentum (SGD+Momentum) is also a viable option~\cite{Karpathy2015}.
% May need to crossvalidate the momentum if using NAG 

	\item Use dropout as a complement to $l_2$-norm regularization. Dropout usually improves results but it may slow network convergence~\cite{Krizhevsky2012}.

	\item Store the network parameters regularly during training. Once per epoch should be enough but it depends on the number of parameters and size of the data. This allows you to come back to different versions of the network and select the one with the best overall validation/test error or one with some special characteristics~\cite{Bengio2014}.

	\item Stop the training process when the validation or test error has not improved since the last learning rate reduction. At this point gradient descent may not have converged but the validation error has and will start to increase (overfit)~\cite{Bengio2012}.

	\item Use the validation or test error to select the best parameters for the network from those stored~\cite{Bengio2014}. 

	\item If you use the test set to refine a model, shuffle the entire data set and choose a diferent training and test set for the new model. Otherwise, you run the risk of overfitting to the test set~\cite{Ng2014}.
\end{itemize}



\paragraph{Sanity checks}
Some simple checks to make sure the training is working properly.
\begin{itemize}
	\item After weight initialization, the network should predict similar scores for each class (uniform probability) and have a loss function (without regularization) equal to $-\log(1/K)$. You can check this by running a test on a small set of examples. Adding regularization should increase the loss~\cite{Karpathy2015}.

	\item If you implement back propagation manually or believe it may not be working properly you can run a gradient check. Gradient checks compare the analytic gradient produced by backpropagation with a numerical gradient produced by a finite difference approximation~\cite{Karpathy2015}.

	\item Train the network with a very small subset of data (20 examples, for instance) and make sure it produces zero loss (without regularization). If it cannot overfit a tiny subset of examples the model is too simple~\cite{Ng2014}.

	\item During training, the training loss should always decrease or only slightly increase. Otherwise, gradient descent may not be working properly either because of an implementation error or poorly tuned hyperparamters (high learning rate, low momentum)~\cite{Karpathy2015}.

	\item Monitor the training and validation loss during training to identify overfitting and underfitting. Underfitting is characterized for a high training loss, overfitting is characterized for a big gap between training and validation (or test) loss~\cite{Ng2014}.
\end{itemize}

\paragraph{Data augmentation}
One of the easiest ways to reduce overfitting in image data is to generate additional examples from the original data by applying some simple label-preserving transformations. Data augmentation allows the network to see different views of the same object thus enabling it to identify features that do not depend on the invariance introduced by the transformations. 
For instance, if we present it with images of a book on different rotations, we expect it to learn to identify a book no matter its position.

\begin{itemize}
	\item There are many transformations one can apply: rotations, translations, horizontal and vertical reflections, crops (sample patches of the original image), zooms, etc. For color images, adding some noise to (jittering) the colors is also a valid transformation.
% Dieleman galaxies didn't like jittering or scales and crops
% Lo liked rotation and horizontal flipping

	\item Exploit the invariances you expect in the data set. For instance, galaxies are rotation invariant given that in space there is no up or down~\cite{Dieleman2015} but trees are not as we rarely see an upside down tree.

	\item When combining different transformations in the same image be careful to preserve the original label. An overly modified image may lose its meaning. 

	\item Most transformations are affine in the geometric plane and can be combined into a single one. If you plan to apply various transformations to the same image, applying a single affine transformation is faster and reduces information loss~\cite{Dieleman2015}.

	\item Generate the augmented images during training. This saves storage and can be performed alongside the training~\cite{Krizhevsky2012}.

	\item Data augmentation can also be used at test time by presenting the network with various versions of the same image and averaging its predictions~\cite{Krizhevsky2012}.
\end{itemize}

\begin{comment}
\paragraph{Image segmentation}
image segmentation
We account for some details of training a convolutional network for image segmentation. 

	\item We usually require to post-process, upsample and threshold the output of a convolutional network. We can fit them using the validation set for no cost given that, at this point, the network is already trained.

\item image postprocessing, using a simple threshold is bad because it doesnt account for multiple comparisons, 

\item CRf work remarkadly well (cite the paper with CRFs) and say that these preprocessing  

%Or I could add this back in hyperparameters
%	\item Use the validation set to select among standard post-processing techniques; this is cheap/efficient because the network does not need to be retrained. It may also be done by visual analysis. some techniques have eeven more paramteres.

	\item Keeo the image sizes manageable with respect to the network because the memory requirements could be excessive. If images are big use small mini-batch sizes.
	\item Use bilinear interpolation. this is usually not important
	\item Set the threshold
	
\end{comment}

% Need some sources for this section. I've read it all around.
\paragraph{Unbalanced data}
Having very few examples of one class compared to the rest is common in practice. We offer here some advice to deal with unbalanced classes using standard convolutional networks. We note that there is no accepted way to manage this problem.
\begin{itemize}
	\item For a binary classifier, if the positive class is the rare class use PRAUC as a performance metric. If you are reporting the $F_1$ score, you should select an appropiate threshold using a validation set~\cite{Davis2006}.

	\item If using PRAUC or selecting the threshold with a validation set is impractical, a simple adjustment is to divide the predicted probabilities by their corresponding class priors and to renormalize the values.

	\item For multiclass classification, use the macro-averaged $F_1$ score, an average of $F_1$ scores per class, with validated thresholds~\cite{Ozgur2005}. A multiclass PRAUC also exists but it is not as easy to interpret.

	\item As the threshold setting does not affect training it can be fitted independently of other hyperparameters once the network has already been trained. If fitting more than one, consider each one as a separate hyperparameter and use random search to find the best combination.

	\item One of the preferred methods to learn with unbalanced data sets is to use a modified loss function which gives a higher weight to errors in the rare class so that during training errors in the rare class will produce higher learning in the network parameters. Specific knowledge of the domain is required to estimate the cost of each class of error.

	\item Oversampling and undersampling, repeating the examples of the rare class or discarding some examples from the dominant class, are discouraged because they either not add information or throw away some of it.

	\item Replicating rare examples (oversampling) is useful when the examples are very scarce and the classifier simply does not have enough data to learn. This could be achieved by balancing the classes on each mini-batch via stratified sampling or by augmenting the rare class more than the dominant class during data augmentation.

	\item Data augmentation differs from data replication in that it only tries to enrich the data set with invariant images but actually leaves the proportion of classes unchanged.
\end{itemize}

\begin{comment}
\paragraph{All convolutional networks}
The research community has been moving towards discarding the pooling layers and using all convolutional networks. This can be interpreted as letting the network learn the pooling operation. We offer a couple of guidelines for implementing all convolutional networks.  
\begin{itemize}
	\item Replace each pooling layer with a convolutional layer with as many feature maps as the previous layer and filter size $2\times 2$ with stride $2$ for normal pooling or $3\times 3$ with stride $2$ for overlapping pooling~\cite{Springenberg2014}. 

	\item For small data sets pooling layers also work as a regularizer because they reduce the number of learnable parameters and replacing them with convolutional layers may not be convenient~\cite{Karpathy2015}.
\end{itemize}

\paragraph{Transfer learning}
When we have a small data set we could use a pretrained convolutional network either as a feature extractor for the new examples or to provide initializations for the new convolutional network, this is called transfer learning. We offer some tips for using a pretrained model specifically for mammographic images.

\begin{itemize}
	\item Using a convolutional network pretrained in natural images, such as the ImageNet database, CIFAR-10, CIFAR-100, etc., may not work for mammographic images because features useful for one kind of classification are not very useful for the other. Nonetheless, given that features become more specific at higher layers, we could discard the higher layers of the network and use only the cropped network~\cite{Karpathy2015}.
 
	\item Depending on the amount of data that we have we could: (1) add some fully connected layers on top of the pretrained network and train only these new layers, (2) add some convoutional and fully connected layers and train these new layers or (3) add convolutional and/or fully connected layers and train the entire network~\cite{Karpathy2015}.

	\item When training on a pretrained model or fine-tuning use an smaller learning rate than when training a network from scratch. Using a small learning rate assures that we do not disturb very much the already good network parameters.~\cite{Karpathy2015}.
\end{itemize}
\end{comment}

\paragraph{Software}A short description of four of the most popular packages for deep learning. They are pretty similar in capabilities and availability (open-source).

\begin{itemize}
	\item Tensorflow~\cite{Abadi2015} is a Python/C++ library released by Google that supports automatic differention on data flow graphs---neural networks, included, training on clusters of CPUs and GPUs, and easy deployment in multiple platforms. It has been quickly adopted by the deep learning community.
	\item Caffe~\cite{Jia2014}: Caffe is an already mature deep learning framework developed in C++/CUDA by the Berkeley Vision and Learning Center (BVLC) and community contributors. It offers a command line, Python and Matlab interface, reference models and tutorials and very fast code with easy GPU activation.
	\item Theano~\cite{Bergstra2010, Bastien2012}: Theano is a Python library developed in Python/CUDA at the University of Montreal. It is tightly integrated with NumPy, performs symbolic automatic differentiation and uses the GPU to efficiently evaluate mathematical expressions involving multi dimensional arrays.
	\item Torch7~\cite{Collobert2011}: Torch is a scientific computing framework developed in C/Lua/CUDA at the IDIAP Research Institute. It offers n-dimensional arrays (tensors), automatic differentiation, a command line and Lua interface, GPU support and easy building of complex neural network architectures.
%	\item Cuda-ConvNet2~\cite{Krizhevsky2014}: Cuda-ConvNet2 is a highly optimized convolutional network library developed in C++/CUDA by Alex Krizhevsky. It offers different off-the-shelf configurations for convolutional networks, a command line interface and multi-GPU training.
\end{itemize}

\bigskip

We acknowledge that mammographic data is different from common image segmentation data: labelling is imperfect, image sizes and ratio change, images are bigger, quality varies, objects of interest are small in relation to the background, data sets are smaller, texture is uniform across the image, among others. Therefore, some of the advice given above may prove counterproductive. If possible, design decisions should be based on data and results.
%Deep Learning (Bengio Lecture): (https://www.youtube.com/watch?v=JuimBuvEWBg)


\section{Methodology}
\label{sec:Methodology}
In order to achieve the proposed objectives and test our hypotheses we will need to carry out various tasks. We list them here in the order in which we plan to execute them:

\begin{itemize}
	\item Literature review: A thorough review of the published work using the databases and resources available in the institution. By the end of this task, a complete theoretical background should be obtained and written. This will also help refine the scope of the project and the experiments to be conducted.
	\item Software review: Once a clear idea of what are the possible experiments to be executed, we will need to find appropiate software to perform them. Software for database managing, preprocessing and implementation of different neural networks should be either located or developed.
	\item Database preprocessing: We will ready the database images for the experiments; these implies joining different databases, obtaining the required features, preprocessing the images, assigning labels, etc.
	\item Assesing image preprocessing: We will train a standard convolutional network with fixed parameters on mammograms with three different preprocessings: no preprocessing, image enhancement using median or gaussian filters and wavelet filtered images. Furthermore, we will train a deeper convolutional network on nonpreprocessed images. We want to answer three research questions: which is the best preprocessing for convolutional networks, is using the best filter significantly better than using nonpreprocessed images and can a convolutional network automatically preprocess the images?
% Q: Is it better to make different preprocessings oin the same convolutional network or to fit each convolutional network for each preprocessing, thus, giving it the best chance to perform but taking more time.
	\item Exploratory experiments: We will train standard convolutional networks in two different inputs: small image patches obtained from mammograms and whole mammogram images. We will also train a linear classifier, probably rectified linear units, on the features obtained from a convolutional network trained on the ImageNet database, i.e., we will use an already trained convolutional network instead of one trained specifically in mammograms. Here we will use the image preprocessing technique that showed better results in the previous step. We want to answer two research questions: Can a convolutional network trained on whole mammograms perform as well as one trained on small patches and can we use an already trained convolutional network to classify mammograms?
	\item Model selection: Using the insights from previous sections and the current literature on convolutional networks, we will select a network architecture along with novel features, preprocessing, training and regularization procedures. We aspire to find the best convolutional network configuration for mammogram classification.
	\item Further experiments: We will train the chosen convolutional network on our mammographic database. We will perform crossvalidation to adjust the most important network parameters and use regularization to avoid possible overfitting. We want to answer two research questions: is the performance of the convolutional network considerably improved by parameter tuning and, more importantly, is this a good performance?.
%Maybe train one with no tune fitting.
	\item Gathering results: Produce results on the test set and elaborate figures and tables. This could be obtained directly from software output or from further program executions.
	\item Reporting results: Write the thesis and any article or technical guide which may result from this work. Both this and the previous step will be performed along the execution of the project, hopefully benefiting from the supervisors' feedback.
\end{itemize}
Finally, we want to note that this is an idealized workflow and some changes may occur due to time limitations or resources unavailability. In the unlikely case that the work is finished before the project deadline, we will either reiterate on model selection, experiments, result gathering and reporting or look into digital tomosynthesis, network ensembles or evolving convolutional networks.

\begin{comment}
La {\it Metodología} (o lo que algunos autores llaman el {\it Método})
 es el proceso o
conjunto de pasos que debe efectuarse para llegar a cumplir con los
objetivos. Esos pasos deben contener  los experimentos a realizar, la forma de
llevarlos a cabo, la evaluación de los resultados, la prueba de las hipótesis,
la respuesta a las preguntas de investigación y el último paso debe ser el
reporte escrito de los resultados.
\end{comment}


\section{Work Plan}
\label{sec:WorkPlan}
We present here the expected work plan for this master's thesis. A description of the activities can be found in Section~\ref{sec:Methodology}
\begin{figure}[h]
	\centering
	\includegraphics[width = \textwidth]{plots/workplan.png}
	\caption[Thesis Work Plan]{Thesis work plan.}
	\label{fig:workplan}
\end{figure}

\begin{comment}
Una vez establecida la {\it Metodología} es importante establecer las
actividades con sus tiempos respectivos en lo que se llama el {\it Plan de
Trabajo}. Ello nos da una idea clara de la extensión en tiempo del trabajo
propuesto. Además de ser necesario, lo cual es normalmente cierto en
propuestas de proyecto industrial, es importante establecer el PLAN FINANCIERO
el cual desglosa los recursos necesarios para el desarrollo del proyecto y sus
costos.

{\bf Un ejemplo de Plan de Trabajo}

La figura \ref{ttphd} presenta el cronograma de las actividades que llevarán a
cabo los objetivos de esta tesis. 

\begin{figure}[th]
	\centerline{\includegraphics[width=4in,height=3in]{plots/ttphd.pdf}}
	\caption{Cronograma de Actividades para desarrollar la Tesis}
	\label{ttphd}
\end{figure}
\end{comment}


% Bibliography
\bibliographystyle{apalike}
\bibliography{bibliography}

\end{document} 
